
缓冲区是 I/O 操作期间用于传输数据的连续内存区域。

Boost.Asio 定义了两种类型的缓冲区:可变缓冲区 (boost::asio::mutable\_buffer),其中可以写入数据;常量缓冲区 (boost::asio::const\_buffers),用于创建只读缓冲区。可变缓冲区可以转换为常量缓冲区,但不能转换为常量缓冲区。这两种类型的缓冲区都提供溢出保护。

还有 boost::buffer 函数来帮助从不同数据类型(指向原始内存和大小的指针、字符串 (std::string) 或普通旧数据 (POD) 结构数组或向量(意味着类型、结构或类没有用户定义的复制赋值运算符或析构函数,也没有私有或受保护的非静态数据成员)创建可变或常量缓冲区。例如,要从字符数组创建缓冲区。

可以使用以下方式创建:

\begin{cpp}
char data[1024];
mutable_buffer buffer = buffer(data, sizeof(data));
\end{cpp}

另请注意,缓冲区的所有权和生存期是程序的责任,而不是 Boost.Asio 库的责任。

\mySubsubsection{9.7.1.}{分散-聚集操作}

可以通过使用分散-集中操作来高效使用缓冲区,其中多个缓冲区一起用于接收数据(分散- 读取)或发送数据(集中-写入)。

分散-读取:是将数据从唯一源读取到不同的不连续内存缓冲区的过程。

聚集-写入:是逆向的过程;数据从不同的不连续的内存缓冲区聚集并写入单个目的地。

这些技术可以减少系统调用或数据复制的次数,从而提高效率和性能。不仅用于 I/O 操作,还可用于其他用例,例如:数据处理、机器学习或并行算法(例如排序或矩阵乘法)。

为了允许分散-集中操作,可以将多个缓冲区一起传递到容器(std::vector、 std::list、 std::array 或 boost::array)内的异步操作。

下面是一个散射读取的例子,其中套接字将一些数据异步读入 buf1 和 buf2 缓冲区:

\begin{cpp}
std::array<char, 128> buf1, buf2;
std::vector<boost::asio::mutable_buffer> buffers = {
    boost::asio::buffer(buf1),
    boost::asio::buffer(buf2)
};
socket.async_read_some(buffers, handler);
\end{cpp}

以下是如何实现聚集读取:

\begin{cpp}
std::array<char, 128> buf1, buf2;
std::vector<boost::asio::const_buffer> buffers = {
    boost::asio::buffer(buf1),
    boost::asio::buffer(buf2)
};
socket.async_write_some(buffers, handler);
\end{cpp}

套接字执行相反的操作,将两个缓冲区中的一些数据写入套接字缓冲区以进行异步发送。

\mySubsubsection{9.7.2.}{流缓冲区}

还可以使用流缓冲区来管理数据。流缓冲区由 boost::asio::basic\_streambuf 类定义,该类基于 std::basic\_streambuf类,并在 <streambuf> 头文件中定义。允许动态缓冲区,其大小可以适应正在传输的数据量。

以下示例中看看流缓冲区如何与分散-集中操作协同工作。本例中,实现了一个 TCP 服务器,该服务器监听并接受来自给定端口的客户端连接,将客户端发送的消息读入两个流缓冲区,并将其内容打印到控制台。由于这里有兴趣了解流缓冲区和分散-集中操作,可通过使用同步操作来简化示例。

与上例一样,在 main() 函数中,使用 boost::asio::ip::tcp::acceptor 对象来设置 TCP 服务器用于接受连接的协议和端口。然后,在无限循环中,服务器使用该 acceptor 对象连接 TCP 套接字 (boost::asio::ip::tcp::socket) 并调用 handle\_client() 函数:

\begin{cpp}
#include <array>
#include <iostream>
#include <boost/asio.hpp>
#include <boost/asio/streambuf.hpp>

using boost::asio::ip::tcp;

constexpr int port = 1234;

int main() {
    try {
        boost::asio::io_context io_context;
        tcp::acceptor acceptor(io_context,
                        tcp::endpoint(tcp::v4(), port));
        std::cout << "Server is running on port "
                  << port << "...\n";

        while (true) {
            tcp::socket socket(io_context);
            acceptor.accept(socket);
            std::cout << "Client connected...\n";

            handle_client(socket);
            std::cout << "Client disconnected...\n";
        }
    } catch (std::exception& e) {
        std::cerr << "Exception: " << e.what() << '\n';
    }
    return 0;
}
\end{cpp}

handle\_client() 函数创建两个流缓冲区: buf1 和 buf2,并将它们添加到容器(在本例中为 std ::array)中,以用于分散-聚集操作。

然后,调用套接字的同步 read\_some() 函数。此函数返回从套接字读取的字节数并将它们复制到缓冲区中。如果套接字连接出现问题,则会在错误代码对象中返回错误,服务器将输出错误消息并退出。

实现如下:

\begin{cpp}
void handle_client(tcp::socket& socket) {
    const size_t size_buffer = 5;
    boost::asio::streambuf buf1, buf2;

    std::array<boost::asio::mutable_buffer, 2> buffers = {
        buf1.prepare(size_buffer),
        buf2.prepare(size_buffer)
    };

    boost::system::error_code ec;
    size_t bytes_recv = socket.read_some(buffers, ec);
    if (ec) {
        std::cerr << "Error on receive: "
                  << ec.message() << '\n';

        return;
    }

    std::cout << "Received " << bytes_recv << " bytes\n";

    buf1.commit(5);
    buf2.commit(5);

    std::istream is1(&buf1);
    std::istream is2(&buf2);
    std::string data1, data2;
    is1 >> data1;
    is2 >> data2;

    std::cout << "Buffer 1: " << data1 << std::endl;
    std::cout << "Buffer 2: " << data2 << std::endl;
}
\end{cpp}

如果没有错误,则使用流缓冲区的 commit() 函数将五个字节传输到流缓冲区 buf1 和 buf2 中。使用 std::istream 对象提取这些缓冲区的内容并将其输出到控制台。

要执行此示例,需要打开两个终端。在一个终端中,执行服务器,在另一个终端中,执行 telnet 命令,如前所示。在 telnet 终端中,可以输入一条消息(例如, "Hello World")。此消息将发送到服务器。然后服务器终端将显示以下内容:

\begin{shell}
Server is running on port 1234...
Client connected...
Received 10 bytes
Buffer 1: Hello
Buffer 2: Worl
Client disconnected...
\end{shell}

可以看到,只有 10 个字节被处理并分配到两个缓冲区中。两个单词之间的空格字符被处理但在 iostream 对象解析输入时丢弃。

当传入数据的大小可变且事先未知时,流缓冲区非常有用。这些类型的缓冲区可以与固定大小的缓冲区一起使用。



























