
std::async 是 C++ 中的一个函数模板,由 C++ 标准在 <future> 标头中引入,作为 C++11 线程支持库的一部分,用于异步运行函数,允许主线程(或其他线程)继续并发运行。

 std::async 是 C++ 中用于异步编程的强大工具,可以更轻松地并行运行任务并有效地管理其结果。

\mySubsubsection{7.2.1.}{启动异步任务}

要使用 std::async 异步执行函数,我们可以使用与第 3 章启动线程时相同的方法,但使用不同的可调用对象。

一种方法是使用函数指针:

\begin{cpp}
void func() {
    std::cout << "Using function pointer\n";
}
auto fut1 = std::async(func);
\end{cpp}

另一种方法是使用 lambda 函数:

\begin{cpp}
auto lambda_func = []() {
    std::cout << "Using lambda function\n";
};
auto fut2 = std::async(lambda_func);
\end{cpp}

还可以使用嵌入的 lambda 函数:

\begin{cpp}
auto fut3 = std::async([]() {
    std::cout << "Using embedded lambda function\n";
});
\end{cpp}

可以使用 operator() 重载的函数对象:

\begin{cpp}
class FuncObjectClass {
    public:
    void operator()() {
        std::cout << "Using function object class\n";
    }
};
auto fut4 = std::async(FuncObjectClass());
\end{cpp}

可以使用非静态成员函数,通过传递成员函数的地址和对象的地址来调用成员函数:

\begin{cpp}
class Obj {
    public:
    void func() {
        std::cout << "Using a non-static member function"
        << std::endl;
    }
};
Obj obj;
auto fut5 = std::async(&Obj::func, &obj);
\end{cpp}

还可以使用静态成员函数,由于方法是静态的,只需要成员函数的地址:

\begin{cpp}
class Obj {
    public:
    static void static_func() {
        std::cout << "Using a static member function"
        << std::endl;
    }
};
auto fut6 = std::async(&Obj::static_func);
\end{cpp}

当调用 std::async 时,会返回一个未来,其中会存储函数的结果。

\mySubsubsection{7.2.2.}{传递值}

同样,与创建线程时传递参数类似,参数可以通过值、引用或指针传递给线程。

这里,可以看到如何通过值传递参数:

\begin{cpp}
void funcByValue(const std::string& str, int val) {
    std::cout << "str: " << str << ", val: " << val
              << std::endl;
}
std::string str{"Passing by value"};
auto fut1 = async(funcByValue, str, 1);
\end{cpp}

按值传递表明可创建一个临时对象并将参数值复制到其中。这可以避免数据竞争,但成本更高。

下一个示例显示如何通过引用传递值:

\begin{cpp}
void modifyValues(std::string& str) {
    str += " (Thread)";
}
std::string str{"Passing by reference"};
auto fut2 = std::async(modifyValues, std::ref(str));
\end{cpp}

还可以将值作为 const 引用传递:

\begin{cpp}
void printVector(const std::vector<int>& v) {
    std::cout << "Vector: ";
    for (int num : v) {
        std::cout << num << " ";
    }
    std::cout << std::endl;
}
std::vector<int> v{1, 2, 3, 4, 5};
auto fut3 = std::async(printVector, std::cref(v));
\end{cpp}

引用传递通过使用 std::ref()(非常量引用)或 std::cref()(常量引用)来实现,这两个函数均定义在 <functional> 头文件中,让定义线程构造函数的可变参数模板(支持任意数量参数的类或函数模板)将参数视为引用。

还可以将对象移动到由 std::async 创建的线程中:

\begin{cpp}
auto fut4 = std::async(printVector, std::move(v));
\end{cpp}

注意,向量数组v在移动后,处于有效的空状态。

最后,还可以通过 lambda 捕获传递值:

\begin{cpp}
std::string str5{"Hello"};
auto fut5 = std::async([&]() {
    std::cout << "str: " << str5 << std::endl;
});
\end{cpp}

此示例中,std::async 执行的 lambda 函数作为引用访问了str 变量。

\mySubsubsection{7.2.3.}{返回值}

当调用 std:async 时,会立即返回一个Future,将保存函数或可调用对象将计算的值。

前面的例子中,没有使用 std::async 返回的对象。这里重写第 6 章打包任务部分的最后一个示例,使用 std::packaged\_task 对象来计算两个值的幂。本例中,将使用 std::async 生成几个异步任务来计算这些值,等待任务完成存储结果,最后在控制台中显示:

\begin{cpp}
#include <chrono>
#include <cmath>
#include <future>
#include <iostream>
#include <thread>
#include <vector>
#include <syncstream>

#define sync_cout std::osyncstream(std::cout)

using namespace std::chrono_literals;

int compute(unsigned taskId, int x, int y) {
    std::this_thread::sleep_for(std::chrono::milliseconds(
                                rand() % 200));
    sync_cout << "Running task " << taskId << '\n';
    return std::pow(x, y);
}

int main() {
    std::vector<std::future<int>> futVec;
    for (int i = 0; i <= 10; i++)
        futVec.emplace_back(std::async(compute,
                            i+1, 2, i));

    sync_cout << "Waiting in main thread\n";
    std::this_thread::sleep_for(1s);

    std::vector<int> results;
    for (auto& fut : futVec)
        results.push_back(fut.get());

    for (auto& res : results)
        std::cout << res << ' ';
    std::cout << std::endl;
    return 0;
}
\end{cpp}

compute() 函数仅获取两个数字 $x$ 和 $y$,并计算 $x^y$ 。需要获取一个代表任务标识符的数字,并等待最多两秒钟,然后在控制台中打印消息并计算结果。

在 main() 函数中,主线程启动几个任务来计算一系列 2 的幂值。调用 std::async 返回的 Future 存储在 futVec 向量中。主线程等待一秒钟后,模拟一些工作。最后,遍历 futVec 并在每个 Future 元素中调用 get() 函数,从而等待该特定任务完成并返回一个值,然后将返回的值存储在另一个名为 results 的向量数组中。然后,在退出程序之前输出 results 向量数组的内容。

这是运行该程序时的输出:

\begin{shell}
Waiting in main thread
Running task 11
Running task 9
Running task 2
Running task 8
Running task 4
Running task 6
Running task 10
Running task 3
Running task 1
Running task 7
Running task 5
1 2 4 8 16 32 64 128 256 512 1024
\end{shell}

每个任务完成所需的时间不同,输出不是按任务标识符排序的。但当按顺序遍历 futVec 向量数组获取结果时,这些结果会按顺序显示。

现在,已经了解了如何启动异步任务并传递参数和返回值。接下来,介绍如何使用启动策略来控制执行方法。



























