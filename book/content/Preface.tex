

异步编程是构建高效、响应迅速且高性能软件的必备实践,尤其是在当今的多核处理器和实时数据处理领域。本书深入探讨了掌握 C++ 中异步编程的原理和实用技巧,为您提供处理从线程管理到性能优化等所有事务所需的知识。

开发异步软件有几个关键支柱

\begin{itemize}
\item
线程管理和同步

\item
异步编程概念、模型和库

\item
调试、测试和优化多线程和异步软件
\end{itemize}

虽然许多资源都侧重于并行编程或通用软件开发的基础知识,但本书旨在全面探索这些支
柱。它涵盖了管理并发、调试复杂系统和优化软件性能的基本技术,同时将这些概念应用
于实际场景。

我们将根据以下内容通过实际示例指导您了解异步编程的各个方面:

\begin{itemize}
\item
我们开发高性能软件的丰富经验

\item
从金融到研究中心等不同行业的工作中学习到的最佳实践
\end{itemize}

随着多核处理器和并行计算架构越来越成为现代应用程序不可或缺的一部分,对异步编程专业知识的需求正在迅速增长。掌握本书中介绍的技术不仅可以帮助您应对当今复杂的软件开发挑战,还可以为性能关键型软件的未来发展做好准备。

无论您正在使用低延迟金融系统、开发高吞吐量应用程序,还是只是想提高编程技能,本书都将为您提供成功的工具和知识。

\mySubsectionNoFile{}{适读人群}

本书面向希望使用最新 C++ 版本加深对异步编程的理解并优化软件性能的软件工程师、开发人员和技术主管。主要目标受众包括:

\begin{itemize}
\item
软件工程师:希望提高 C++ 技能并获得多线程和异步编程、调试和性能优化方面的实用见解的人。

\item
技术领导:旨在实施高效异步系统的领导者将找到管理复杂软件开发和提高团队生产力的策略和最佳实践。

\item
学生和爱好者:渴望了解高性能计算和异步编程的个人将受益于全面的解释和示例,帮助他们在技术职业生涯中取得进步。
\end{itemize}

本书将帮助读者应对现实世界的挑战并在技术面试中脱颖而出,并为他们提供在当今快节奏的软件领域蓬勃发展的知识。

\mySubsectionNoFile{}{关于本书}

第 1 章,\textit{并行编程范式},探讨了构建并行系统的不同架构和模型,以及各种并行编程范式及其性能指标。

第 2 章,\textit{进程、线程和服务},深入研究操作系统中的进程,研究其生命周期、进程间通信以及线程的作用,包括守护进程和多线程。

第 3 章,\textit{如何在 C++ 中创建和管理线程},指导如何创建和管理线程、传递参数、检索结果以及处理异常以确保在多线程环境中高效执行。

第 4 章,\textit{使用锁进行线程同步},解释了 C++ 标准库同步原语(包括互斥锁和条件变量)的使用,同时解决了竞争条件、死锁和活锁。

第 5 章,\textit{原子操作},探讨了 C++ 原子类型、内存模型以及如何实现基本的 SPSC 无锁队列,为未来的性能增强做好准备。

第 6 章,\textit{Promise和Future},介绍了异步编程概念,包括承诺、未来和打包任务,并展示了如何使用这些工具解决实际问题。

第 7 章,\textit{异步函数},探讨了 std::async 用于执行异步任务、定义启动策略、处理异常和优化性能的功能。

第 8 章,\textit{使用协程},描述了 C++ 协程、它们的基本要求以及如何实现生成器和解析器,同时处理协程内的异常。

第 9 章,\textit{使用 Boost.Asio 进行异步编程},讲解了如何使用 Boost.Asio 管理与外部资源相关的异步任务,重点关注 I/O 对象、执行上下文和事件处理。

第 10 章,\textit{使用 Boost.Cobalt 实现协程},探索使用 Boost.Cobalt 库轻松实现协程,避免低级复杂性并专注于函数式编程需求。

第 11 章,\textit{记录和调试异步软件},解释了如何有效地使用日志和调试工具来识别和解决异步应用程序中的问题,包括死锁和竞争条件。

第 12 章,\textit{消杀和测试异步软件},探讨了如何使用清理器对多线程代码进行清理,并探讨了使用 GoogleTest 库为异步软件量身定制的测试技术。

第 13 章,\textit{提高异步软件性能},研究性能测量工具和技术,包括高分辨率计时器、缓存优化以及避免虚假和真实共享的策略。


\mySubsectionNoFile{}{编译环境}

您需要具有使用 C++ 进行编程的经验,以及如何使用调试器查找错误。由于我们使用 C++2 0 功能,并且在某些示例中使用 C++23,因此您需要安装 GCC 14 和 Clang 18。所有源代码示例均已在 Ubuntu 和 macOS 中进行了测试,但由于它们与平台无关,因此它们应该可以在任何平台上编译和运行。

% Please add the following required packages to your document preamble:
% \usepackage{longtable}
% Note: It may be necessary to compile the document several times to get a multi-page table to line up properly
\begin{longtable}{|l|l|}
\hline
\textbf{书中涉及的软件/硬件} & \textbf{操作系统}                \\ \hline
\endfirsthead
%
\endhead
%
C++20 和 C++23       & Linux (测试过Ubuntu 24.04)      \\ \hline
GCC 14.2            & macOS (测试过macOS Sonoma 14.x) \\ \hline
Clang 18            & Windows 11                   \\ \hline
Boost 1.86          &                              \\ \hline
GDB 15.1            &                              \\ \hline
\end{longtable}

每章都包含一个\textit{环境搭建}部分,重点介绍如何安装编译本章示例所需的工具和库的相关信息。

\textbf{如果你正在使用这本书的数字版本,我们建议自己输入代码或通过 GitHub 库访问代码 (链接在下一节中提供)。这样做将帮助您避免与复制和粘贴代码相关的任何潜在错误。}

\mySubsectionNoFile{}{源码下载}

本书的代码托管在 GitHub 上,地址为 \url{https://github.com/PacktPublishing/Asynchronous-Programming-with-CPP}。此外,还可以在 \url{https://github.com/PacktPublishing/} 浏览图书和视频目录中的其他代码包。欢迎查看!
