
本节中,将实现最后一个例子:一个简单的字符串解析器。协程将等待输入(一个 std::string 对象),并在解析输入字符串后产生输出(一个数字)。为了简化示例,假设数字的字符串表示没有错误,并且数字的结尾由井号 \# 表示;还假设数字类型为 i nt64\_t,并且字符串不包含该整数类型范围之外的值。

\mySubsubsection{8.6.1.}{解析算法}

看看如何将表示整数的字符串转换为数字。例如,字符串“-12321\#”表示数字 -1232 1。要将字符串转换为数字,可以编写如下函数:

\begin{cpp}
int64_t parse_string(const std::string& str) {
    int64_t num{ 0 };
    int64_t sign { 1 };

    std::size_t c = 0;
    while (c < str.size()) {
        if (str[c] == '-') {
            sign = -1;
        }
        else if (std::isdigit(str[c])) {
            num = num * 10 + (str[c] - '0');
        }
        else if (str[c] == '#') {
            break;
        }
        ++c;
    }
    return num * sign;
}
\end{cpp}

由于假设字符串格式正确,因此代码非常简单。如果读取减号 -,则将其更改为 -1(默认情况下,假设为正数,如果有 + 符号,则将其忽略)。然后,逐个读取数字,并按如下方式计算数值。

num 的初始值是 0。读取第一位数字,并将其数值添加到当前 num 值乘以 10 之后。这就是读取数字的方式:最左边的数字将乘以 10,次数与其右边的数字数量相同。

使用字符来表示数字时,可根据 ASCII 表示具有一些值(我们假设不使用宽字符或任何其他字符类型)。字符 0 到 9 具有连续的 ASCII 码,因此只需减去 0 即可将它们转换为数字。

即使对于上述代码,最后一个字符检查不是必需的,但还是在此处包含了它。当解析器例程找到 \# 字符时,它会终止解析循环并返回最终的数字值。

我们可以使用此函数来解析任何字符串并获取数字值,但需要完整的字符串才能将其转换为数字。

考虑一下这个场景:从网络连接接收到字符串,需要解析它并将其转换为数字。可以将字符保存到临时字符串中,然后调用前面的函数。

但还有另一个问题:如果字符到达速度很慢,比如每隔几秒才到达一次,因为这就是字符的传输方式,那该怎么办?希望让 CPU 保持忙碌,如果可能的话,在等待每个字符到达时执行一些其他任务(或任务)。

解决这个问题的方法有很多种。可以创建一个线程并同时处理字符串,但对于这样一个简单的任务来说,这会浪费大量的计算机时间,也可以使用 std::async。

\mySubsubsection{8.6.2.}{解析协程}

本章将使用协程,因此将使用 C++ 协程实现字符串解析器。不需要多余的线程,而且由于协程的异步特性,在字符到达时执行其他处理都非常容易。

解析协程所需的样板代码,与前面的示例中已经看到的代码几乎相同。解析器本身则完全不同。请参阅以下代码:

\begin{cpp}
async_parse<int64_t, char> parse_string() {
    while (true) {
        char c = co_await char{ };
        int64_t number { };
        int64_t sign { 1 };

        if (c != '-' && c != '+' && !std::isdigit(c)) {
            continue;
        }
        if (c == '-') {
            sign = -1;
        }
        else if (std::isdigit(c)) {
            number = number * 10 + c - '0';
        }

        while (true) {
            c = co_await char{};
            if (std::isdigit(c)) {
                number = number * 10 + c - '0';
            }
            else {
                break;
            }
        }
        co_yield number * sign;
    }
}
\end{cpp}

现在可以轻松识别返回类型(async\_parse<int64\_t、 char>),并且解析器协程会暂停自身以等待输入字符。解析完成后,协程将在产生数字后暂停自身。

但也可以看到,前面的代码并不像第一次尝试将字符串解析为数字那么简单。

首先,解析器协程会逐个解析字符,不会获取完整的字符串进行解析,因此会无限循环while (true)。不知道完整字符串中有多少个字符,因此需要继续接收和解析。

外循环意味着协程将随着字符的到来而一个接一个地解析数字——永远如此。但它会暂停自身以等待字符,这样就不会浪费 CPU 时间。

现在,一个字符到达。首先,检查它是否是我们数字的有效字符。如果该字符不是减号 -、加号 + 或数字,则解析器等待下一个字符。

如果下一个字符是有效字符,则适用以下情况:

\begin{itemize}
\item
如果是减号,将符号值改为 -1

\item
如果是加号,忽略

\item
如果是数字,将其解析为数字,并使用与解析器的第一个版本中看到的相同方法更新当前数字值
\end{itemize}

在第一个有效字符之后,进入一个新的循环来接收其余的字符,无论是数字还是分隔符 (\#)。注意,当说有效字符时,指的是适合数字转换。仍然假设输入的字符形成一个正确终止的有效数字。

一旦数字转换,协程就会将其返回,然后再次执行外循环。这里需要终止字符,输入字符流理论上是无穷无尽的,并且可以包含许多数字。

协程其余部分的代码可以在 GitHub 中找到,遵循与其他协程相同的约定。首先,定义返回类型:

\begin{cpp}
template <typename Out, typename In>
struct async_parse {
    // …
};
\end{cpp}

使用模板来提高灵活性,其允许参数化输入和输出数据类型。本例中,这些类型分别是 int64\_t 和 char。

输入和输出数据项如下:

\begin{cpp}
std::optional<In> input_data { };
Out output_data { };
\end{cpp}

对于输入,可使用 std::optional<In>,因为需要一种方法来知道我们是否收到了一个字符。这里,使用 put() 函数将字符发送到解析器:

\begin{cpp}
void put(char c) {
    handle.promise().input_data = c;
    if (!handle.done()) {
        handle.resume();
    }
}
\end{cpp}

此函数仅将值赋给 std::optional input\_data 变量。为了管理字符的等待,实现了以下 awaiter 类型:

\begin{cpp}
auto await_transform(char) noexcept {
    struct awaiter {
        promise_type& promise;

        [[nodiscard]] bool await_ready() const noexcept {
            return promise.input_data.has_value();
        }

        [[nodiscard]] char await_resume() const noexcept {
            assert (promise.input_data.has_value());
            return *std::exchange(
                            promise.input_data,
                            std::nullopt);
        }

        void await_suspend(std::coroutine_handle<
                           promise_type>) const noexcept {
        }
    };
    return awaiter(*this);
}
\end{cpp}

awaiter 结构实现了两个函数来处理输入数据:

\begin{itemize}
\item
await\_ready():如果可选的 input\_data 变量包含有效值,则返回 true。否则返回 false。

\item
await\_resume():返回存储在可选 input\_data 变量中的值并将其清空,并将其分配给 std::nullopt。
\end{itemize}

本节中,了解了如何使用 C++ 协程实现一个简单的解析器。这是最后一个例子,说明了使用协程实现的一个非常基本的流处理功能。下一节中,将介绍协程中的异常。











