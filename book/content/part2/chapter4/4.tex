
上一节中,C++ 标准库提供的不同类型的互斥锁。本节中,将介绍提供的类,以便更轻松地使用互斥锁。这是通过使用不同的包装器类来实现的。下表总结了锁管理类及其主要功能:

% Please add the following required packages to your document preamble:
% \usepackage{longtable}
% Note: It may be necessary to compile the document several times to get a multi-page table to line up properly
\begin{longtable}{|l|l|l|}
\hline
\textbf{互斥管理器类} & \textbf{支持的互斥锁类型}                                                          & \textbf{互斥对象管理} \\ \hline
\endfirsthead
%
\endhead
%
std::lock\_guard  & 所有类型 & 1            \\ \hline
std::scoped\_lock & 所有类型 & 零或更多 \\ \hline
std::unique\_lock & 所有类型 & 1            \\ \hline
std::shared\_lock            & \begin{tabular}[c]{@{}l@{}}std::shared\_mutex \\ std::shared\_timed\_mutex\end{tabular} & 1                        \\ \hline
\end{longtable}

\begin{center}
表 4.3:锁管理类别及其特性
\end{center}

\mySubsubsection{4.4.1.}{std::lock\_guard}

std::lock\_guard 类是一个资源获取即初始化 (RAII) 类,使使用互斥锁更加容易,并保证在调用 lock\_guard 析构函数时释放互斥锁。这在处理异常等情况下非常有用。

以下代码展示了 std::lock\_guard 的用法,以及当已获取锁时如何更容易地处理异常:

\begin{cpp}
#include <format>
#include <iostream>
#include <mutex>
#include <thread>

std::mutex mtx;
uint32_t counter{};

void function_throws() { throw std::runtime_error("Error"); }

int main() {
    auto worker = [] {
        for (int i = 0; i < 1000000; ++i) {
            mtx.lock();
            counter++;
            mtx.unlock();
        }
    };

    auto worker_exceptions = [] {
        for (int i = 0; i < 1000000; ++i) {
            try {
                std::lock_guard<std::mutex> lock(mtx);
                counter++;
                function_throws();
            } catch (std::system_error& e) {
                std::cout << e.what() << std::endl;
                return;
            } catch (...) {
                return;
            }
        }
    };

    std::thread t1(worker_exceptions);
    std::thread t2(worker);

    t1.join();
    t2.join();

    std::cout << "Final counter value: " << counter << std::endl;
}
\end{cpp}

function\_throws() 函数只是一个会引发异常的实用函数。

代码示例中, worker\_exceptions() 函数由 t1 执行。这种情况下,处理异常以打印有意义的消息。未显式获取/释放锁。这委托给 lock,一个 std::lock\_guard 对象。构造锁时,会包装互斥锁并调用 mtx.lock(),获取锁。当锁销毁时,互斥锁会自动释放。如果发生异常,互斥锁也将被释放,因为已退出定义锁的范围。

为 std::lock\_guard 实现了另一个构造函数,接收 std::adopt\_lock\_t 类型的参数。基本上,这个构造函数可以包装已经获取的非共享互斥锁,将在 std::lock\_guard 析构函数中自动释放。

\mySubsubsection{4.4.2.}{std::unique\_lock}

std::lock\_guard 类只是一个简单的 std::mutex 包装器,自动在其构造函数中获取互斥锁(线程将阻塞,等待另一个线程释放互斥锁)并在其析构函数中释放互斥锁。这非常有用,但有时需要更多控制。例如, std::lock\_guard 将在互斥锁上调用 lock() 或假定互斥锁已被获取。可能更倾向于或确实需要调用 try\_lock,可能还希望 std::mutex 包装器不要在其构造函数中获取锁;所以希望将锁定推迟到稍后。所有这些功能均能由 std::unique\_lock 实现。

std::unique\_lock 构造函数接受一个标签作为其第二个参数,以指示要对底层互斥锁执行的操作。这里有三个选项

\begin{itemize}
\item
std::defer\_lock:不获取互斥锁的所有权。互斥锁在构造函数中未锁定,如果从未获取过,则不会在析构函数中解锁。

\item
std::adopt\_lock:假设调用线程已获取互斥锁。将在析构函数中释放。此选项也适用于std::lock\_guard。

\item
std::try\_to\_lock:尝试在不阻塞的情况下获取互斥锁。
\end{itemize}

如果仅将互斥锁作为唯一参数传递给 std::unique\_lock 构造函数,其行为与 std::lock\_guard 中的行为相同:其将阻塞直到互斥锁可用,然后获取锁,并将在析构函数中释放锁。

std::unique\_lock 类与 std::lock\_guard 不同,它允许调用 lock() 和 unlock() 来分别获取和释放互斥锁。

\mySubsubsection{4.4.3.}{std::scoped\_lock}

std::scoped\_lock 类与 std::unique\_lock 一样,是实现 RAII 机制的 std::mutex 包装器(如果获取了互斥锁,则会在析构函数中释放)。主要区别在于,std::unique\_lock 只包装一个互斥锁,而 std::scoped\_lock 包装零个或多个互斥锁。此外,互斥锁的获取顺序与传递给 std::scoped\_lock 构造函数的顺序相同,可以避免死锁。

来看看下面的代码:

\begin{cpp}
std::mutex mtx1;
std::mutex mtx2;

// Acquire both mutexes avoiding deadlock
std::scoped_lock lock(mtx1, mtx2);

// Same as doing this
// std::lock(mtx1, mtx2);
// std::lock_guard<std::mutex> lock1(mtx1, std::adopt_lock);
// std::lock_guard<std::mutex> lock2(mtx2, std::adopt_lock);
\end{cpp}

前面的代码片段展示了如何使用两个互斥锁。

\mySubsubsection{4.4.4.}{std::shared\_lock}

std::shared\_lock 类是另一个通用互斥锁所有权包装器。与 std::unique\_lock 和 std::scoped\_lock 一样,允许延迟锁定和转移锁所有权。 std::unique\_lock 和 std::shared\_lock 之间的主要区别在于,后者用于在共享模式下获取/释放包装的互斥锁,而前者用于在独占模式下执行相同操作。

本节中,介绍了互斥包装类及其主要功能。接下来,将介绍另一种同步机制: 条件变量。













