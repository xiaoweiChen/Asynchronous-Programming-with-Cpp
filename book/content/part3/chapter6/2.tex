
Future 是一个对象,表示一些未确定的结果,这些结果将在未来某个时间完成。 Promise 是该结果的提供者。

自版本 C++11 以来,Promise和Future一直是 C++ 标准的一部分,可以通过包含 <future> 头文件、通过类 std::promise 获得Promise,以及通过类 std::future 获得Future来。

std::promise 和 std::future 对实现了一次性生产者-消费者通道,其中 Promise 为生产者, Future 为消费者。消费者 (std::future) 可以阻塞,直到生产者 (std::promise) 的结果可用为止。

\myGraphic{0.8}{content/part3/chapter6/images/1.png}{图 6.1 – {} Promise-Future 沟通渠道}

许多现代编程语言都提供了类似的异步方法,例如 Python(带有 asyncio 库)、 Scala(scala.concurrent 库中)、 JavaScript(核心库)、 Rust(标准库 (std) 或诸如 prospector\_future 之类的包中)、 Swift(Combine 框架中)和 Kotlin 等。

使用 Promise 和 Future 实现异步执行的基本原理是,想要运行以生成结果的函数在后台执行,使用新线程或当前线程,初始线程使用 Future 对象来检索函数计算的结果。函数完成时将存储此结果值,同时将使用 Future 对象作为占位符。异步函数将使用 Promise 对象将结果存储在 Future 中,而无需在初始线程和后台线程之间建立显式同步机制。当初始线程需要该值时,将从 Future 对象中检索该值。如果该值仍未准备好,则阻止执行初始线程,直到 Future 准备好为止。

有了这个思路,让一个函数异步运行就变得很简单了。只要知道这个函数可以在一个单独的线程上运行,需要避免数据竞争,但线程之间的结果通信和同步由 Promise-Future 对管理。

使用Promise 和 Future可以通过卸载计算来提高响应能力,并提供一种与线程和回调相比处理异步操作的结构化方法。

现在来了解一下这两个对象。

\mySubsubsection{6.2.1.}{Promise}

Promise在 <future> 头文件中定义为 std::promise。

通过Promise,达成了一项协议,即结果将在未来的某个时间可用。这样,就可以让后台任务完成其工作并计算结果。同时,主线程也将继续执行其任务,并在需要结果时请求它。那时,结果可能已经准备好了。

此外,Promise可以传达是否引发异常而不是返回有效值,并且它们将确保其生命周期持续到线程完成并将结果写入其中。

因此, Promise 是一种存储结果(值或异常)的工具,稍后可通过 Future 异步获取结果。 Promise 对象仅供使用一次,之后无法修改。

除了结果之外,每个 Promise 还拥有一个共享状态。共享状态是一个存储完成状态、同步机制和指向结果的指针的内存区域。允许 Promise 存储结果或异常、在完成时发出信号以及允许 Future 访问结果(如果 Promise 尚未准备好则阻塞)来确保 Promise 和 Future 之间的正确通信和同步。 Promise 可以使用以下操作更新其共享状态:

\begin{itemize}
\item
准备就绪: Promise 将结果存储在共享状态中,并使 Promise 的状态变为就绪状态,解除等待与 Promise 关联的 Future 线程的阻塞。记住,结果可以是值(甚至是 void)或异常。

\item
释放: Promise 释放对共享状态的引用,如果这是最后一个引用,则该引用将销毁。此内存释放机制类似于共享指针,及其控制块所使用的机制。除非共享状态由 std::async 创建且尚未处于就绪状态,否则此操作不会阻塞。

\item
放弃: Promise 中存储了一个类型为 std::future\_error 的异常,错误代码为 std::future\_errc ::broken\_promise,使共享状态处于准备状态,然后释放它。
\end{itemize}

可以使用其默认构造函数或自定义分配器来构造 std::promise 对象。在这两种情况下,都会创建一个具有空共享状态的新Promise。还可以使用移动构造函数来构造Promise,新Promise将具有另一个Promise拥有的共享状态。初始Promise没有共享状态。

移动 Promise 在与资源管理、通过避免副本进行优化,以及保持正确的所有权语义相关的场景中很有用;例如,当 Promise 需要在另一个线程中完成、存储在容器中、返回给 API 调用的调用者或发送到回调处理程序。

Promise无法复制(其复制构造函数或复制赋值运算符被删除),避免两个Promise对象共享相同的共享状态,并在结果存储在共享状态时存在数据竞争的风险。

Promise可以移动,同样也可以交换。标准模板库 (STL) 中的 std::swap 函数具有针对承诺的模板特化。

当一个Promise对象删除时,关联的Future仍将有权访问共享状态。 如果在Promise设置值之后发生删除,则共享状态将处于释放模式,因此Future可以访问结果并使用。 但如果在设置结果值之前删除了Promise,则共享状态将移至放弃状态,并且Future在尝试获取结果时将获得 std::future\_errc::broken\_promise。

可以使用 std::promise 函数 set\_value() 设置值,使用 set\_exception() 函数设置异常。结果以原子方式存储在Promise的共享状态中,使其状态准备就绪。来看一个例子:

\begin{cpp}
auto threadFunc = [](std::promise<int> prom) {
    try {
        int result = func();
        prom.set_value(result);
    } catch (...) {
        prom.set_exception(std::current_exception());
    }
};

std::promise<int> prom;
std::jthread t(threadFunc, std::move(prom));
\end{cpp}

创建prom 并作为参数移入 threadFunc lambda 函数中。由于Promise不可复制,需要使用按值传递并将Promise移入参数以避免复制。

在 lambda 函数内部,调用 func() 函数,并使用 set\_value() 将其结果存储在 promise 中。如果 func() 抛出异常,则会使用 set\_exception() 捕获该异常并将其存储在 promise 中,可以使用 Future 在调用线程中提取此结果(值或异常)。

C++14 中,还可以使用广义 lambda 捕获将 Promise 传递到 lambda 捕获中:

\begin{cpp}
using namespace std::literals;
std::promise<std::string> prom;
auto t = std::jthread([prm = std::move(prom)] mutable {
    std::this_thread::sleep_for(100ms);
    prm.set_value("Value successfully set");
});
\end{cpp}

\verb|prm = std::move(prom)| 将外部prom 移至 lambda 的内部prm。默认情况下,参数被捕获为常量,因此需要将 lambda 指定为可变的,以允许修改 prm 。

如果Promise没有共享状态(错误代码设置为 no\_state)或者共享状态已经有存储的结果(错误代码设置为 promise\_already\_satisfied),则 set\_value() 会抛出 std::future\_error 异常。

set\_value() 也可以在不指定值的情况下使用。这种情况下,只是使状态准备就绪,这可以用作栅栏。

图 6.2 显示了表示不同共享状态转换的图表。

\myGraphic{1.0}{content/part3/chapter6/images/2.png}{图 6.2 – {} Promise 共享状态转换图}

还有两个函数可以设置 Promise 的值,即 set\_value\_at\_thread\_exit 和 set\_exception\_at\_thread\_exit。与以前一样,结果会立即存储,但使用这些新函数时,状态尚未准备就绪。当线程退出且所有线程局部变量均已销毁时,状态将变为就绪状态。当线程管理在退出前需要清理的资源(即使发生异常)时,或者当提供准确的日志活动或监控线程退出时,此功能非常有用。

抛出异常或避免数据竞争的同步机制方面,这两个函数的行为都与 set\_value() 和 set\_exception() 相同。

现在了解了如何将结果存储在Promise中,让我们来了解这个组合的另一个成员,即Future。

\mySubsubsection{6.2.2.}{Future}

<future> 头文件中定义了 std::future。

正如我们之前看到的, Future 是通信渠道的消费者端,提供对 Promise 存储的结果的访问。

必须通过调用 get\_future() 从 std::promise 对象创建 std::future 对象,或者通过 std::packaged\_task 对象(本章后面有更多详细信息)或调用 std::async 函数(在第 7 章中):

\begin{cpp}
std::promise<int> prom;
std::future<int> fut = prom.get_future();
\end{cpp}

与Promise一样,出于同样的原因, Future 可以移动但不能复制。要从多个 Future 引用相同的共享状态,需要使用共享 Future。

get() 方法可用于检索结果。如果共享状态仍未就绪,此调用将通过内部调用 wait() 进行阻塞。当共享状态就绪时,将返回结果值。如果共享状态中存储了异常,则将重新抛出该异常:

\begin{cpp}
try {
    int result = fut.get();
    std::cout << "Result from thread: " << result << '\n';
} catch (const std::exception& e) {
    std::cerr << "Exception: " << e.what() << '\n';
}
\end{cpp}

上面的例子中,使用 get() 函数从 fut future 中检索结果。如果结果是一个值,将以 "Result from thread" 开头的一行输出出来。另一方面,如果抛出异常并将其存储在 Promise 中,将在调用者线程中重新抛出并捕获,并输出出以 “Exception” 开头的一行。

调用 get() 方法后, valid() 将返回 false。出于某种原因在 valid() 为 false 时调用 get(),则行为未定义,但 C++ 标准建议抛出 std::future\_error 异常,错误代码为 std::future\_errc::no\_state。 valid() 函数返回 false 的 Future 仍可移动。

当 Future 销毁时,会释放其共享状态引用。如果这是最后一个引用,则共享状态将销毁。除非在使用 std::async 的特定情况下,否则这些操作不会阻塞,我们将在第 7 章中进行介绍。

\mySamllsection{Future错误和错误代码}

正如前面的例子,一些处理异步执行和共享状态的函数可能会抛出 std::future\_error 异常。

此异常类继承自 std::logic\_error,而后者又继承自 std::exception,分别在 <stdexcept> 和 <exception> 头文件中定义。

与 STL 中定义的其他异常一样,可以使用其 code() 检查错误代码函数或通过使用其 what() 函数的解释字符串。

Future 报告的错误代码由 std::future\_errorc(一个范围枚举(枚举类))定义。 C++ 标准定义了以下错误代码,但实现可能会定义其他错误代码:

\begin{itemize}
\item
broken\_promise:设置结果之前删除Promise时报告,因此共享状态在生效之前释放。

\item
future\_already\_retrieved:当 std::promise::get\_future() 调用多次时发生。

\item
promise\_already\_satisfied:如果共享状态已经有存储的结果,则由 std::promise:: set\_value() 报告。

\item
no\_state:当使用某些方法但没有共享状态时报告,因为Promise是使用默认构造函数创建的或从中移动的。当调用某些打包任务 (std::packaged\_task) 方法(例如 get\_future()、 make\_ready\_at\_thread\_exit() 或 reset())时,当共享状态尚未创建时,或者当使用 std::future::get() 和尚未准备好的Future时(std::future::valid() 返回 false),就会发生这种情况。
\end{itemize}

\mySamllsection{等待结果}

std::future 还提供了用于阻塞线程并等待结果可用的函数。这些函数是 wait()、 wait\_for() 和wait\_until()。 wait() 函数将无限期阻塞,直到结果准备好为止, wait\_for() 阻塞一段时间, wait\_until() 阻塞直到达到特定时间为止。在等待期内,一旦结果可用,所有函数都将立即返回。

仅当 valid() 为真时才必须调用这些函数;否则,行为未定义。但 C++ 标准鼓励抛出带有 std::future\_errc::no\_state 错误代码的 std::future\_error 异常。

如前所述,使用 std::promise::set\_value() 而不指定值会将共享状态设置为就绪状态。它与 std::future::wait() 一起可用于实现屏障并阻止线程继续运行,直到收到信号为止。以下示例展示了此机制的作用。

首先添加所需的头文件:

\begin{cpp}
#include <algorithm>
#include <cctype>
#include <chrono>
#include <future>
#include <iostream>
#include <iterator>
#include <sstream>
#include <thread>
#include <vector>
#include <set>
using namespace std::chrono_literals;
\end{cpp}

在 main() 函数中,程序将首先创建两个Promise, numbers\_promise 和 letters\_promise,以及对应的Future, numbers\_ready 和 letters\_ready:

\begin{cpp}
std::promise<void> numbers_promise, letters_promise;
auto numbers_ready = numbers_promise.get_future();
auto letter_ready = letters_promise.get_future();
\end{cpp}

然后, input\_data\_thread 模拟两个按顺序运行的 I/O 线程操作,一个将数字复制到向量数组中,另一个将字母插入集合中:

\begin{cpp}
std::istringstream iss_numbers{"10 5 2 6 4 1 3 9 7 8"};
std::istringstream iss_letters{"A b 53 C,d 83D 4B ca"};
std::vector<int> numbers;
std::set<char> letters;

std::jthread input_data_thread([&] {
    // Step 1: Emulating I/O operations.
    std::copy(std::istream_iterator<int>{iss_numbers},
              std::istream_iterator<int>{},
              std::back_inserter(numbers));

    // Notify completion of Step 1.
    numbers_promise.set_value();

    // Step 2: Emulating further I/O operations.
    std::copy_if(std::istreambuf_iterator<char>
                    {iss_letters},
                 std::istreambuf_iterator<char>{},
                 std::inserter(letters,
                               letters.end()),
                               ::isalpha);

    // Notify completion of Step 2.
    letters_promise.set_value();
});
// Wait for numbers vector to be filled.
numbers_ready.wait();
\end{cpp}

在此过程中,主线程使用 numbers\_ready.wait() 停止执行,等待 numbers\_promise 准备就绪。读取所有数字后, input\_data\_thread 将调用 numbers\_promise.set\_value(),唤醒主线程并继续执行。

如果尚未读取字母,则使用 letters\_ready 的 wait\_for() 函数对数字进行排序并打印,并检查它是否超时:

\begin{cpp}
std::sort(numbers.begin(), numbers.end());
if (letter_ready.wait_for(1s) == std::future_status::timeout) {
    for (int num : numbers) std::cout << num << ' ';
    numbers.clear();
}
// Wait for letters vector to be filled.
letter_ready.wait();
\end{cpp}

这部分代码展示了主线程如何执行一些工作。与此同时, input\_data\_thread 继续处理传入的数据。然后,主线程将通过调用 letters\_ready.wait() 再次等待。

最后,当所有字母都添加到集合中时,主线程将通过使用 letters\_promise.set\_value() 再次发出信号来唤醒,并且数字(如果尚未打印)和字母将按顺序打印:

\begin{cpp}
for (int num : numbers) std::cout << num << ' ';
std::cout << std::endl;
for (char let : letters) std::cout << let << ' ';
std::cout << std::endl;
\end{cpp}

正如前面的例子中看到的,等待函数返回一个未来状态对象。接下来,了解一下这些对象是什么。

\mySamllsection{Future状态}

wait\_for() 和 wait\_until() 返回一个 std::future\_status 对象。

Future可以处于以下任一状态:

\begin{itemize}
\item
就绪:共享状态已就绪,表明可以检索结果。

\item
延迟:共享状态包含延迟函数,只有在明确请求时才会计算结果。将在下一章介绍 std::async 时了解有关延迟函数的更多信息。

\item
超时:共享状态准备就绪之前已过了指定的超时时间。
\end{itemize}

接下来,将介绍如何使用共享 Future 在多个 Future 之间共享一个 Promise 结果。

\mySubsubsection{6.2.3.}{共享Future}

std::future 只能移动,因此只有一个 Future 对象可以引用特定的异步结果。另一方面, std::shared\_future 是可复制的,因此多个共享 Future 对象可以引用相同的共享状态。

因此, std::shared\_future 允许从不同线程对同一共享状态进行线程安全访问。共享 Future 可用于在多个消费者或相关方之间共享计算密集型任务的结果,从而减少冗余计算。此外,还可用于通知事件或用作同步机制,其中多个线程必须等待单个任务的完成。本章后面,将介绍如何使用共享 Future 链接异步操作。

std::shared\_object 的接口与 std::future 的接口相同,因此有关等待和 getter 函数的所有解释都适用于此。

可以通过使用 std::future::share()创建共享对象:

\begin{cpp}
std::shared_future<int> shared_fut = fut.share();
\end{cpp}

这会使原来的未来无效(其valid()函数将返回false)。

下面的示例展示了如何同时将相同的结果发送给多个线程:

\begin{cpp}
#define sync_cout std::osyncstream(std::cout)

int main() {
    std::promise<int> prom;
    std::future<int> fut = prom.get_future();
    std::shared_future<int> shared_fut = fut.share();
    std::vector<std::jthread> threads;
    for (int i = 1; i <= 5; ++i) {
        threads.emplace_back([shared_fut, i]() {
            sync_cout << "Thread " << i << ": Result = "
            << shared_fut.get() << std::endl;
        });
    }
    prom.set_value(5);
    return 0;
}
\end{cpp}

首先创建一个Promise prom,从中获取Future fut,最后通过调用 share() 获取共享Future shared\_fut。

然后,创建五个线程并将其添加到一个向量数组中,每个线程都有一个共享Future实例和一个索引。所有这些线程都将通过调用 shared\_future.get() 等待Promise prom 准备就绪。当在Promise共享状态中设置一个值时,所有线程都可以访问该值。运行上面的程序的输出如下:

\begin{shell}
Thread 5: Result = 5
Thread 3: Result = 5
Thread 4: Result = 5
Thread 2: Result = 5
Thread 1: Result = 5
\end{shell}

因此,共享Future也可用于同时向多个线程发出信号。

\mySubsubsection{6.2.4.}{打包任务}

打包任务或 std::packaged\_task 也在 <future> 头文件中定义,它是一个类模板,用于包装要异步调用的可调用对象。其结果存储在共享状态中,可通过 std::future 对象访问。要创建 st d::packaged\_task 对象,需要将表示将要调用的任务的函数签名定义为模板参数,并将所需函数作为其构造函数参数传递。以下是一些示例:

\begin{cpp}
// Using a thread.
std::packaged_task<int(int, int)> task1(
                       std::pow<int, int>);
std::jthread t(std::move(task1), 2, 10);

// Using a lambda function.
std::packaged_task<int(int, int)> task2([](int a, int b)
{
    return std::pow(a, b);
});
task2(2, 10);

// Binding to a function.
std::packaged_task<int()> task3(std::bind(std::pow<int, int>, 2, 10));
task3();
\end{cpp}

上面的例子中, task1 是使用函数创建的,并使用线程执行的。另一方面, task2 是使用 lambda 函数创建的,并通过调用其方法 operator() 来执行。最后, task3 是使用 std::bind 的转发调用包装器创建的。

要获取与任务相关的future,只需从其packaged\_task对象中调用get\_future():

\begin{cpp}
std::future<int> result = task1.get_future();
\end{cpp}

与 Promise 和 Future 一样,可以使用默认构造函数、移动构造函数或分配器构建没有共享状态的打包任务。因此,打包任务只能移动且不可复制,赋值运算符和 swap 函数的行为与 Promise 和 Future 类似。

打包任务的析构函数的行为类似于Promise的析构函数;如果共享状态在有效之前被释放,则会抛出 std::future\_error 异常,错误代码为 std::future\_errc::broken\_promise。与 Future 一样,打包任务定义了一个 valid() 函数,如果 std::packaged\_task 对象具有共享状态,则该函数返回 true。

与Promise一样, get\_future() 只能调用一次。如果多次调用此函数,则会抛出带有 future\_already\_retrieved 代码的 std::future\_error 异常。如果打包任务是从默认构造函数创建的,因此没有共享状态,则错误代码将为 no\_state。

如前面的例子所示,可以使用 operator() 来调用存储的可调用对象:

\begin{cpp}
task1(2, 10);
\end{cpp}

有时,仅当运行打包任务的线程退出并且其所有线程本地对象销毁时,才使结果准备就绪。这可以通过使用 make\_ready\_at\_thread\_exit() 函数来实现。即使结果在线程退出之前尚未准备好,也会像往常一样立即计算结果,其计算也不会被推迟。

作为示例,定义以下函数:

\begin{cpp}
void task_func(std::future<void>& output) {
    std::packaged_task<void(bool&)> task{[](bool& done){
            done = true;
    }};
    auto result = task.get_future();
    bool done = false;
    task.make_ready_at_thread_exit(done);

    std::cout << "task_func: done = "
              << std::boolalpha << done << std::endl;

    auto status = result.wait_for(0s);
    if (status == std::future_status::timeout)
        std::cout << "task_func: result not ready\n";

    output = std::move(result);
}
\end{cpp}

此函数创建一个名为 task 的打包任务,将其布尔参数设置为 true。还从此任务创建一个名为result 的Future。当通过调用 make\_ready\_at\_thread\_exit() 执行任务时,其 done 参数设置为 true,但Future结果仍未标记为就绪。当 task\_func 函数退出时,结果Future将移动到传递的引用。此时,线程退出,结果Future将设置为就绪。

因此,假设使用以下代码从主线程调用此任务:

\begin{cpp}
std::future<void> result;

std::thread t{task_func, std::ref(result)};
t.join();

auto status = result.wait_for(0s);
if (status == std::future_status::ready)
    std::cout << "main: result ready\n";
\end{cpp}

该程序将显示以下输出:

\begin{shell}
task_func: done = true
task_func: result not ready
main: result ready
\end{shell}

如果没有共享状态(no\_state 错误代码)或者任务已经调用(promise\_already\_satisfied 错误代码), make\_ready\_at\_thread\_exit() 将抛出 std::future\_error 异常。

打包的任务状态也可以通过调用 reset() 来重置,该函数将放弃当前状态并构造一个新的共享状态。显然,在调用 reset() 时没有状态,则会抛出 no\_state 错误代码的异常。重置后,必须通过调用 get\_future() 获取新的 Future。

以下示例打印前 10 个 2 的幂数,每个数字都是通过调用相同的 packaged\_task 对象计算得出的。每次循环迭代中, packaged\_task 都会重置,并检索一个新的 Future 对象:

\begin{cpp}
std::packaged_task<int(int, int)> task([](int a, int b){
    return std::pow(a, b);
});

for (int i=1; i<=10; ++i) {
    std::future<int> result = task.get_future();
    task(2, i);
    std::cout << "2^" << i << " = "
              << result.get() << std::endl;
    task.reset();
}
\end{cpp}

这是执行上述代码时的输出:

\begin{shell}
2^1 = 2
2^2 = 4
2^3 = 8
2^4 = 16
2^5 = 32
2^6 = 64
2^7 = 128
2^8 = 256
2^9 = 512
2^10 = 1024
\end{shell}

下一章中, std::async 提供了一种更简单的方法来实现相同的结果。 std::packaged\_task 的唯一优势是能够准确指定任务将在哪个线程中运行。

现在,了解了如何使用promise、future和打包任务,现在是时候了解这种方法的优点,以及可能存在的缺点了。