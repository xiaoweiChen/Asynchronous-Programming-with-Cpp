Linux 操作系统领域,守护进程是一个基本组件,在后台运行,默默地执行基本任务。这些进程传统上用以字母 d 结尾的名称来标识,例如 ssh\textbf{d} 表示安全 Shell (SSH) 守护进程, http\textbf{d} 表示 Web 服务器守护进程。处理对操作系统上运行的系统级任务着至关重要。

守护进程可用于多种用途,包括文件服务、 Web 服务和网络通信,以及日志记录和监控服务。守护进程具有自主性和弹性,从系统启动时开始运行持续运行,直到系统关闭。与用户启动和控制的普通进程不同,守护进程具有以下特点:

\begin{itemize}
\item
后台操作:
\begin{itemize}
\item
守护进程在后台运行

\item
无用于直接用户交互的控制终端

\item
不需要用户界面或手动干预执行
\end{itemize}

\item
用户独立性:
\begin{itemize}
\item
守护进程独立于用户运行

\item
自主运行,无需用户参与

\item
待系统事件或特定请求来触发
\end{itemize}

\item
以任务为导向:
\begin{itemize}
\item
每个守护进程都经过定制,以执行特定任务或一组任务

\item
旨在处理特定功能或监听特定事件或请求

\item
可确保高效执行任务
\end{itemize}
\end{itemize}

创建守护进程不仅涉及在后台运行进程。为了确保守护进程有效运行,开发人员必须考虑几个关键:

\begin{enumerate}
\item
终端分离:使用 fork() 系统调用将守护进程从终端分离。父进程在 fork 之后退出,子进程则在后台运行。

\item
会话创建: setsid() 系统调用创建新会话,并将调用进程指定为会话和进程组的领导者。此步骤对于完全从终端分离至关重要。

\item
更改目录:防止阻塞文件系统的卸载,守护进程通常将其工作目录更改为根目录。

\item
处理文件描述符:守护进程关闭继承的文件描述符,stdin、stdout 和 stderr 通常会重定向到 /dev/null。

\item
信号处理:正确处理信号(例如:用于重新加载配置的 SIGHUP 或用于正常关闭的 SIGTERM )对于有效的守护进程管理至关重要。
\end{enumerate}

守护进程通过各种 IPC 机制与其他进程或守护进程进行通信,是许多异步系统架构不可或缺的一部分,提供基本服务无需直接与用户交互。

守护进程的一些经典用例:

\begin{itemize}
\item
Web 服务器: httpd 和 nginx 等守护进程响应客户端请求提供网页,同时处理多个请求并确保无缝的网页浏览。

\item
数据库服务器: mysqld 和 postgresql 等守护进程管理数据库服务,允许各种应用程序异步访问和操作数据库。

\item
文件服务器: smbd、 nfsd 等守护进程提供网络文件服务,实现不同系统之间的异步文件共享和访问。

\item
日志记录和监控: syslogd 和 snmpd 等守护进程收集和记录系统事件,提供系统健康和性能的异步监控。
\end{itemize}

总而言之,守护进程是 Linux 系统必不可少的组件,其自主性和弹性维护着系统稳定性,并为用户和应用程序提供基本服务。

我们已经了解了进程和守护进程(一种特殊的进程)。一个进程可以有一个或多个执行线程。在下一节中,我们将介绍线程。






