在前面的章节中,我们学习了使用 C++ 管理和同步线程执行的基础知识。我们还在第 3 章中提到,要从线程返回值,我们可以使用承诺和未来。现在是时候学习如何在 C++ 中使用这些功能来做到这一点以及做更多事情了。

Future 和 Promise 是实现异步编程必不可少的块。它们定义了一种管理将来完成的任务结果的方法,通常在单独的线程中完成。

在本章中,我们将讨论以下主要主题:

\begin{itemize}
\item
什么是Future和Promise?

\item
什么是共享Future?它与普通Future有何不同?

\item
什么是打包任务以及我们何时使用它们?

\item
我们如何检查未来的状态和错误?

\item
使用承诺和未来有哪些优点和缺点?

\item
现实场景和解决方案的示例
\end{itemize}

那么,让我们开始吧!