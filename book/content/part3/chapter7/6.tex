如果没有足够的线程来运行多个 std::async 调用,则会引发 std::runtime\_system异常并指示资源耗尽。

可以通过使用计数信号量(std::counting\_semaphore)创建线程限制器来实现一个简单的解决方案,这是第 4 章中解释的多线程同步机制。

这里使用一个 std::counting\_semaphore 对象,将其初始值设置为系统允许的最大并发任务数,可以通过调用 std::thread::hardware\_concurrency() 函数来检索,然后在任务函数中使用该信号量来阻止异步任务的总数超过最大并发任务数。

以下代码实现了这个想法:

\begin{cpp}
#include <chrono>
#include <future>
#include <iostream>
#include <semaphore>
#include <syncstream>
#include <vector>

#define sync_cout std::osyncstream(std::cout)

using namespace std::chrono_literals;

void task(int id, std::counting_semaphore<>& sem) {
    sem.acquire();

    sync_cout << "Running task " << id << "...\n";
    std::this_thread::sleep_for(1s);

    sem.release();
}

int main() {
    const int total_tasks = 20;
    const int max_concurrent_tasks =
              std::thread::hardware_concurrency();

    std::counting_semaphore<> sem(max_concurrent_tasks);

    sync_cout << "Allowing only "
              << max_concurrent_tasks
              << " concurrent tasks to run "
              << total_tasks << " tasks.\n";

    std::vector<std::future<void>> futures;
    for (int i = 0; i < total_tasks; ++i) {
        futures.push_back(
            std::async(std::launch::async,
                task, i, std::ref(sem)));
    }

    for (auto& fut : futures) {
        fut.get();
    }
    std::cout << "All tasks completed." << std::endl;
    return 0;
}
\end{cpp}

该程序首先设置将要启动的任务总数。然后,创建一个计数信号量 sem,并将其初始值设置为硬件并发值。最后,启动所有任务并等待其Future准备就绪。

本例的关键点是,每个任务在执行其工作之前都会获取信号量,从而减少内部计数器或阻塞直到计数器可以减少。当工作完成后,释放信号量,这会增加内部计数器并解除阻塞,此时尝试获取信号量的其他任务。所以,只有当存在是用于该任务的空闲硬件线程。否则,线程将阻塞,直到另一个任务释放信号量。

探讨一些实际场景之前,先了解一下使用 std::async 的缺点。






