
使用 std::async 时不支持从异步任务到主线程的异常传播。为了启用异常传播,我们可能需要一个承诺对象来存储异常,稍后可以通过调用 std::async 返回的 Future 来访问该异常。但std::async 无法访问或提供该承诺对象。

实现此目的的一种可行方法是使用 std::packaged\_task 对象包装异步任务。但如果是这种情况,我们应该直接使用上一章中描述的打包任务。

我们还可以使用自 C++11 起可用的嵌套异常,方法是使用 std::nested\_exception,这是一个可以捕获和存储当前异常的多态混合类,允许任意类型的嵌套异常。从 std::nested\_exception 对象中,我们可以使用 nested\_ptr() 方法检索存储的异常,或通过调用 rethrow\_nested() 重新抛出它。

要创建嵌套异常,我们可以使用 std::throw\_with\_nested() 方法抛出异常。如果我们只想在异常嵌套时重新抛出异常,我们可以使用 std::rethrow\_if\_nested()。所有这些函数都在 <exception> 标头中定义。

使用所有这些函数,我们可以实现以下示例,其中异步任务抛出 std::runtime\_error 异常,该异常在异步任务的主体中被捕获并作为嵌套异常重新抛出。然后,此嵌套异常对象在主函数中再次被捕获,并打印出异常序列,如以下代码所示:

\begin{cpp}
#include <exception>
#include <future>
#include <iostream>
#include <stdexcept>
#include <string>

void print_exceptions(const std::exception& e,
                      int level = 1) {
    auto indent = std::string(2 * level, ‹ ‹);
    std::cerr << indent << e.what() << '\n';
    try {
        std::rethrow_if_nested(e);
    } catch (const std::exception& nestedException) {
        print_exceptions(nestedException, level + 1);
    } catch (...) { }
}

void func_throwing() {
    throw std::runtime_error(
    "Exception in func_throwing");
}

int main() {
    auto fut = std::async([]() {
        try {
            func_throwing();
        } catch (...) {
            std::throw_with_nested(
                std::runtime_error(
                    "Exception in async task."));
        }
    });

    try {
        fut.get();
    } catch (const std::exception& e) {
        std::cerr << "Caught exceptions:\n";
        print_exceptions(e);
    }
    return 0;
}
\end{cpp}

正如我们在示例中看到的,创建了一个异步任务,该任务在 try-catch 块内执行 func\_throwin g() 函数。此函数仅抛出一个 std::runtime\_error 异常,该异常被捕获,然后通过使用 std::throw\_with\_nested() 函数作为 std::nested\_exception 类的一部分重新抛出。稍后,在主线程中,当我们尝试通过调用其 get() 方法从 fut future 对象中检索结果时,嵌套异常被抛出并在主 tr y-catch 块中再次捕获,其中 print\_exceptions() 函数被调用,并使用捕获的嵌套异常作为参数。

print\_exceptions() 函数打印当前异常(e.what())的原因,如果嵌套则重新抛出异常,从而再次捕获它并按嵌套级别缩进递归打印异常原因。

由于每个异步任务都有自己的未来,程序可以分别处理来自多个任务的异常。

\mySubsubsection{7.4.1.}{调用 std::async 时发生异常}

除了异步任务中发生的异常之外,还存在 std::async 可能抛出异常的情况。这些异常如下:

\begin{itemize}
\item
std::bad\_alloc:如果没有足够的内存来存储 std::async 所需的内部数据结构。

\item
std:system\_error:如果在使用 std::launch::async 作为启动策略时无法启动新线程。在这种情况下,错误条件将是 std::errc::resource\_unavailable\_try\_again。根据实现,如果策略是默认策略,它可能会回退到延迟调用或实现定义的策略。
\end{itemize}

大多数情况下,这些异常是由于资源耗尽而引发的。一种解决方案是,当当前正在运行的一些异步任务完成并释放其资源后,稍后重试。另一种更可靠的解决方案是限制在给定时间内运行的异步任务数量。我们将很快实现此解决方案,但首先,让我们了解 std::async 返回的 Future,以及如何在处理它们时实现更好的性能。
































