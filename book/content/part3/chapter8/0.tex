在前面的章节中,我们了解了在 C++ 中编写异步代码的不同方法。我们使用了线程(基本执行单元)和一些更高级的异步代码机制,例如 Future、 Promise 和 std::async。我们将在下一章中介绍 Boost.Asio 库。所有这些方法通常使用由内核创建和管理的多个系统线程。

例如,我们程序的主线程可能需要访问数据库。这种访问可能很慢,因此我们会在另一个线程中读取数据,以便主线程可以继续执行其他任务。另一个示例是生产者-消费者模型,其中一个或多个线程生成要处理的数据项,并且一个或多个线程以完全异步的方式处理这些项。

上述两个示例都使用了线程(也称为系统(内核)线程),并且需要不同的执行单元,每个线程一个。

在本章中,我们将研究一种编写异步代码的另一种方式——协程。协程是 20 世纪 50 年代末的一个古老概念,直到最近才添加到 C++ 中,从 C++20 开始。它们不需要单独的线程( 当然,我们可以让不同的线程运行协程)。协程是一种机制,它允许我们在单个线程中执行多项任务。

在本章中,我们将讨论以下主要主题:

\begin{itemize}
\item
什么是协同程序以及 C++ 如何实现和支持它们?

\item
实现基本协程,了解 C++ 协程的要求

\item
生成器协程和新的 C++23 std::generator

\item
用于解析整数的字符串解析器

\item
协程中的异常
\end{itemize}

本章介绍不使用任何第三方库实现的 C++ 协程。这种编写协程的方式相当低级,我们需要编写代码来支持编译器。

