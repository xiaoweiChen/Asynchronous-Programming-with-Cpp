

信号处理能够捕获操作系统发送的信号,并在操作系统决定终止应用程序的进程之前正常关闭该应用程序。

Boost.Asio 为此目的提供了 boost::asio::signal\_set 类,启动异步等待一个或多个信号的发生。

这是如何处理 SIGINT 和 SIGTERM 信号的示例:

\begin{cpp}
#include <boost/asio.hpp>
#include <iostream>

int main() {
    try {
        boost::asio::io_context io_context;
        boost::asio::signal_set signals(io_context,
                                SIGINT, SIGTERM);

        auto handle_signal = [&](
                const boost::system::error_code& ec,
                int signal) {
            if (!ec) {
                std::cout << "Signal received: "
                          << signal << std::endl;

                // Code to perform cleanup or shutdown.
                io_context.stop();
            }
        };

        signals.async_wait(handle_signal);
        std::cout << "Application is running. "
                  << "Press Ctrl+C to stop...\n";

        io_context.run();
        std::cout << "Application has exited cleanly.\n";
    } catch (std::exception& e) {
        std::cerr << "Exception: " << e.what() << '\n';
    }
    return 0;
}
\end{cpp}

信号对象是 signal\_set,列出了程序等待的信号, SIGINT 和 SIGTERM。此对象有一个 async\_wait() 方法,该方法异步等待这些信号的发生,并触发完成处理程序 handle\_signal()。

与完成处理程序中通常的情况一样, handle\_signal() 检查错误代码 ec,如果没有错误,则可能会执行一些清理代码以干净利落地退出程序。此示例中,只需调用 io\_context.stop( ) 即可停止事件处理循环。

还可以使用 signals.wait() 方法同步等待信号。如果应用程序是多线程的,则信号事件处理程序必须与 io\_context 对象在同一个线程中运行,通常是主线程。

下一节中,将介绍如何取消操作。









