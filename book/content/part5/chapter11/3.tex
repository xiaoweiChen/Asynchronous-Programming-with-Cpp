

调试是查找和修复计算机程序中的错误的过程。
在本节中,我们将探讨几种调试多线程和异步软件的技术。您必须具备一些关于如何使用调试器(例如 GDB(GNU 项目调试器)或 LLDB(LLVM 底层调试器))以及调试过程术语(例如断点、观察器、回溯、框架和崩溃报告)的知识。

GDB 和 LLDB 都是出色的调试器,它们的大多数命令相同,只有少数命令不同。如果在 m acOS 上调试程序或针对大型代码库,则可能首选 LLDB。另一方面, GDB 拥有悠久的历史,为许多开发人员所熟悉,并且支持更广泛的架构和平台。在本节中,我们将使用 GDB 15.1,因为它是 GNU 框架的一部分,并且旨在与 g++ 编译器一起使用,但在调试使用 clang ++ 编译的程序时,后面显示的大多数命令也可以与 LLDB 一起使用。

由于一些处理多线程和异步代码的调试器功能仍在开发中,因此请始终将调试器更新到最新版本以包含最新的功能和修复。

\mySubsubsection{11.3.1.}{一些有用的 GDB 命令}

让我们从一些在调试任何类型的程序时很有用的 GDB 命令开始,并为下一部分奠定基础。

调试程序时,我们可以启动调试器并将程序作为参数传递。可以使用 -{}-args 选项传递程序可能需要的额外参数:

\begin{shell}
$ gdb <program> --args <args>
\end{shell}

或者,我们可以使用进程标识符(PID)将调试器附加到正在运行的程序:

\begin{shell}
$ gdb –p <PID>
\end{shell}

进入调试器后,我们可以运行程序(使用 run 命令)或启动程序(使用 start 命令)。运行意味着程序一直执行直到到达断点或完成。 start 只是在 main() 函数的开头放置一个临时断点并运行程序,在程序开头停止执行。

例如,如果我们想调试一个已经崩溃的程序,我们可以使用崩溃生成的核心转储文件,该文件可能存储在系统中的特定位置(在 Linux 系统上通常是 /var/lib/apport/coredump/,但请访问官方文档以查看系统中的确切位置)。另请注意,通常情况下,核心转储默认是禁用的,需要在程序崩溃之前在同一个 shell 中运行 \verb|ulimit -c unlimited| 命令。如果我们处理的程序非常大或系统磁盘空间不足,则可以将 unlimited 参数更改为某个任意限制。

生成 coredump 文件后,只需将其复制到程序二进制文件所在的位置,并使用以下命令:

\begin{shell}
$ gdb <program> <coredump>
\end{shell}

请注意,所有二进制文件都必须有调试符号,因此要使用 –g 选项进行编译。在生产系统中,发布二进制文件通常会剥离符号并将其存储在单独的文件中。有 GDB 命令可以包含这些符号,还有命令行工具可以检查它们,但这个主题超出了本书的范围。

调试器启动后,我们可以使用 GDB 命令浏览代码或检查变量。一些有用的命令如下:

\begin{itemize}
\item
info args:显示有关调用当前函数所用参数的信息。

\item
info locals:显示当前范围内的局部变量。

\item
whatis:显示给定变量或表达式的类型。

\item
return:从当前函数返回,不执行其余指令。可以指定返回值。

\item
backtrace:列出了当前调用堆栈中的所有堆栈帧。

\item
frame:这可以让您改变到特定的堆栈框架。

\item
up和down:在调用堆栈中移动,朝向当前函数的调用者(向上)或被调用者(向下)。

\item
print:计算并显示表达式的值,该表达式可以是变量名、类成员、指向内存区域的指针或直接是内存地址。我们还可以定义漂亮的打印机来显示我们自己的类。
\end{itemize}

让我们用最基本但也最常用的程序调试技术之一来结束本节。这种技术称为 printf。每个开发人员都使用过 printf 或其他命令来打印代码路径上的变量内容,以在战略代码位置显示其内容。在 GDB 中, dprintf 命令有助于设置 printf 样式的断点,当遇到这些断点时,这些断点会打印信息,但不停止程序执行。这样,我们可以在调试程序时使用 print 语句,而无需修改代码、重新编译和重新启动程序。

其语法如下:

\begin{shell}
$ dprintf <location>, <format>, <args>
\end{shell}

例如,如果我们想在第 25 行设置一个 printf 语句来打印 x 变量的内容,但仅当其值大于 5 时,则命令如下:

\begin{shell}
$ dprintf 25, "x = %d\n", x if x > 5
\end{shell}

现在我们已经有了一些基础知识,让我们开始调试多线程程序。

\mySubsubsection{11.3.2.}{调试多线程程序}

这里显示的示例永远不会完成,因为两个互斥锁被不同的线程以不同的顺序锁定,从而导致死锁,正如本章前面介绍日志记录时所解释的那样:

\begin{cpp}
#include <chrono>
#include <mutex>
#include <thread>

using namespace std::chrono_literals;

int main() {
    std::mutex mtx1, mtx2;

    std::thread t1([&]() {
        std::lock_guard lock1(mtx1);
        std::this_thread::sleep_for(100ms);
        std::lock_guard lock2(mtx2);
    });

    std::thread t2([&]() {
        std::lock_guard lock2(mtx2);
        std::this_thread::sleep_for(100ms);
        std::lock_guard lock1(mtx1);
    });

    t1.join();
    t2.join();

    return 0;
}
\end{cpp}

首先,让我们使用 g++ 编译此示例,并添加调试符号(-g 选项)并禁止代码优化(-O0 选项),以防止编译器重组二进制代码,并使调试器更难通过使用 -{}-fno-omit-frame-pointer 选项来查找和显示相关信息。

以下命令编译 test.cpp 源文件并生成测试二进制文件。我们也可以使用相同选项的 clang++ :

\begin{shell}
$ g++ -o test –g -O0 --fno-omit-frame-pointer test.cpp
\end{shell}

如果我们运行生成的程序,它将永远不会完成:

\begin{shell}
$ ./test
\end{shell}

要调试正在运行的程序,我们首先使用 ps Unix 命令检索其 PID:

\begin{shell}
$ ps aux | grep test
\end{shell}

然后,通过提供 pid 附加调试器并开始调试程序:

\begin{shell}
$ gdb –p <pid>
\end{shell}

假设调试器以以下消息启动:

\begin{shell}
ptrace: Operation not permitted.
\end{shell}

然后,只需运行以下命令:

\begin{shell}
$ sudo sysctl -w kernel.yama.ptrace_scope=0
\end{shell}

一旦 GDB 正确启动,您将能够在其提示符中输入命令。

我们可以执行的第一个命令是检查哪些线程正在运行的下一个命令:

\begin{shell}
(gdb) info threads
Id Target Id Frame
* 1 Thread 0x79d1f3883740 (LWP 14428) "test" 0x000079d1f3298d61 in
__futex_abstimed_wait_common64 (private=128, cancel=true, abstime=0x0,
op=265, expected=14429, futex_word=0x79d1f3000990)
at ./nptl/futex-internal.c:57
2 Thread 0x79d1f26006c0 (LWP 14430) "test" futex_wait (private=0,
expected=2, futex_word=0x7fff5e406b00) at ../sysdeps/nptl/futexinternal.h:146
3 Thread 0x79d1f30006c0 (LWP 14429) "test" futex_wait (private=0,
expected=2, futex_word=0x7fff5e406b30) at ../sysdeps/nptl/futexinternal.h:146
\end{shell}

输出显示, GDB 标识符为 1 的 0x79d1f3883740 线程是当前线程。如果有很多线程,而我们只对特定子集感兴趣,比如线程 1 和 3,我们可以使用以下命令仅显示这些线程的信息:

\begin{shell}
(gdb) info thread 1 3
\end{shell}

运行 GDB 命令将影响当前线程。例如,运行 bt 命令将显示线程 1 的回溯(输出已简化):

\begin{shell}
(gdb) bt
#0 0x000079d1f3298d61 in __futex_abstimed_wait_common64
(private=128, cancel=true, abstime=0x0, op=265, expected=14429, futex_word=0x79d1f3000990) at ./nptl/futex-internal.c:57
#5 0x000061cbaf1174fd in main () at 11x18-debug_deadlock.cpp:22
\end{shell}

要切换到另一个线程,例如线程 2,我们可以使用 thread 命令:

\begin{shell}
(gdb) thread 2
[Switching to thread 2 (Thread 0x79d1f26006c0 (LWP 14430))]
\end{shell}

现在, bt 命令将显示线程 2 的回溯(输出简化):

\begin{shell}
(gdb) bt
#0 futex_wait (private=0, expected=2, futex_word=0x7fff5e406b00) at
../sysdeps/nptl/futex-internal.h:146
#2 0x000079d1f32a00f1 in lll_mutex_lock_optimized
(mutex=0x7fff5e406b00) at ./nptl/pthread_mutex_lock.c:48
#7 0x000061cbaf1173fa in operator() (__closure=0x61cbafd64418) at 11x18-debug_deadlock.cpp:19
\end{shell}

要在不同的线程中执行命令,只需使用线程应用命令,在本例中,在线程 1 和 3 上执行 bt 命令:

\begin{shell}
(gdb) thread apply 1 3 bt
\end{shell}

要在所有线程中执行命令,只需使用 \verb|thread apply all <command>|。
请注意,当在多线程程序中到达断点时,所有执行线程都会停止运行,从而允许检查程序的整体状态。当通过 continue、 step 或 next 等命令重新启动执行时,所有线程都会恢复。当前线程将向前移动一个语句,但其他线程则不能保证,它们可能会向前移动几个语句,甚至在语句中间停止。

当执行停止时,调试器将跳转并显示当前线程中的执行上下文。为了避免调试器通过锁定调度程序在线程之间跳转,我们可以使用以下命令:

\begin{shell}
(gdb) set scheduler-locking <on/off>
\end{shell}

我们还可以使用以下命令来检查调度程序锁定状态:

\begin{shell}
(gdb) show scheduler-locking
\end{shell}

现在我们已经学习了一些用于多线程调试的新命令,让我们检查一下附加到调试器的应用程序发生了什么。

如果我们检索线程 2 和 3 中的回溯,我们可以看到以下内容(输出简化,仅显示相关部分):

\begin{shell}
(gdb) thread apply all bt
Thread 3 (Thread 0x79d1f30006c0 (LWP 14429) "test"):
#0 futex_wait (private=0, expected=2, futex_word=0x7fff5e406b30) at
../sysdeps/nptl/futex-internal.h:146
#5 0x000061cbaf117e20 in std::mutex::lock (this=0x7fff5e406b30) at /
usr/include/c++/14/bits/std_mutex.h:113
#7 0x000061cbaf117334 in operator() (__closure=0x61cbafd642b8) at
11x18-debug_deadlock.cpp:13
Thread 2 (Thread 0x79d1f26006c0 (LWP 14430) "test"):
#0 futex_wait (private=0, expected=2, futex_word=0x7fff5e406b00) at
../sysdeps/nptl/futex-internal.h:146
#5 0x000061cbaf117e20 in std::mutex::lock (this=0x7fff5e406b00) at /
usr/include/c++/14/bits/std_mutex.h:113
#7 0x000061cbaf1173fa in operator() (__closure=0x61cbafd64418) at
11x18-debug_deadlock.cpp:19
\end{shell}

请注意,运行 std::mutex::lock() 之后,两个线程都在第 13 行等待线程 3,在第 19 行等待线程 2,这分别与 std::thread t1 中的 std::lock\_guard lock2 和 std::thread t2 中的 std::lock\_guard l ock1 匹配。

因此,我们检测到这些代码位置的线程中发生了死锁。

现在让我们通过捕获竞争条件来进一步了解如何调试多线程软件。

\mySubsubsection{11.3.3.}{调试条件竞争}

竞争条件是最难检测和调试的错误之一,因为它们通常偶尔发生且每次的影响都不同,并且有时在程序达到失败点之前会发生一些昂贵的计算。

这种不稳定的行为不仅仅是由竞争条件引起的。与不正确的内存分配相关的其他问题也会导致类似的症状,因此,除非我们进行调查并得出根本原因诊断,否则不可能将错误归类为竞争条件。

调试竞争条件的一种方法是使用观察点手动检查变量是否在当前线程中执行的任何语句都没有修改它的情况下更改了它的值,或者将断点放置在由特定线程触发的战略位置,如下所示:

\begin{shell}
(gdb) break <linespec> thread <id> if <condition>
\end{shell}

例如,请参阅以下内容:

\begin{shell}
(gdb) break test.cpp:11 thread 2
\end{shell}

或者,甚至使用断言并检查不同线程访问的任何变量的当前值是否具有预期值。 下一个示例遵循此方法:

\begin{cpp}
#include <cassert>
#include <chrono>
#include <cmath>
#include <iostream>
#include <mutex>
#include <thread>

using namespace std::chrono_literals;

static int g_value = 0;
static std::mutex g_mutex;

void func1() {
    const std::lock_guard<std::mutex> lock(g_mutex);
    for (int i = 0; i < 10; ++i) {
        int old_value = g_value;
        int incr = (rand() % 10);
        g_value += incr;
        assert(g_value == old_value + incr);
        std::this_thread::sleep_for(10ms);
    }
}

void func2() {
    for (int i = 0; i < 10; ++i) {
        int old_value = g_value;
        int incr = (rand() % 10);
        g_value += (rand() % 10);
        assert(g_value == old_value + incr);
        std::this_thread::sleep_for(10ms);
    }
}

int main() {
    std::thread t1(func1);
    std::thread t2(func2);

    t1.join();
    t2.join();

    return 0;
}
\end{cpp}

这里,两个线程 t1 和 t2 正在运行函数,这些函数将 g\_value 全局变量增加一个随机值。每次增加时,都会将 g\_value 与预期值进行比较,如果不相等,则 assert 指令将停止程序。

编译该程序并运行调试器,如下所示:

\begin{shell}
$ g++ -o test -g -O0 test
$ gdb ./test
\end{shell}

调试器启动后,使用 run 命令运行程序。程序将运行,并在某个时刻通过接收 SIGABRT 信号中止,表明断言未得到满足。

\begin{shell}
test: test.cpp:29: void func2(): Assertion `g_value == old_value + incr' failed.
Thread 3 "test" received signal SIGABRT, Aborted.
\end{shell}

当程序停止后,我们可以使用 backtrace 命令检查该点的回溯,并将该故障点的源代码更改为特定的框架或列表。

这个例子非常简单,所以通过检查断言输出可以清楚地发现 g\_value 变量出了问题,而且这很可能是竞争条件。

但是对于更复杂的程序来说,手动调试问题的过程非常艰巨,因此让我们关注另一种可以帮助解决这个问题的技术,称为反向调试。

\mySubsubsection{11.3.4.}{反向调试}

反向调试,也称为时间旅行调试,允许调试器在程序失败后停止程序并返回程序执行的历史记录以调查失败的原因。此功能通过记录每条机器指令的执行来实现被调试程序的运行情况以及内存和寄存器值的每次变化,然后使用这些记录随意重放和后退程序。
在 Linux 上,我们可以使用 GDB(自 7.0 版起)、 rr(最初由 Mozilla 开发, \url{https://rr-project.org})或 Undo 的时间旅行调试器 (UDB) (\url{https://docs.undo.io})。在 Windows 上,我们可以使用时间旅行调试 (\url{https://learn.microsoft.com/en-us/windows-hardware/drivers/debuggercmds/tim e-travel-debugging-overview})。

反向调试仅受有限数量的 GDB 目标支持,例如远程目标 Simics、系统集成和设计 (SID) 模拟器或本机 Linux 的进程记录和重放目标(仅适用于 i386、 amd64、 moxie-elf 和 arm)。在撰写本书时, Clang 的反向调试功能仍在开发中。

因此,由于这些限制,我们决定使用 rr 做一个小展示。请按照项目网站上的说明构建和安装 rr 调试工具: \url{https://github.com/rr-debugger/rr/wiki/Building-And-Installing}。

安装后,要录制和回放程序,请使用以下命令:

\begin{shell}
$ rr record <program> --args <args>
$ rr replay
\end{shell}

例如,如果我们有一个名为 test 的程序,命令序列将如下所示:

\begin{shell}
$ rr record test
rr: Saving execution to trace directory `/home/user/.local/share/rr/test-1'.
\end{shell}

假设显示的是以下致命错误:

\begin{shell}
[FATAL src/PerfCounters.cc:349:start_counter()] rr needs /proc/sys/kernel/perf_event_paranoid <= 3, but it is 4. Change it to <= 3.
Consider putting 'kernel.perf_event_paranoid = 3' in /etc/sysctl.d/10-rr.conf.
\end{shell}

然后,使用以下命令调整内核变量 kernel.perf\_event\_paranoid:

\begin{shell}
$ sudo sysctl kernel.perf_event_paranoid=1
\end{shell}

一旦有记录可用,请使用重放命令开始调试程序:

\begin{shell}
$ rr replay
\end{shell}

或者,如果程序崩溃而您只想在录制结束时开始调试,请使用 –e 选项:

\begin{shell}
$ rr replay -e
\end{shell}

此时, rr 将使用 GDB 调试器启动程序并加载其调试符号。然后,您可以使用以下任意命令进行逆向调试:

\begin{itemize}
\item
reverse-continue:开始反向执行程序。当到达断点或由于同步异常时,执行将停止。

\item
reverse-next:向后运行到当前堆栈帧中执行的上一行的开头。

\item
reverse-nexti:这将反向执行单个指令,跳过那些移动到内部堆栈帧的指令。

\item
reverse-step:向后运行程序,直到控制到达新源行的开头。

\item
reverse-stepi:反向执行一条机器指令。

\item
reverse-finish:一直执行到当前函数调用,也就是当前函数的开始。
\end{itemize}

我们还可以反转调试的方向,并在相反方向上使用常规的正向调试命令(例如next, step, continue等),方法是使用以下命令:

\begin{shell}
(rr) set exec-direction reverse
\end{shell}

要将执行方向重新设置为正向,请使用以下命令:

\begin{shell}
(rr) set exec-direction forward
\end{shell}

作为练习,安装 rr 调试器并尝试使用反向调试来调试前面的示例。

现在让我们继续讨论如何调试协程,由于其异步特性,这是一项具有挑战性的任务。

\mySubsubsection{11.3.5.}{调试协程}

正如我们目前所见,异步代码可以像同步代码一样进行调试,方法是在具有特定条件的战略位置使用断点,使用观察点检查变量,并进入或跳过代码。此外,使用前面描述的技术选择特定线程并锁定调度程序有助于避免调试时不必要的干扰。

我们已经知道,异步代码很复杂,例如执行异步代码时将使用哪个线程,这使其调试起来更加困难。对于 C++ 协程,由于其暂停/恢复特性,调试甚至更难掌握。

Clang 使用协程编译程序分为两步: Clang 进行语义分析, LLVM 中端构建并优化协程框架。由于调试信息是在 Clang 前端生成的,因此在编译过程的后期生成协程框架时,调试信息会不足。 GCC 遵循类似的方法。

此外,如果执行在协程内部中断,当前帧将只有一个变量 frame\_ptr。在协程中,没有指针或函数参数。协程在暂停之前将其状态存储在堆中,并且在执行期间仅使用堆栈。 frame\_ptr 用于访问协程正常运行所需的所有必要信息。

我们来调试一下第 9 章中实现的 Boost.Asio 协程示例,这里只展示相关指令,完整源代码请访问第 9 章协程部分:

\begin{cpp}
boost::asio::awaitable<void> echo(tcp::socket socket) {
    char data[1024];

    while (true) {
        std::cout << "Reading data from socket...\n";//L12
        std::size_t bytes_read = co_await
            socket.async_read_some(
                boost::asio::buffer(data),
                             boost::asio::use_awaitable);
        /* .... */
        co_await boost::asio::async_write(socket,
                boost::asio::buffer(data, bytes_read),
                boost::asio::use_awaitable);
    }
}

boost::asio::awaitable<void>
listener(boost::asio::io_context& io_context,
         unsigned short port) {
    tcp::acceptor acceptor(io_context,
                           tcp::endpoint(tcp::v4(), port));
    while (true) {
        std::cout << "Accepting connections...\n"; // L45
        tcp::socket socket = co_await
            acceptor.async_accept(
                boost::asio::use_awaitable);

        boost::asio::co_spawn(io_context,
            echo(std::move(socket)),
            boost::asio::detached);
    }
}
/* main function */
\end{cpp}

由于我们使用 Boost,因此让我们包含 Boost.System 库以在编译源代码时添加更多符号以供调试:

\begin{shell}
$ g++ --std=c++20 -ggdb -O0 --fno-omit-frame-pointer -lboost_system test.cpp -o test
\end{shell}

然后,我们使用生成的程序启动调试器,并在第 12 行和第 45 行设置断点,它们是每个协程的 while 循环内第一条指令的位置:

\begin{shell}
$ gdb –q ./test
(gdb) b 12
(gdb) b 45
\end{shell}

我们还启用 GDB 内置漂亮打印机来显示标准模板库容器的可读输出:

\begin{shell}
(gdb) set print pretty on
\end{shell}

如果现在我们运行该程序(run 命令),它将在接受连接之前到达协程侦听器内部第 42 行的断点。使用 info locals 命令,我们可以检查局部变量。

协程会创建一个具有多个内部字段的状态机,例如承诺对象、附加线程、调用者对象的地址、待处理的异常等。此外,它们还存储恢复和销毁回调。这些结构依赖于编译器,与编译器的实现相关联,如果我们使用 Clang,则可以通过 frame\_ptr 访问。

如果我们继续运行程序(使用 continue 命令),服务器将等待客户端连接。要退出等待状态,我们使用 telnet(如第 9 章所示)将客户端连接到服务器。此时,执行将停止,因为到达了 echo() 协程内第 12 行的断点,并且 info locals 显示每个 echo 连接使用的变量。

使用 backtrace 命令将显示一个调用堆栈,该堆栈可能由于协程的暂停特性而具有一些复杂性。

在第 8 章描述的纯 C++ 例程中,有两个表达式设置断点可能会很有趣:

\begin{itemize}
\item
co\_await:执行暂停,直到等待的操作完成。可以通过检查底层 await\_suspend、 await\_resume 或自定义 awaitable 代码在协程恢复的位置设置断点。

\item
co\_yield:暂停执行并产生一个值。在调试期间,进入 co\_yield 以观察协程及其调用函数之间的控制流。
\end{itemize}

由于协程在 C++ 世界中相当新,并且编译器也在不断发展,我们希望协程的调试很快就能变得更加简单。

一旦我们发现并调试了一些错误,并且可以重现导致这些特定错误的场景,设计一些涵盖这些情况的测试将很方便,以避免将来更改代码而导致类似的问题或事件。让我们在下一章中学习如何测试多线程和异步代码。






