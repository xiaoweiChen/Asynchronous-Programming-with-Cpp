本节中,将介绍栅栏门闩,这是 C++20 中引入的两个新同步原语。这些机制允许线程相互等待,从而协调并发任务的执行。

\mySubsubsection{4.8.1.}{std::latch}

std::latch 门闩是一种同步原语,允许一个或多个线程阻塞,直到指定数量的操作完成。它是一个一次性对象,当计数达到零,就无法重置。

以下示例简单说明了门闩在多线程应用程序中的使用。要编写一个函数,将向量数组的每个元素乘以二,然后将向量数组的所有元素相加。这里,使用三个线程将向量数组中的元素乘以二,然后使用一个线程将向量数组中的所有元素相加并得出结果。

这里需要两个门闩。第一个门闩将由三个线程中的每一个线程乘以两个向量元素而减少。添加线程将等待此门闩为零。然后,主线程将等待第二个门闩以同步打印添加所有向量数组中元素的结果。也可以等待执行添加的线程在其上调用 join,这也可以使用门闩来完成。

现在,分析功能块中的代码:

\begin{cpp}
std::latch map_latch{ 3 };
auto map_thread = [&](std::vector<int>& numbers, int start, int end) {
    for (int i = start; i < end; ++i) {
        numbers[i] *= 2;
    }
    map_latch.count_down();
};
\end{cpp}

每个乘法线程都会运行此 lambda 函数,将向量数组中某个范围(从开始到结束)的两个元素相乘。线程完成后,会将 map\_latch 计数器减一。所有线程完成任务后,门闩计数器将为零,而阻塞等待 map\_latch 的线程将能够继续将向量的所有元素相加。注意,线程访问向量数组的不同元素,因此不需要同步对向量数组本身的访问,但在完成所有乘法之前,不能立即进行累加。

累加线程的代码如下:

\begin{cpp}
std::latch reduce_latch{ 1 };
auto reduce_thread = [&](const std::vector<int>& numbers, int& sum) {
    map_latch.wait();

    sum = std::accumulate(numbers.begin(), numbers.end(), 0);

    reduce_latch.count_down();
};
\end{cpp}

该线程等待 map\_latch 计数器降至零,然后添加向量的所有元素,最后减少 reduce\_latch 计数器(将降至零)以便主线程能够输出最终结果:

\begin{cpp}
reduce_latch.wait();
std::cout << "All threads finished. The sum is: " << sum << '\n';
\end{cpp}

了解了门闩的基本应用之后,接下来让介绍一下栅栏。

\mySubsubsection{4.8.2.}{std::barrier}

std::barrier 栅栏是另一个用于同步一组线程的同步原语。 std::barrier 栅栏是可重复使用的。每个线程到达栅栏并等待,直到所有参与线程都到达相同的栅栏点(与使用门闩的情况一样)。

std::barrier 和 std::latch 之间的主要区别在于重置功能。 std::latch 门闩是一次性栅栏,具有无法重置的倒计时机制。一旦达到零,就会保持在零。相比之下, std::barrier 是可重复使用的。会在所有线程到达栅栏后重置,从而允许同一组线程多次在同一个栅栏处同步。

何时使用门闩,何时使用栅栏?当有线程的一次性聚集点时,请使用 std::latch,例如:等待多个初始化完成后再继续。当需要在任务的多个阶段或迭代计算中重复同步线程时,请使用 std::barrier。

现在将重写前面的示例,这次使用栅栏,而非门闩。每个线程将将其对应的向量数组元素范围乘以两倍,然后将其相加。此示例中,主线程将使用 join() 等待处理完成,然后将每个线程获得的结果相加。

工作线程的代码:

\begin{cpp}
std::barrier map_barrier{ 3 };
auto worker_thread = [&](std::vector<int>& numbers, int start, int
end, int id) {
    std::cout << std::format("Thread {0} is starting...\n", id);

    for (int i = start; i < end; ++i) {
        numbers[i] *= 2;
    }

    map_barrier.arrive_and_wait();

    for (int i = start; i < end; ++i) {
        sum[id] += numbers[i];
    }

    map_barrier.arrive();
};
\end{cpp}

代码与栅栏同步。当工作线程完成乘法运算时,会减少 map\_barrier 计数器并等待栅栏计数器为零。一旦降为零,线程就会结束等待并开始执行加法。重置栅栏计数器,其值再次等于三。当完成加法运算时,栅栏计数器就会再次减少,但这一次,线程不会等待,因为其任务已经完成。

当然——每个线程都可以先完成加法,然后乘以 2。它们不需要互相等待,线程完成的工作都独立于其他线程的工作,但这是一个用简单示例解释栅栏如何工作的好方法。

主线程只需使用 join 等待工作线程完成,然后输出结果:

\begin{cpp}
for (auto& t : workers) {
    t.join();
}
std::cout << std::format("The total sum is {0}\n",
                         std::accumulate(sum.begin(), sum. End(), 0));
\end{cpp}

以下是门闩和栅栏示例的完整代码:

\begin{cpp}
#include <algorithm>
#include <barrier>
#include <format>
#include <iostream>
#include <latch>
#include <numeric>
#include <thread>
#include <vector>

void multiply_add_latch() {
    const int NUM_THREADS{3};

    std::latch map_latch{NUM_THREADS};
    std::latch reduce_latch{1};

    std::vector<int> numbers(3000);
    int sum{};
    std::iota(numbers.begin(), numbers.end(), 0);

    auto map_thread = [&](std::vector<int>& numbers, int start, int
    end) {
        for (int i = start; i < end; ++i) {
            numbers[i] *= 2;
        }
        map_latch.count_down();
    };

    auto reduce_thread = [&](const std::vector<int>& numbers, int&
    sum) {
        map_latch.wait();

        sum = std::accumulate(numbers.begin(), numbers.end(), 0);

        reduce_latch.count_down();
    };

    for (int i = 0; i < NUM_THREADS; ++i) {
        std::jthread t(map_thread, std::ref(numbers), 1000 * i, 1000 *
        (i + 1));
    }

    std::jthread t(reduce_thread, numbers, std::ref(sum));

    reduce_latch.wait();

    std::cout << "All threads finished. The total sum is: " << sum << '\n';
}

void multiply_add_barrier() {
    const int NUM_THREADS{3};

    std::vector<int> sum(3, 0);
    std::vector<int> numbers(3000);
    std::iota(numbers.begin(), numbers.end(), 0);

    std::barrier map_barrier{NUM_THREADS};

    auto worker_thread = [&](std::vector<int>& numbers, int start, int
    end, int id) {
        std::cout << std::format("Thread {0} is starting...\n", id);

        for (int i = start; i < end; ++i) {
            numbers[i] *= 2;
        }

        map_barrier.arrive_and_wait();

        for (int i = start; i < end; ++i) {
            sum[id] += numbers[i];
        }

        map_barrier.arrive();
    };

    std::vector<std::jthread> workers;
    for (int i = 0; i < NUM_THREADS; ++i) {
        workers.emplace_back(worker_thread, std::ref(numbers), 1000 *
        i, 1000 * (i + 1), i);
    }

    for (auto& t : workers) {
        t.join();
    }

    std::cout << std::format("All threads finished. The total sum is: {0}\n",
    std::accumulate(sum.begin(), sum.end(), 0));
}

int main() {
    std::cout << "Multiplying and reducing vector using barriers..." << std::endl;
    multiply_add_barrier();

    std::cout << "Multiplying and reducing vector using latches..." << std::endl;
    multiply_add_latch();
    return 0;
}
\end{cpp}

本节中,介绍了栅栏和门闩。虽然不像互斥锁、条件变量和信号量那样常用,但了解一下总是好的。这里介绍的简单示例说明了栅栏和门闩的常见用途:同步在不同阶段执行处理的线程。

最后,将看到一种只执行一次代码的机制,该代码会在不同的线程中多次调用。

























