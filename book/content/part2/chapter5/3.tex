第 4 章中,介绍了同步队列的实现,并使用互斥锁和条件变量作为同步原语。因为线程被(操作系统)阻塞,所以与锁同步的数据结构称为阻塞数据结构。

不使用锁的数据结构称为非阻塞数据结构,大多数(但不是全部)非阻塞数据结构都是无锁的。

如果每个同步操作在有限的步骤内完成,而不允许无限期地等待条件变为真或假,则数据结构或算法可认为是无锁的。

无锁数据结构的类型如下:

\begin{itemize}
\item
无阻塞:如果所有其他线程都暂停,则线程将在有限的步骤内完成其操作

\item
无锁:当多个线程同时处理数据结构时,一个线程将在有限的步骤内完成其操作

\item
无等待:当多个线程同时处理数据结构时,所有线程将在有限的步骤内完成其操作
\end{itemize}

实现无锁数据结构非常复杂,需要确保它是必要的。

使用无锁数据结构的原因如下:

\begin{itemize}
\item
实现最大并发性:如前所述,当数据访问同步涉及细粒度数据(如本机类型变量)时,原子操作是一个不错的选择。根据前面的定义,无锁数据结构允许至少一个访问数据结构的线程,在有限数量的步骤中取得一些进展。无等待结构将允许所有访问数据结构的线程取得一些进展。但当我们使用锁时,一个线程拥有锁,而其余线程只是等待锁可用,使用无锁数据结构可以实现的并发性会好得多。

\item
无死锁:不涉及锁,所以代码中不可能出现死锁。

\item
性能:某些应用程序必须实现尽可能低的延迟,因此等待锁定是不可接受的。当线程尝试获取锁定但无法获取时,操作系统会阻止该线程。当线程阻塞时,调度程序会进行上下文切换,以便能够调度另一个线程执行。这些上下文切换需要时间,而对于低延迟应用程序(例如:高性能网络数据包接收器/处理器),这段时间可能太多了。
\end{itemize}

已经介绍了什么是阻塞和非阻塞数据结构,以及什么是无锁代码。我们将在下一节介绍 C++ 内存模型。






















