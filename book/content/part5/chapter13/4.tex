
现代 CPU 的计算速度非常快,当我们想要实现最大性能时,内存访问是主要瓶颈。内存访问的合理估计时间约为 150 纳秒。在此期间,我们的 3.6 GHz CPU 经历了 540 个时钟周期。粗略估计,如果 CPU 每两个周期执行一条指令,则有 270 条指令。对于普通应用程序,内存访问是一个问题,即使编译器可能会重新排序其生成的指令,并且 CPU 也可能重新排序指令以优化内存访问并尝试运行尽可能多的指令。

因此,为了提高现代 CPU 的性能,我们拥有所谓的 CPU 缓存或内存缓存,即芯片中用于存储数据和指令的内存。这种内存比 RAM 快得多,允许 CPU 更快地检索数据,从而显著提高整体性能。

以现实生活中的缓存为例,想象一下厨师。他们需要一些食材来为餐厅客户做午餐。现在,想象一下,他们只在客户来到餐厅点餐时才购买这些食材。这会非常慢。他们也可以去超市购买一整天的食材。现在,他们可以在更短的时间内为所有客户做饭并为他们提供餐点。

CPU 缓存遵循相同的概念:当 CPU 需要访问一个变量(比如一个 4 字节整数)时,它会读取 64 个字节(此大小可能不同,具体取决于 CPU,但大多数现代 CPU 都使用该大小)的连续内存,以防需要访问更多连续的数据。

从内存访问的角度来看,线性内存数据结构(如 std::vector)的性能会更好,因为在这些情况下,缓存可以大大提高性能。对于其他类型的数据结构(如 std::list),情况并非如此。
当然,这只是为了优化缓存的使用。

您可能想知道,如果 CPU 内置缓存如此好,那么为什么所有内存不都是这样呢?答案是成本。缓存非常快(比 RAM 快得多),但也非常昂贵。

现代 CPU 采用分层缓存结构,通常由三个级别组成,分别称为 L1、 L2 和 L3:

\begin{itemize}
\item
L1 缓存最小、速度最快。它也是距离 CPU 最近的缓存,也是最昂贵的缓存。它通常分为两部分:用于存储指令的指令缓存和用于存储数据的数据缓存。典型大小为 64 KB,分为 32 KB 用于存储指令和 32 KB 用于存储数据。 L1 缓存的典型访问时间为 1 到 3 纳秒。

\item
L2 缓存比 L1 缓存更大,速度也稍慢,但仍比 RAM 快得多。典型的 L2 缓存大小在 128
Kb 到 512 Kb 之间(用于运行本章示例的 CPU 每个核心有 512 Kb 的 L2 缓存)。 L2 缓
存的典型访问时间约为 3 到 5 纳秒。

\item
L3 缓存是三者中最大且最慢的。 L1 和 L2 缓存是每个核心的(每个核心都有自己的 L1
和 L2 缓存),但 L3 由多个核心共享。我们的 CPU 有 32 Mb 的 L3 缓存,由每组八个
核心共享。典型的访问时间约为 10 到 15 纳秒。
\end{itemize}

因此,让我们将注意力转向与内存缓存相关的另一个重要概念。

\mySubsubsection{13.4.1.}{缓存一致性}

CPU 不直接访问 RAM。这种访问始终通过缓存进行,并且只有当 CPU 在缓存中找不到所需数据时才会访问 RAM。在多核系统中,每个核心都有自己的缓存,这意味着一块 RAM 可能同时存在于多个核心的缓存中。这些副本需要始终保持同步;否则,计算结果可能不正确。

到目前为止,我们已经看到每个核心都有自己的 L1 缓存。让我们回到我们的例子,思考一下当我们使用非对齐内存运行该函数时会发生什么。

在这种情况下, result\_data 的每个实例都是 8 个字节。我们创建一个包含 8 个 result\_data 实例的数组,每个线程一个。占用的总内存将为 64 个字节,所有实例在内存中都是连续的。每次线程更新随机数的总和时,它都会更改存储在缓存中的值。请记住, CPU 总是会一次读取和写入 64 个字节(称为缓存行 - 您可以将其视为最小的内存访问单元)。所有变量都在同一个缓存行中,即使线程不共享它们(每个线程都有自己的变量 - sum), CPU 也不知道这一点,需要让所有核心都看到这些变化。

这里,我们有 8 个核心,每个核心都在运行一个线程。每个核心都将 RAM 中的 64 字节内存加载到 L1 缓存中。由于线程只读取变量,所以一切正常,但是只要一个线程修改其变量,缓存行的内容就会失效。

现在,由于缓存行在其余 7 个核心中无效, CPU 需要将更改传播到所有核心。如前所述,即使线程不共享变量, CPU 也不可能知道这一点,它会更新所有核心的所有缓存行以保持值一致。这称为缓存一致性。如果线程共享变量,则不将更改传播到所有核心是不正确的。

在我们的示例中,缓存一致性协议在 CPU 内部产生大量流量,因为所有线程都共享变量所在的内存区域,尽管从程序的角度来看并非如此。这就是我们称之为伪共享的原因:变量共享是因为缓存和缓存一致性协议的工作方式。

当我们将数据对齐到 64 字节边界时,每个实例占用 64 个字节。这保证了它们位于自己的缓存行中,并且不需要缓存一致性流量,因为在这种情况下没有数据共享。在第二种情况下,性能要好得多。

让我们使用 perf 来确认这确实发生了。

首先,我们在执行 sum\_random\_unaligned 时运行 perf。我们想要查看程序访问缓存的次数以及缓存未命中的次数。每次缓存需要更新,因为它包含的数据也位于另一个核心的缓存行中,这算作一次缓存未命中:

\begin{shell}
perf stat -e cache-references,cache-misses ./13x07-false_sharing
\end{shell}

\begin{shell}
Performance counter stats for './13x07-false_sharing':

     251,277,877     cache-references
     242,797,999     cache-misses
                     # 96.63% of all cache refs
\end{shell}

大多数缓存引用都是缓存未命中。由于虚假分享,这是预期的。

现在,如果我们运行 sum\_random\_aligned,结果会大不相同:

\begin{shell}
Performance counter stats for './13x07-false_sharing':

          851,506    cache-references
          231,703    cache-misses
                     # 27.21% of all cache refs
\end{shell}

缓存引用和缓存未命中的数量都少得多。这是因为无需不断更新所有核心中的缓存来保持缓存一致性。

在本节中,我们了解了多线程代码最常见的性能问题之一:虚假共享。我们了解了有虚假共享和没有虚假共享的函数示例,以及虚假共享对性能的负面影响。

在下一节中,我们将回到第 5 章中实现的 SPSC 无锁队列并改进其性能。















