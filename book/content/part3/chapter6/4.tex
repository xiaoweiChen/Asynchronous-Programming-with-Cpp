
现在已经了解了一些创建异步程序的新构建块,让我们为一些实际场景构建解决方案。本节中,介绍如何执行以下操作:

\begin{itemize}
\item
取消异步操作

\item
返回合并结果

\item
链接异步操作并创建管道

\item
创建线程安全的单生产者单消费者 (SPSC) 任务队列
\end{itemize}

\mySubsubsection{6.4.1.}{取消异步操作}

正如之前看到的, future 提供了在等待结果之前检查完成或超时的功能。这可以通过检查 std::future、 wait\_for() 或 wait\_until() 函数返回的 std::future\_status 对象来完成。

通过将 Future 与取消标志(通过 std::atomic\_bool)或超时等机制相结合,可以在必要时优雅地终止长时间运行的任务。只需使用 wait\_for() 和 wait\_until() 函数即可实现超时取消。

使用取消标志或令牌取消任务可以通过传递对定义为 std::atomic\_bool 的取消标志的引用来实现,其中调用者线程将其值设置为 true 以请求取消任务,而工作线程会定期检查此标志以及它是否已设置。如果已设置,将正常退出并执行要完成的清理工作。

首先定义一个长时间运行的任务函数:

\begin{cpp}
const int CHECK_PERIOD_MS = 100;

bool long_running_task(int ms,
            const std::atomic_bool& cancellation_token) {
    while (ms > 0 && !cancellation_token) {
        ms -= CHECK_PERIOD_MS;
        std::this_thrsead::sleep_for(100ms);
    }
    return cancellation_token;
}
\end{cpp}

long\_running\_task 函数接受一个以毫秒 (ms) 为单位的任务运行周期和一个表示取消标记的原子布尔值 (cancellation\_token) 的引用作为参数,函数将定期检查取消标记是否设置为 true。当运行周期过去或取消标记设置为 true 时,线程将退出。

主线程中可以使用两个封装好的task对象来执行该函数, task1在线程t1中运行,时长500ms, task2在线程t2中运行,时长1s,二者共享同一个取消标记:

\begin{cpp}
std::atomic_bool cancellation_token{false};
std::cout << "Starting long running tasks...\n";

std::packaged_task<bool(int, const std::atomic_bool&)>
                task1(long_running_task);

std::future<bool> result1 = task1.get_future();
std::jthread t1(std::move(task1), 500,
                std::ref(cancellation_token));

std::packaged_task<bool(int, const std::atomic_bool&)>
                task2(long_running_task);
std::future<bool> result2 = task2.get_future();
std::jthread t2(std::move(task2), 1000,
                std::ref(cancellation_token));

std::cout << "Cancelling tasks after 600 ms...\n";
this_thread::sleep_for(600ms);
cancellation_token = true;

std::cout << "Task1, waiting for 500 ms. Cancelled = "
          << std::boolalpha << result1.get() << "\n";
std::cout << "Task2, waiting for 1 second. Cancelled = "
          << std::boolalpha << result2.get() << "\n";
\end{cpp}

两个任务启动后,主线程休眠 600 毫秒。唤醒时,将取消标记设置为 true。此时,任务 1 已经完成,但任务 2 仍在运行。因此,任务 2 取消。

这个解释与获得的输出一致:

\begin{shell}
Starting long running tasks...
Cancelling tasks after 600 ms...
Task1, waiting for 500 ms. Cancelled = false
Task2, waiting for 1 second. Cancelled = true
\end{shell}

接下来看看如何将几个异步计算结果合并到一个 Future 中。

\mySubsubsection{6.4.2.}{返回合并结果}

异步编程中的另一种常见方法是,使用多个 Promise 和 Future 将复杂任务分解为较小的独立子任务。每个子任务都可以在单独的线程中启动,其结果存储在相应的 Promise 中。然后,主线程可以使用 Future 等待所有子任务完成并合并以获得结果。

这种方法有助于实现多个独立任务的并行处理,从而可以有效利用多个核心来加快计算速度。

看一个模拟值计算和 I/O 操作的任务示例。我们希望该任务返回一个元组,其中包含结果、计算值(int 值)和信息从文件中读取为字符串对象。因此,定义 CombineFunc 函数,该函数接受一个 CombineProm Prom 作为参数,该Promise包含一个包含结果类型的元组。

此函数将创建两个线程, computeThread 和 fetchData,各自的 computeProm 和 fetchProm,以及 computeFut 和 fetchFut:

\begin{cpp}
void combineFunc(std::promise<std::tuple<int,
                 std::string>> combineProm) {
    try {
        // Thread to simulate computing a value.
        std::cout << "Starting computeThread...\n";
        auto computeVal = [](std::promise<int> prom)
                             mutable {
            std::this_thread::sleep_for(1s);
            prom.set_value(42);
        };
        std::promise<int> computeProm;
        auto computeFut = computeProm.get_future();
        std::jthread computeThread(computeVal,
                                std::move(computeProm));

        // Thread to simulate downloading a file.
        std::cout << "Starting dataThread...\n";
        auto fetchData = [](
        std::promise<std::string> prom) mutable {
            std::this_thread::sleep_for(2s);
            prom.set_value("data.txt");
        };
        std::promise<std::string> fetchProm;
        auto fetchFut = fetchProm.get_future();
        std::jthread dataThread(fetchData,
                                std::move(fetchProm));

        combineProm.set_value({
            computeFut.get(),
            fetchFut.get()
        });
    } catch (...) {
        combineProm.set_exception(
                    std::current_exception());
    }
}
\end{cpp}

可以看到,两个线程将异步且独立地执行,生成结果并将其存储在各自的 Promise 中。

组合Promise的设置方式是,在每个未来上调用 get() 函数,并将它们的结果组合成一个元组,该元组用于通过调用其 set\_value() 函数来设置组合Promise的值:

\begin{cpp}
combineProm.set_value({computeFut.get(), fetchFut.get()});
\end{cpp}

通过使用线程并设置 CombineProm 及其 CombineFut,可以像往常一样调用 CombineFunc 任务。在此 Future 上调用 get() 函数将返回一个元组:

\begin{cpp}
std::promise<std::tuple<int, std::string>> combineProm;
auto combineFuture = combineProm.get_future();
std::jthread combineThread(combineFunc,
                           std::move(combineProm));

auto [data, file] = combineFuture.get();
std::cout << "Value [ " << data
          << " ] File [ « << file << « ]\n»;
\end{cpp}

运行该示例将显示以下结果:

\begin{shell}
Creating combined promise...
Starting computeThread...
Starting dataThread...
Value [ 42 ] File [ data.txt ]
\end{shell}

现在,让继续了解如何使用Promise和Future创建管道。

\mySubsubsection{6.4.3.}{链接异步操作}

Promise 和 Future 可以串联在一起,按顺序执行多个异步操作。可以创建一个管道,其中一个 Future 的结果将成为下一个操作的 Promise 的输入。这允许组合复杂的异步工作流,其中一个任务的输出将输入到下一个任务中。

此外,可以在管道中进行分支,并保持某些任务处于关闭状态,直到需要时才执行。这可以通过使用具有延迟执行的 Future 来实现,这在计算成本很高但结果可能并不总是需要的情况下很有用。因此,可以使用 Future 异步启动计算,并仅在需要时检索结果。由于只能使用 std::async 创建具有延迟状态的 Future,将在下一章中讨论这个问题。

本节中,将重点创建以下任务图:

\myGraphic{0.8}{content/part3/chapter6/images/3.png}{图 6.3 – 管道示例}

首先定义一个名为 Task 的模板类,该类接受可调用函数作为模板参数,定义要执行的函数。该类还创建与依赖任务共享Future的任务。这些任务将使用共享的Future等待前置任务通过在相关Promise中调用 set\_value() 来发出完成信号,然后再运行自己的任务:

\begin{cpp}
#define sync_cout std::osyncstream(std::cout)

template <typename Func>
class Task {
    public:
    Task(int id, Func& func)
        : id_(id), func_(func), has_dependency_(false) {
        sync_cout << "Task " << id
        << " constructed without dependencies.\n";
        fut_ = prom_.get_future().share();
    }

    template <typename... Futures>
    Task(int id, Func& func, Futures&&... futures)
        : id_(id), func_(func), has_dependency_(true) {
        sync_cout << "Task " << id
        << " constructed with dependencies.\n";
        fut_ = prom_.get_future().share();
        add_dependencies(
        std::forward<Futures>(futures)...);
    }

    std::shared_future<void> get_dependency() {
        return fut_;
    }

    void operator()() {
        sync_cout << "Running task " << id_ << '\n';
        wait_completion();
        func_();
        sync_cout << "Signaling completion of task "
                  << id_ << '\n';
        prom_.set_value();
    }

private:
    template <typename... Futures>
    void add_dependencies(Futures&&... futures) {
        (deps_.push_back(futures), ...);
    }

    void wait_completion() {
        sync_cout << "Waiting completion for task "
        << id_ << '\n';
        if (!deps_.empty()) {
            for (auto& fut : deps_) {
                if (fut.valid()) {
                    sync_cout << "Fut valid so getting "
                              << "value in task "
                              << id_ << '\n';
                    fut.get();
                }
            }
        }
    }

private:
    int id_;
    Func& func_;
    std::promise<void> prom_;
    std::shared_future<void> fut_;
    std::vector<std::shared_future<void>> deps_;
    bool has_dependency_;
};
\end{cpp}

一步一步拆解这个 Task 类是如何实现的 。

有两个构造函数:一个用于初始化与其他任务没有依赖关系的 Task 类型对象,另一个模板化构造函数用于初始化具有可变数量依赖任务的任务。两者都初始化标识符 (id\_)、执行任务时调用的函数 (func\_)、布尔值变量指示任务是否具有依赖项(has\_dependency\_),以及共享Future fut\_,与依赖于此任务的任务共享。fut\_ 是从用于发出任务完成信号的 prom\_ 中检索的,模板化构造函数还调用 add\_dependencies 函数转发作为参数传递的Future,这些Future将存储在 deps\_ 向量数组中。

get\_dependency() 函数仅返回依赖任务用于等待当前任务完成的共享Future。

最后, operator() 通过调用 wait\_completion() 等待先前任务的完成,这会检查存储在 deps\_ 向量数组中的每个共享Future是否有效,并通过调用 get() 等待结果准备就绪。当所有共享Future都准备就绪,即所有先前任务都已完成,就会调用 func\_ 函数来运行任务,然后通过调用 set\_value() 将 prom\_ promise 设置为就绪,从而触发依赖任务。

主线程中,定义的管道如下,创建如图 6.3 所示的图:

\begin{cpp}
auto sleep1s = []() { std::this_thread::sleep_for(1s); };
auto sleep2s = []() { std::this_thread::sleep_for(2s); };

Task task1(1, sleep1s);
Task task2(2, sleep2s, task1.get_dependency());
Task task3(3, sleep1s, task2.get_dependency());
Task task4(4, sleep2s, task2.get_dependency());
Task task5(5, sleep2s, task3.get_dependency(),
                       task4.get_dependency());
\end{cpp}

然后,需要通过触发所有任务并调用它们的 operator() 来启动管道。由于 task1 没有依赖项,将立即开始运行。所有其他任务都将等待其上游任务完成其工作:

\begin{cpp}
sync_cout << "Starting the pipeline..." << std::endl;
task1();
task2();
task3();
task4();
task5();
\end{cpp}

最后,需要等待管道完成所有任务的执行。可以通过简单地等待最后一个任务 task 5 返回的共享 Future 准备就绪来实现:

\begin{cpp}
sync_cout << "Waiting for the pipeline to finish...\n";
auto finish_pipeline_fut = task5.get_dependency();
finish_pipeline_fut.get();
sync_cout << "All done!" << std::endl;
\end{cpp}

以下是运行此示例的输出:

\begin{shell}
Task 1 constructed without dependencies.
Getting future from task 1
Task 2 constructed with dependencies.
Getting future from task 2
Task 3 constructed with dependencies.
Getting future from task 2
Task 4 constructed with dependencies.
Getting future from task 4
Getting future from task 3
Task 5 constructed with dependencies.
Starting the pipeline...
Running task 1
Waiting completion for task 1
Signaling completion of task 1
Running task 2
Waiting completion for task 2
Fut valid so getting value in task 2
Signaling completion of task 2
Running task 3
Waiting completion for task 3
Fut valid so getting value in task 3
Signaling completion of task 3
Running task 4
Waiting completion for task 4
Fut valid so getting value in task 4
Signaling completion of task 4
Running task 5
Waiting completion for task 5
Fut valid so getting value in task 5
Fut valid so getting value in task 5
Signaling completion of task 5
Waiting for the pipeline to finish...
Getting future from task 5
All done!
\end{shell}

必须注意这种方法可能出现的一些问题。首先,依赖关系图必须是有向无环图 (DAG) ,因此依赖任务之间不能有循环或环路。否则,就会发生死锁,因为任务可能在等待未来发生的任务,但尚未启动。此外,需要足够的线程来同时运行所有任务或按顺序启动任务;否则,线程等待尚未启动的任务完成也可能导致死锁。

这种方法的一个常见用例可以在 MapReduce 算法中找到,其中大型数据集在多个节点上并行处理,并且可以使用Future和线程来同时执行 map 和 Reduce 任务,从而实现高效的分布式数据处理。

\mySubsubsection{6.4.4.}{线程安全的 SPSC 任务队列}

最后一个例子中,将展示如何使用Promise和Future来创建 SPSC 队列。

生产者线程会为要添加到队列中的每个项目创建一个Promise。消费者线程会等待从空队列槽中获取的Future。生产者添加完项目后,它会在相应的Promise上设置值,通知等待的消费者。这允许在线程之间进行高效的数据交换,同时保持线程安全。

首先定义线程安全的队列类:

\begin{cpp}
template <typename T>
class ThreadSafeQueue {
public:
    void push(T value) {
        std::lock_guard<std::mutex> lock(mutex_);
        queue_.push(std::move(value));
        cond_var_.notify_one();
    }

    T pop() {
        std::unique_lock<std::mutex> lock(mutex_);
        cond_var_.wait(lock, [&]{
            return !queue_.empty();
        });
        T value = std::move(queue_.front());
        queue_.pop();
        return value;
    }

private:
    std::queue<T> queue_;
    std::mutex mutex_;
    std::condition_variable cond_var_;
};
\end{cpp}

此示例中,仅使用互斥锁在推送或弹出元素时对所有队列数据结构进行互斥。我们希望保持此示例简单,并专注于Promise和Future的交互。更好的方法可能是使用向量或循环数组,并使用互斥锁控制对队列中各个元素的访问。

队列还使用条件变量 cond\_var\_,在尝试弹出元素时等待队列是否为空,并在推送元素时通知一个等待线程。通过移动元素,可以将元素移入和移出队列。这是必要的,因为队列将存储Future,而Future可移动,但不可复制。

线程安全队列将用于定义一个存储Future的任务队列:

\begin{cpp}
using TaskQueue = ThreadSafeQueue<std::future<int>>;
\end{cpp}

然后,定义一个函数producer,接受对队列的引用,以及将要生成的值 val。此函数仅创建一个Promise,从Promise中检索Future,并将该Future推送到队列中,通过让线程等待随机数毫秒来模拟运行并生成值 val 的任务。最后,该值存储在Promise中:

\begin{cpp}
void producer(TaskQueue& queue, int val) {
    std::promise<int> prom;
    auto fut = prom.get_future();
    queue.push(std::move(fut));

    std::this_thread::sleep_for(
            std::chrono::milliseconds(rand() % MAX_WAIT));

    prom.set_value(val);
}
\end{cpp}

通信通道的另一端,消费者函数接受对同一队列的引用。同样,可通过等待随机的毫秒数来模拟在消费者端运行的任务。然后,从队列中弹出一个 Future,并检索其结果:

\begin{cpp}
void consumer(TaskQueue& queue) {
    std::this_thread::sleep_for(
            std::chrono::milliseconds(rand() % MAX_WAIT));

    std::future<int> fut = queue.pop();
    try {
        int result = fut.get();
        std::cout << "Result: " << result << "\n";
    } catch (const std::exception& e) {
        std::cerr << "Exception: " << e.what() << '\n';
    }
}
\end{cpp}

对于此示例,将使用以下常量:

\begin{cpp}
const unsigned VALUE_RANGE = 1000;
const unsigned RESULTS_TO_PRODUCE = 10; // Numbers of items to produce.
const unsigned MAX_WAIT = 500; // Maximum waiting time (ms) when producing items.
\end{cpp}

主线程中,启动了两个线程;第一个线程运行生产者函数producerFunc,将一些Future推送到队列中,而第二个线程运行消费者函数consumerFunc,从队列中使用元素:

\begin{cpp}
TaskQueue queue;

auto producerFunc = [](TaskQueue& queue) {
    auto n = RESULTS_TO_PRODUCE;
    while (n-- > 0) {
        int val = rand() % VALUE_RANGE;
        std::cout << "Producer: Sending value " << val
                  << std::endl;
        producer(queue, val);
    }
};

auto consumerFunc = [](TaskQueue& queue) {
    auto n = RESULTS_TO_PRODUCE;
    while (n-- > 0) {
        std::cout << "Consumer: Receiving value"
                  << std::endl;
        consumer(queue);
    }
};

std::jthread producerThread(producerFunc, std::ref(queue));
std::jthread consumerThread(consumerFunc, std::ref(queue));
\end{cpp}

以下是执行此代码的示例输出:

\begin{shell}
Producer: Sending value 383
Consumer: Receiving value
Producer: Sending value 915
Result: 383
Consumer: Receiving value
Producer: Sending value 386
Result: 915
Consumer: Receiving value
Producer: Sending value 421
Result: 386
Consumer: Receiving value
Producer: Sending value 690
Result: 421
Consumer: Receiving value
Producer: Sending value 926
Producer: Sending value 426
Result: 690
Consumer: Receiving value
Producer: Sending value 211
Result: 926
Consumer: Receiving value
Result: 426
Consumer: Receiving value
Producer: Sending value 782
Producer: Sending value 862
Result: 211
Consumer: Receiving value
Result: 782
Consumer: Receiving value
Result: 862
\end{shell}

使用像这样的生产者-消费者队列,消费者和生产者是分离的,并且它们的线程异步通信,从而允许生产者和消费者在另一方生成或处理值时的其他工作。


