当程序运行的结果取决于其指令的执行顺序时,就会发生条件竞争。举一个简单的例子,展示条件竞争是如何发生的,然后了解如何解决这个问题。

下面的代码中,计数器全局变量由两个同时运行的线程增加:

\begin{cpp}
#include <iostream>
#include <thread>

int counter = 0;

int main() {
    auto func = [] {
        for (int i = 0; i < 1000000; ++i) {
            counter++;
        }
    };

    std::thread t1(func);
    std::thread t2(func);

    t1.join();
    t2.join();

    std::cout << counter << std::endl;
    return 0;
}
\end{cpp}

运行上述代码三次后,我们得到以下计数器值:

\begin{shell}
1056205
1217311
1167474
\end{shell}

这里看到两个问题:首先,计数器的值不正确;其次,程序每次执行都会以不同的计数器值结束。结果不确定,而且大多数情况下都不正确。

这个场景涉及两个线程 t1 和 t2,同时运行并修改同一个变量,该变量本质上是某个内存区域。这应该可以正常工作,因为只有一行代码会增加计数器值,从而修改内存内容(顺便说一句,使用像 counter++ 中的后增量运算符或像 ++counter 中的前增量运算符都没有关系;结果同样是错误的)。

仔细查看上面的代码,仔细研究以下行:

\begin{cpp}
counter++;
\end{cpp}

分三步增加计数器:

\begin{itemize}
\item
计数器变量存储的内存地址的内容被加载到 CPU 寄存器中。本例中, int 数据类型从内存加载到 CPU 寄存器中。

\item
寄存器中的值增加一。

\item
寄存器中的值保存在计数器变量的内存地址中。
\end{itemize}

现在,考虑两个线程尝试同时增加计数器的可能情况,来看看表 4.1:

% Please add the following required packages to your document preamble:
% \usepackage{longtable}
% Note: It may be necessary to compile the document several times to get a multi-page table to line up properly
\begin{longtable}{|l|l|}
\hline
线程 1                          & 线程 2                          \\ \hline
\endfirsthead
%
\endhead
%
{[}1{]} 将计数器值装入寄存器 & {[}3{]} 将计数器值装入寄存器 \\ \hline
{[}2{]} 增加寄存器的值  & {[}5{]} 增加寄存器的值  \\ \hline
{[}4{]} 寄存器计数器 & {[}6{]} 寄存器计数器 \\ \hline
\end{longtable}

\begin{center}
表 4.1:两个线程同时增加计数器
\end{center}

线程 1 执行 [1] 并将计数器的当前值(假设为 1)加载到 CPU 寄存器中。然后,它将寄存器中的值加1 [2](现在,寄存器值为 2)。

线程 2 被安排执行,并且 [3] 将计数器的当前值(记住 - 它还没有被修改,所以它仍然为 1 )加载到 CPU 寄存器中。

现在,线程 1 再次执行,并且 [4] 将更新的值存储到内存中。计数器的值现在等于 2 。
最后,线程 2 再次执行 [5] 和 [6]。寄存器值增加1,然后将2存储在内存中。计数器变量只增加了一次,而它应该增加两次,其值应该是三。

发生上述问题是,计数器上的递增操作不是原子的。如果每个线程都可以在不中断的情况下,执行递增计数器变量所需的三条指令,则计数器将按预期递增两次。但根据操作执行的顺序,结果可能会有所不同,这称为条件竞争。

为了避免条件竞争,需要确保以受控的方式访问和修改共享资源。实现此目的的一种方法是使用锁。锁是一种同步原语,一次只允许一个线程访问共享资源。当线程想要访问共享资源时,必须首先获取锁。线程获取了锁后,就可以访问共享资源而不会受到其他线程的干扰。当线程访问完共享资源后,必须释放锁,以便其他线程可以访问相应的共享资源。

避免条件竞争的另一种方法是使用原子操作,可保证在单个不可分割的步骤中执行的操作。在执行原子操作时,没有其他线程可以干扰该操作。原子操作通常使用设计为不可分割的硬件指令来实现。

本节中,介绍了多线程代码产生的最常见和最重要的问题:条件竞争。根据执行操作的顺序,结果可能会有所不同。考虑到这个问题,将在下一节中研究如何解决它。





























