深入研究使用 C++ 进行并行编程之前,我们将重点了解构建并行软件的不同方法,以及软件如何与机器硬件交互的一些基础知识。

本章中,将介绍并行编程,以及在开发高效、响应迅速、可扩展并发和异步软件时,可以使用的不同范式和模型。

对开发并行软件的不同方法进行分类时,有很多种方法可以对概念和方法进行分组。由于本书重点介绍使用 C++ 构建的软件,因此可以将不同的并行编程范例划分为以下几种:并发、异步编程、并行编程、反应式编程、数据流、多线程编程和事件驱动编程。

特定范式可能比其他范式更适合解决给定场景,了解不同的范式将有助于分析问题并缩小搜索最佳解决方案的范围。

本章中,将讨论以下主题:

\begin{itemize}
\item
什么是并行编程?为什么重要?

\item
有哪些不同的并行编程范式?为什么需要了解它们?

\item
将在本书中学到什么?
\end{itemize}