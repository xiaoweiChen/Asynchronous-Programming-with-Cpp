
我们已经了解了 C++ 标准库的原子特性,例如原子类型和操作以及内存模型和排序。现在我们将看到一个使用原子实现 SPSC 无锁队列的完整示例。

这个队列的主要特点如下:

\begin{itemize}
\item
SPSC:此队列设计为使用两个线程工作,一个线程将元素推送到队列,另一个线程从
队列中获取元素。

\item
有界:此队列具有固定大小。我们需要一种方法来检查队列何时达到其容量以及何时没
有元素。

\item
无锁:此队列使用在现代 Intel x64 CPU 上始终无锁的原子类型。
\end{itemize}

在开始开发队列之前,请记住无锁与无等待不同(还请记住无等待并不能完全消除等待; 它只是确保每个队列推送/弹出所需的步骤数有限制)。第 13 章将讨论一些主要影响性能的方面。在该章中,我们还将优化队列的性能。目前,在本章中,我们将构建一个正确且性能良好的 SPSC 无锁队列——我们将在后面展示如何提高其性能。

我们在第 4 章中使用互斥锁和条件变量创建了一个 SPSC 队列,消费者线程和生产者线程可以安全访问该队列。本章将使用原子操作来实现相同的目标。
我们将使用相同的数据结构 std::vector<T> 来存储队列中的项目,其大小固定,即 2 的幂。

这样,我们可以提高性能并快速找到下一个头和尾索引,而无需使用需要除法指令的模数运算符。当使用无锁原子类型来获得更好的性能时,我们需要注意影响性能的一切。

\mySubsubsection{5.6.1.}{为什么我们使用2的幂的缓冲区大小?}

我们将使用一个向量来保存队列项。该向量将具有固定大小,例如 N。我们将使向量的行为类似于环形缓冲区,这意味着访问向量中元素的索引将在结束后循环回到开头。第一个元素将跟在最后一个元素后面。正如我们在第 4 章中学到的,我们可以使用模运算符来做到这一点:

\begin{cpp}
size_t next_index = (curr_index + 1) % N;
\end{cpp}

例如,如果大小为 4 个元素,则下一个元素的索引将按照上述代码计算。对于最后一个索引,我们有以下代码:

\begin{cpp}
next_index = (3 + 1) % 4 = 4 % 4 = 0;
\end{cpp}

因此,正如我们所说,向量将是一个环形缓冲区,因为在最后一个元素之后,我们将返回到第一个元素,然后是第二个元素,依此类推。

我们可以使用此方法获取任何缓冲区大小 N 的下一个索引。但为什么我们只使用 2 的幂的大小?答案很简单:性能。模数 (\%) 运算符需要除法指令,这很昂贵。当大小 N 是 2 的幂时,我们可以执行以下操作:

\begin{cpp}
size_t next_index = curr_index & (N – 1);
\end{cpp}

这比使用模运算符要快得多。

\mySubsubsection{5.6.2.}{缓冲区访问同步}

要访问队列缓冲区,我们需要两个索引:

\begin{itemize}
\item
head:当前要读取的元素的索引

\item
tail:下一个要写入的元素的索引
\end{itemize}

消费者线程将使用头部索引进行读写。生产者线程将使用尾部索引进行读写。我们需要同步对这些变量的访问,因为:

\begin{itemize}
\item
只有一个线程(消费者)写入 head,这意味着它可以以宽松的内存顺序读取它,因为它始终看到自己的更改。读取 tail 由读取器线程完成,它需要与生产者对 tail 的写入同步,因此它需要获取内存顺序。我们可以对所有事情使用顺序一致性,但我们想要最好的性能。当消费者线程写入 head 时,它需要与生产者对它的读取同步,因此它需要释放内存顺序。

\item
对于 tail,只有生产者线程会写入它,因此我们可以使用宽松的内存顺序来读取它,但我们需要释放内存顺序来写入它,并将其与消费者线程的读取同步。为了与消费者线程的写入同步,我们需要获取内存顺序来读取 head。
\end{itemize}

队列类的成员变量如下:

\begin{cpp}
const std::size_t capacity_; // power of two buffer size
std::vector<T> buffer_; // buffer to store queue items handled like a
ring buffer
std::atomic<std::size_t> head_{ 0 };
std::atomic<std::size_t> tail_{ 0 };
\end{cpp}

在本节中,我们了解了如何同步对队列缓冲区的访问。

\mySubsubsection{5.6.3.}{将元素推送到队列}

一旦我们决定了队列的数据表示以及如何同步对其元素的访问,让我们实现将元素推送到队列的函数:

\begin{cpp}
bool push(const T& item) {
    std::size_t tail =
        tail_.load(std::memory_order_relaxed);

    std::size_t next_tail =
        (tail + 1) & (capacity_ - 1);

    if (next_tail != head_.load(std::memory_order_acquire)) {
        buffer_[tail] = item;
        tail_.store(next_tail, std::memory_order_release);
        return true;
    }

    return false;
}
\end{cpp}

当前尾部索引,即数据项(如果可能)被推送到队列的缓冲区槽,在行 [1] 中被原子读取。
如前所述,此读取可以使用 std::memory\_order\_relaxed,因为只有生产者线程会更改此变量,并且它是唯一调用推送的线程。

行 [2] 计算下一个索引模容量(记住缓冲区是一个环)。我们需要这样做来检查队列是否已满。

我们在行 [3] 中执行检查。我们首先使用 std::memory\_order\_acquire 原子地读取 head 的当前值,因为我们希望生产者线程观察消费者线程对此变量所做的修改。然后我们将其值与下一个 head 索引进行比较。

如果下一个尾部值等于当前头值,那么(按照我们的惯例)队列已满,我们返回 false。
如果队列未满,行 [4] 将数据项复制到队列缓冲区。这里值得一提的是,数据复制不是原子的。
行 [5] 原子地将新的尾部索引值写入 tail\_。然后,使用 std::memory\_order\_release 使更改对使用 std::memory\_order\_acquire 原子地读取此变量的消费者线程可见。


\mySubsubsection{5.6.4.}{从队列中弹出元素}

现在我们看看pop函数是如何实现的:

\begin{cpp}
bool pop(T& item) {
    std::size_t head =
        head_.load(std::memory_order_relaxed);

    if (head == tail_.load(std::memory_order_acquire)) {
        return false;
    }

    item = buffer_[head];

    head_.store((head + 1) & (capacity_ - 1), std::memory_order_release);

    return true;
}
\end{cpp}

行 [1] 原子地读取 head\_ 的当前值(要读取的下一个项的索引)。我们使用 std::memory\_order\_relaxed,因为 head\_ 变量仅由消费者线程修改,因此不需要强制执行顺序,而消费者线程是唯一调用 pop 的线程。

行 [2] 检查队列是否为空。如果 head\_ 的当前值与 tail\_ 的当前值相同,则队列为空,函数返回 false。我们使用 std::memory\_order\_acquire 原子读取 tail\_ 的值,以查看生产者线程对 tail\_ 所做的最新更改。

行 [3] 将数据从队列复制到作为 pop 参数传递的项目引用。同样,此复制不是原子的。

最后,行 [4] 更新 head\_ 的值。同样,我们使用 std::memory\_order\_release 原子地写入值,以便消费者线程查看消费者线程对 head\_ 所做的更改。

SPSC无锁队列实现的代码如下

\begin{cpp}
#include <atomic>
#include <cassert>
#include <iostream>
#include <vector>
#include <thread>

template<typename T>
class spsc_lock_free_queue {
    public:
    // capacity must be power of two to avoid using modulo operator
    when calculating the index
    explicit spsc_lock_free_queue(size_t capacity) : capacity_(capacity), buffer_(capacity) {
        assert((capacity & (capacity - 1)) == 0 && "capacity must be a
        power of 2");
    }

    spsc_lock_free_queue(const spsc_lock_free_queue &) = delete;

    spsc_lock_free_queue &operator=(const spsc_lock_free_queue &) = delete;

    bool push(const T &item) {
        std::size_t tail = tail_.load(std::memory_order_relaxed);
        std::size_t next_tail = (tail + 1) & (capacity_ - 1);
        if (next_tail != head_.load(std::memory_order_acquire)) {
            buffer_[tail] = item;
            tail_.store(next_tail, std::memory_order_release);
            return true;
        }

        return false;
    }

    bool pop(T &item) {
        std::size_t head = head_.load(std::memory_order_relaxed);
        if (head == tail_.load(std::memory_order_acquire)) {
            return false;
        }

        item = buffer_[head];
        head_.store((head + 1) & (capacity_ - 1), std::memory_order_release);
        return true;
    }
private:
    const std::size_t capacity_;
    std::vector<T> buffer_;
    std::atomic<std::size_t> head_{0};
    std::atomic<std::size_t> tail_{0};
};
\end{cpp}

完整示例的代码可以在以下书籍存储库中找到:  \url{https://github.com/PacktPublishing/Asynchronous-Programming-in-CPP/blob/main/Chapter_05/5x09-SPSC_lock_free_queue.cpp}

在本节中,我们实现了 SPSC 无锁队列作为原子类型和操作的应用。在第 13 章中,我们将重新讨论此实现并改进其性能。




