
进程和线程代表并发执行代码的两种基本方式,但它们在操作和资源管理方面存在很大差异。进程是正在运行的程序的实例,它拥有一组私有资源,包括内存、文件描述符和执行上下文。进程彼此隔离,为整个系统提供了强大的稳定性,因为一个进程的故障通常不会影响其他进程。

线程是计算机科学中的一个基本概念,代表在单个进程内执行多个任务的一种轻量且高效的方式。与具有自己的私有内存空间和资源的独立实体进程不同,线程与其所属的进程紧密交织在一起。这种密切的关系允许线程共享相同的内存空间和资源,包括文件描述符、 堆内存以及进程分配的任何其他全局数据结构。

线程的一个主要优势是它们能够高效地通信和共享数据。由于进程内的所有线程共享相同的内存空间,因此它们可以直接访问和修改公共变量,而无需复杂的 IPC 机制。这种共享环境可实现快速数据交换,并有助于实现并发算法和数据结构。

然而,共享同一内存空间也带来了管理共享资源访问的挑战。为了防止数据损坏并确保共享数据的完整性,线程必须采用同步机制,例如锁、信号量或互斥锁。这些机制强制执行访问共享资源的规则和协议,确保在任何给定时间只有一个线程可以访问特定资源。

有效的同步在多线程编程中至关重要,以避免竞争条件、死锁和其他与并发相关的问题。

为了应对这些挑战,人们开发了各种同步原语和技术。其中包括互斥锁(提供对共享资源的独占访问)、信号量(允许对有限数量的资源进行受控访问)和条件变量(使线程能够等待特定条件得到满足后再继续执行)
。
通过谨慎管理同步并采用适当的并发模式,开发人员可以利用线程的强大功能,在其应用程序中实现高性能和可扩展性。线程特别适合可并行化的任务,例如图像处理、科学模拟和 Web 服务器,其中可以同时执行多个独立计算。

如前所述,线程是系统线程。这意味着它们由内核创建和管理。但是,有些场景(我们将在第 8 章中深入探讨)需要大量线程。在这种情况下,系统可能没有足够的资源来创建大量系统线程。解决这个问题的方法是使用用户线程。实现用户线程的一种方法是通过协程,协程自 C++20 起已包含在 C++ 标准中。

协程是 C++ 中一个相对较新的功能。协程可以定义为可在特定点暂停和恢复的函数,从而允许在单个线程内进行协作式多任务处理。与从头到尾不间断运行的标准函数不同,协程可以暂停执行并将控制权交还给调用者,后者稍后可以从暂停点恢复协程。

协程比系统线程轻量得多。这意味着它们可以更快地创建和销毁,并且所需的开销更少。

协程是协作性的,这意味着它们必须明确地将控制权移交给调用者才能切换执行上下文。

在某些情况下,这可能是一个缺点,但也可能是一个优点,因为它使用户程序能够更好地控制协程的执行。
协程可用于创建各种不同的并发模式。例如,协程可用于实现任务,这些任务是可以调度并同时运行的轻量级工作单元。协程还可用于实现通道,这些通道是可以在它们之间传递数据的通信通道。

协程可以分为有栈和无栈两类。 C++20 协程是无栈的。我们将在第 8 章深入了解这些概念。

总体而言,协程是 C++ 中创建并发程序的强大工具。它们轻量级、协作性强,可用于实现各种不同的并发模式。它们不能完全用于实现并行性,因为协程仍然需要 CPU 执行上下文,而这只能由线程提供。

\mySubsubsection{2.3.1.}{线程生命周期}

系统线程(通常称为轻量级进程)的生命周期包括从创建到终止的各个阶段。每个阶段在并发编程环境中管理和利用线程方面都发挥着至关重要的作用:

\begin{enumerate}
\item
创建:此阶段始于系统中创建新线程时。创建过程涉及使用函数,该函数需要几个参数。一个关键参数是线程的属性,例如其调度策略、堆栈大小和优先级。另一个重要参数是线程将执行的函数,称为启动例程。成功创建后,线程将分配自己的堆栈和其他资源。

\item
执行:创建后,线程开始执行其指定的启动例程。在执行期间,线程可以独立执行各种任务,或在必要时与其他线程交互。线程还可以创建和管理自己的局部变量和数据结构,使它们自成一体并能够同时执行特定任务。

\item
同步:为了确保有序访问共享资源并防止数据损坏,线程采用同步机制。常见的同步原语包括锁、信号量和屏障。适当的同步允许线程协调其活动,避免并发编程中可能出现的竞争条件、死锁和其他问题。

\item
终止:线程可以通过多种方式终止。它可以显式调用函数来终止自身。它也可以通过从其启动例程返回来终止。在某些情况下,线程可以被另一个使用该函数的线程取消。终止后,系统将回收分配给该线程的资源,并释放该线程持有的任何待处理操作或锁。
\end{enumerate}

了解系统线程的生命周期对于设计和实现并发程序至关重要。通过仔细管理线程的创建、 执行、同步和终止,开发人员可以创建高效且可扩展的应用程序,以充分利用并发的优势。

\mySubsubsection{2.3.2.}{线程调度}

系统线程由操作系统内核的调度程序管理,是抢先调度的。调度程序根据线程优先级、分配时间或互斥阻塞等因素决定何时在线程之间切换执行。这种由内核控制的上下文切换可能会产生大量开销。上下文切换的高成本,加上每个线程的资源使用量(例如其自己的堆栈),使得协程成为某些应用程序的更高效替代方案,因为我们可以在单个线程中运行多个协程。

协程具有多种优势。首先,它们减少了与上下文切换相关的开销。由于协程的 Yield 或 Awa it 上的上下文切换由用户空间代码而不是内核处理,因此该过程更加轻量和高效。这可以显著提高性能,尤其是在频繁发生上下文切换的情况下。

协程还提供了对线程调度的更大控制。开发人员可以根据其应用程序的特定要求定义自定义调度策略。这种灵活性允许微调线程管理、优化资源利用率并实现所需的性能特征。

协程的另一个重要特性是,与系统线程相比,协程通常更轻量。协程不维护自己的堆栈,这是一个很大的资源消耗优势,使其适合资源受限的环境。

总体而言,协程提供了一种更高效、更灵活的线程管理方法,尤其是在需要频繁切换上下文或对线程调度进行细粒度控制的情况下。线程可以访问内存进程,并且该内存由所有线程共享,因此我们需要小心并控制内存访问。这种控制是通过称为同步原语的不同机制实现的。


































