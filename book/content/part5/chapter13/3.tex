在本节中,我们将研究多线程应用程序中的一个常见问题,称为错误共享。

我们已经知道,多线程应用程序的理想实现是最小化不同线程之间共享的数据。理想情况下,我们应该只为读取访问而共享数据,因为在这种情况下,我们不需要同步线程来访问共享数据,因此我们不需要支付运行时成本并处理死锁和活锁等问题。

现在,让我们考虑一个简单的示例:四个线程并行运行,生成随机数并计算它们的总和。

每个线程独立工作,生成随机数并计算存储在自己编写的变量中的总和。这是理想的(尽管对于这个例子来说,有点做作)应用程序,线程独立工作,没有任何共享数据。

以下代码是本节将要分析的示例的完整源代码。你可以在阅读说明时参考它:

\begin{cpp}
#include <chrono>
#include <iostream>
#include <random>
#include <thread>
#include <vector>

struct result_data {
    unsigned long result { 0 };
};

struct alignas(64) aligned_result_data {
    unsigned long result { 0 };
};

void set_affinity(int core) {
    if (core < 0) {
        return;
    }

    cpu_set_t cpuset;
    CPU_ZERO(&cpuset);
    CPU_SET(core, &cpuset);
    if (pthread_setaffinity_np(pthread_self(), sizeof(cpu_set_t),
    &cpuset) != 0) {
        perror("pthread_setaffinity_np");
        exit(EXIT_FAILURE);
    }
}

template <typename T>
auto random_sum(T& data, const std::size_t seed, const unsigned long
iterations, const int core) {
    set_affinity(core);
    std::mt19937 gen(seed);
    std::uniform_int_distribution dist(1, 5);
    for (unsigned long i = 0; i < iterations; ++i) {
        data.result += dist(gen);
    }
}

using namespace std::chrono;
void sum_random_unaligned(int num_threads, uint32_t iterations) {
    auto* data = new(static_cast<std::align_val_t>(64)) result_data[num_threads];

    auto start = high_resolution_clock::now();
    std::vector<std::thread> threads;
    for (std::size_t i = 0; i < num_threads; ++i) {
        set_affinity(i);
        threads.emplace_back(random_sum<result_data>, std::ref(data[i]), i, iterations, i);
    }
    for (auto& thread : threads) {
        thread.join();
    }
    auto end = high_resolution_clock::now();
    auto duration = std::chrono::duration_cast<milliseconds>(end - start);
    std::cout << "Non-aligned data: " << duration.count() << " milliseconds" << std::endl;

    operator delete[] (data, static_cast<std::align_val_t>(64));
}

void sum_random_aligned(int num_threads, uint32_t iterations) {
    auto* aligned_data = new(static_cast<std::align_val_t>(64))
    aligned_result_data[num_threads];
    auto start = high_resolution_clock::now();
    std::vector<std::thread> threads;
    for (std::size_t i = 0; i < num_threads; ++i) {
        set_affinity(i);
        threads.emplace_back(random_sum<aligned_result_data>, std::ref(aligned_data[i]), i, iterations, i);
    }
    for (auto& thread : threads) {
        thread.join();
    }
    auto end = high_resolution_clock::now();
    auto duration = std::chrono::duration_cast<milliseconds>(end - start);
    std::cout << "Aligned data: " << duration.count() << " milliseconds" << std::endl;
    operator delete[] (aligned_data, static_cast<std::align_val_t>(64));
}

int main() {
    constexpr unsigned long iterations{ 100000000 };
    constexpr unsigned int num_threads = 8;

    sum_random_unaligned(8, iterations);
    sum_random_aligned(8, iterations);

    return 0;
}
\end{cpp}

如果你编译并运行上面的代码,你将得到类似以下内容的输出:

\begin{shell}
Non-aligned data: 4403 milliseconds
Aligned data: 160 milliseconds
\end{shell}

该程序仅调用两个函数: sum\_random\_unaligned 和 sum\_random\_aligned。这两个函数的作用相同:它们创建八个线程,每个线程生成随机数并计算其总和。线程之间不共享任何数据。您可以看到这些函数几乎相同,主要区别在于 sum\_random\_unaligned 使用以下数据结构来存储生成的随机数的总和:

\begin{cpp}
struct result_data {
    unsigned long result { 0 };
};
\end{cpp}

sum\_random\_aligned函数使用了一个稍微不同的函数:

\begin{cpp}
struct alignas(64) aligned_result_data {
    unsigned long result { 0 };
};
\end{cpp}

唯一的区别是使用 alignas(64) 通知编译器数据结构实例必须对齐在 64 字节边界。

我们可以看到,由于线程执行的是相同的任务,因此性能差异非常显著。只需将每个线程写入的变量与 64 字节边界对齐,即可大大提高性能。

要了解为什么会发生这种情况,我们需要考虑现代 CPU 的一个功能——内存缓存。












