
Boost.Asio 是一个跨平台的 C++ 库,由 Chris Kohlhoff 创建,提供可移植的网络和底层 I/O 编程,包括套接字、计时器、主机名解析、套接字 iostream、串行端口、文件描述符和 Windows HANDLE,提供一致的异步模型。还提供协程支持,它们现在在 C++20 中可用,因此在本章中仅简要进行介绍。

Boost.Asio 允许程序管理长时间运行的操作,而无需明确使用线程和锁。此外,由于它在操作系统服务之上实现了一层,因此可以使用最合适的底层操作系统机制来实现可移植性、效率、易用性和可扩展性,例如:分散-聚集 I/O 操作或移动数据,同时最大限度地减少昂贵的复制。

首先,先了解基本的 Boost.Asio 块、 I/O 对象和 I/O 执行上下文对象。

\mySubsubsection{9.2.1.}{I/O对象}

有时,应用程序需要访问 OS 服务,在其上运行异步任务并收集结果或错误。 Boost.Asio 提供了一种由 I/O 对象和 I/O 执行上下文对象组成的机制来实现此功能。

I/O 对象是面向任务的对象,表示执行 I/O 操作的实际实体。如图 9.1 所示, Boost.Asio 提供了核心类来管理并发、流、缓冲区或库中的其他核心功能,还包括用于通过传输控制协议/I nternet 协议 (TCP/IP)、用户数据报协议 (UDP) 或 Internet 控制消息协议 (ICMP) 进行网络通信的可移植网络类,用于定义安全层、传输协议和串行端口等任务类,以及用于处理取决于底层操作系统的特定设置的平台特定类。

\myGraphic{1.0}{content/part4/chapter9/images/1.png}{图 9.1 – I/O 对象}

I/O 对象不直接在 OS 中执行其任务,需要通过 I/O 执行上下文对象与 OS 进行通信。上下文对象的实例作为 I/O 对象构造函数中的第一个参数传递。这里,定义一个 I/O 对象(一个到期时间为三秒的计时器)并通过其构造函数传递一个 I/O 执行上下文对象 (io\_context):

\begin{cpp}
#include <boost/asio.hpp>
#include <chrono>

using namespace std::chrono_literals;

boost::asio::io_context io_context;
boost::asio::steady_timer timer(io_context, 3s);
\end{cpp}

大多数 I/O 对象都有名称以 async\_ 开头的方法。这些方法触发异步操作,当操作完成时,将调用完成处理程序(即作为参数传递给方法的可调用对象)。这些方法立即返回,不会阻塞程序流。当前线程可以在任务未完成时继续执行其他任务。完成后,将调用并执行完成处理程序,处理异步任务的结果或错误。

I/O 对象还提供了阻塞对应方法,这些方法将阻塞直至完成。这些方法不需要接收处理程序作为参数。

如前所述,I/O 对象不直接与操作系统交互,需要一个 I/O 执行上下文对象。

让我们了解一下这类对象。

\mySubsubsection{9.2.2.}{I/O执行上下文对象}

为了访问 I/O 服务,程序至少使用一个 I/O 执行上下文对象,该对象代表 OS I/O 服务的网关。使用 boost::asio::io\_context 类实现,为 I/O 对象提供 OS 服务的核心 I/O 功能。Windows 中, boost::asio::io\_context 基于 I/O 完成端口 (IOCP);在 Linux 上,基于 epoll;在FreeBSD/macOS 上,基于 kqueue。

\myGraphic{0.3}{content/part4/chapter9/images/2.png}{图 9.2 – {} Boost.Asio 架构}

boost::asio::io\_context 是 boost::asio::execution\_context 的子类,是函数对象执行的基类,也被其他执行上下文对象继承,例如: boost::asio::thread\_pool 或 boost::asio::system\_context。

本章中,将使用 boost::asio::io\_context 作为执行上下文对象。自 1.66.0 版本以来, boost::asio::io\_context 类已替代了 boost::asio::io\_service类,并采用了 C++ 中更多现代特性和实践。 boost::asio::io\_service 仍可向后兼容。

如前所述, Boost.Asio 对象可以使用以 async\_ 开头的方法调度异步操作。当所有异步任务都调度完毕后,程序需要调用 boost::asio::io\_context::run() 函数来执行事件处理循环,让操作系统处理任务并将结果传递给程序并触发处理程序。

回到之前的示例,现在将设置完成处理程序 on\_timeout(),这是一个可调用对象( 在本例中是一个函数),在调用异步 async\_wait() 函数时将其作为参数传递。以下是代码示例:

\begin{cpp}
#include <boost/asio.hpp>
#include <iostream>

void on_timeout(const boost::system::error_code& ec) {
    if (!ec) {
        std::cout << "Timer expired.\n" << std::endl;
    } else {
        std::cerr << "Error: " << ec.message() << '\n';
    }
}

int main() {
    boost::asio::io_context io_context;
    boost::asio::steady_timer timer(io_context,
                                std::chrono::seconds(3));
    timer.async_wait(&on_timeout);
    io_context.run();
    return 0;
}
\end{cpp}

运行此代码,三秒后应该在控制台中看到消息“Timer expired.”,或者如果异步调用由于任何原因失败,则会看到一条错误消息。

boost::io\_context::run() 是一个阻塞调用。这旨在保持事件循环运行,允许异步操作运行,并防止程序退出。显然,可以在新线程中调用此函数,并使主线程保持畅通,以继续执行其他任务,正如在前面的章节中看到的那样。

当没有待处理的异步操作时, boost::io\_context::run() 将返回。有一个模板类 boost::asio::executor\_work\_guard,可以保持 io\_context 忙碌,并在需要时避免其退出。让我们通过一个例子看看它是如何工作的。

首先定义一个后台任务,该任务将等待两秒钟,然后使用 boost::asio::io\_context::post( ) 函数通过 io\_context 发布一些工作:

\begin{cpp}
#include <boost/asio.hpp>
#include <chrono>
#include <iostream>
#include <thread>

using namespace std::chrono_literals;

void background_task(boost::asio::io_context& io_context) {
    std::this_thread::sleep_for(2s);
    std::cout << "Posting a background task.\n";
    io_context.post([]() {
        std::cout << "Background task completed!\n";
    });
}
\end{cpp}

main() 函数中,创建了 io\_context 对象,并使用该 io\_context 对象构造了一个 work\_guard 对象。

然后,创建两个线程, io\_thread(io\_context 在其中运行)和 worker(background\_task() 在其中运行),还将 io\_context 作为对后台任务的引用传递以发布工作。

有了这个,主线程就会做一些工作(等待五秒钟),然后调用 reset() 函数删除工作保护,让 io\_context 退出其 run() 函数,并在退出之前连接两个线程:

\begin{cpp}
int main() {
    boost::asio::io_context io_context;
    auto work_guard = boost::asio::make_work_guard(
                      io_context);

    std::thread io_thread([&io_context]() {
        std::cout << "Running io_context.\n";
        io_context.run();
        std::cout << "io_context stopped.\n";
    });

    std::thread worker(background_task,
                        std::ref(io_context));

    // Main thread doing some work.
    std::this_thread::sleep_for(5s);
    std::cout << "Removing work_guard." << std::endl;
    work_guard.reset();
    worker.join();
    io_thread.join();
    return 0;
}
\end{cpp}

如果运行上面的代码,则会输出:

\begin{shell}
Running io_context.
Posting a background task.
Background task completed!
Removing work_guard.
io_context stopped.
\end{shell}

可以看到后台线程如何正确地发布后台任务,并且这在工作保护被移除并且 I/O 上下文对象停止执行之前完成。

保持 io\_context 对象处于活动状态,并处理请求的另一种方法是通过连续调用 async\_ 函数,或从完成处理程序发布工作来提供异步任务。这是读取或写入套接字或流时的常见模式:

\begin{cpp}
#include <boost/asio.hpp>
#include <chrono>
#include <functional>
#include <iostream>

using namespace std::chrono_literals;

int main() {
    boost::asio::io_context io_context;
    boost::asio::steady_timer timer(io_context, 3s);

    std::function<void(const boost::system::error_code&)>
                    timer_handler;

    timer_handler = [&timer, &timer_handler](
                    const boost::system::error_code& ec) {
        if (!ec) {
            std::cout << "Handler: Timer expired.\n";
            timer.expires_after(1s);
            timer.async_wait(timer_handler);
        } else {
            std::cerr << "Handler error: "
                      << ec.message() << std::endl;
            }
        };
    timer.async_wait(timer_handler);
    io_context.run();
    return 0;
}
\end{cpp}

timer\_handler 是完成处理程序,定义为一个 lambda 函数,用于捕获计时器和其自身。每秒,当计时器到期时,它会输出 "Handler: Timer expired." 消息,并通过将新的异步任务(async\_wait() 函数)通过计时器对象加入 io\_context 对象来重新启动自身。

io\_context对象可以从线程运行。默认情况下,此对象线程安全,但在某些情况下,我们想要更好的性能,可能希望避免这种安全性。

\mySamllsubsection{并发提示}

io\_context 构造函数接受并发提示作为参数,向实现建议应用于运行完成处理程序的活动线程数。

默认情况下,此值为 BOOST\_ASIO\_CONCURRENCY\_HINT\_SAFE(值 1),表示 io\_conte xt 对象将从单个线程运行,因此可以实现多项优化。这并不意味着 io\_context 只能从一个线程使用;它仍然提供线程安全性,并且可以从多个线程使用 I/O 对象。

可以指定的其他值如下:

\begin{itemize}
\item
BOOST\_ASIO\_CONCURRENCY\_HINT\_UNSAFE:禁用锁定,因此对 io\_context 或 I/O 对象的所有操作都必须在同一个线程中发生。

\item
BOOST\_ASIO\_CONCURRENCY\_HINT\_UNSAFE\_IO:禁用反应器中的锁定,但将其保留在调度程序中,因此 io\_context 对象中的所有操作都可以使用除 run() 函数和与执行事件处理循环相关的其他方法之外的不同线程。在解释库背后的设计原理时,将了解调度程序和反应器。
\end{itemize}

现在,我们了解什么是事件处理循环,以及如何管理它。

\mySubsubsection{9.2.3.}{事件处理循环}

使用 boost::asio::io\_context::run() 方法, io\_context 会阻塞并继续处理 I/O 异步任务,直到所有任务完成并通知完成处理程序。此 I/O 请求处理在内部事件处理循环中完成。

还有其他方法可以控制事件循环并避免阻塞,直到所有异步事件都处理完毕。这些方法如下:

\begin{itemize}
\item
poll:运行事件处理循环来执行已就绪的处理程序

\item
poll\_one:运行事件处理循环来执行一个就绪的处理程序

\item
run\_for:运行事件处理循环指定的时间

\item
run\_until:与上一个相同,但只到指定时间

\item
run\_one:运行事件处理循环以执行最多一个处理程序

\item
run\_one\_for:与上一个相同,但只持续指定的时间

\item
run\_one\_until:与上一个相同,但只到指定时间
\end{itemize}

还可以通过调用 boost::asio::io\_context::stop() 方法来停止事件循环,或者通过调用 boost:asio::io\_context::stopped() 检查其状态是否已停止。当事件循环未运行时,已安排的任务将继续执行,其他任务将保持待处理状态。可以通过使用前面提到的方法之一,再次启动事件循环来恢复待处理的任务并收集待处理的结果。

前面的示例中,应用程序通过调用异步方法或使用 post() 函数将一些工作发送给 io\_context。现在我们了解 dispatch() 与 post() 的区别。

给 io\_context 分配一些工作除了通过不同 I/O 对象的异步方法或使用 executor\_work\_guard(如下所述)将工作发送到 io\_context 之外,还可以使用 boost::asio::post() 和 boost::asio::dispatch() 模板方法。这两个函数都用于将一些工作安排到 io\_context 对象中。

post() 函数保证任务一定会执行。它把完成处理程序放入执行队列,最终会执行该任务:

\begin{cpp}
boost::asio::io_context io_context;
io_context.post([] {
    std::cout << "This will always run asynchronously.\n";
});
\end{cpp}

另一方面,如果 io\_context 或 strand(本章后面会详细介绍 strands)与分派任务的线程相同,则 dispatch() 可以立即执行任务,否则将其放置在队列中以进行异步执行:

\begin{cpp}
boost::asio::io_context io_context;
io_context.dispatch([] {
    std::cout << "This might run immediately or be queued.\n";
});
\end{cpp}

使用 dispatch(),可以通过减少上下文切换或排队延迟来优化性能。

即使有其他待处理事件排队,已调度事件也可以直接从当前工作线程执行。已发布的事件必须始终由 I/O 执行上下文管理,等待其他处理程序完成后才可执行。

现在,已经了解了一些基本概念,接下来介绍同步和异步操作的工作原理。











