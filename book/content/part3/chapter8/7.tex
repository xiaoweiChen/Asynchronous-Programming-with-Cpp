前面的部分中,实现了几个基本示例来学习主要的 C++ 协程概念。首先实现了一个非常基本的协程,以了解编译器对我们的要求:返回类型(有时称为包装器类型,包装了promise类型)和promise类型。

即使对于这样一个简单的协程,也必须实现我们在编写示例时解释过的一些函数。但有一个函数尚未解释:

\begin{cpp}
void unhandled_exception() noexcept {}
\end{cpp}

当时假设协程不会抛出异常,但事实是它们会抛出异常。我们可以在 unhandled\_exception() 函数体中添加处理异常的功能。

协程中的异常可能发生在创建返回类型或promiase类型对象时,以及执行协程时(与在正常函数中一样,协程可能会引发异常)。

不同之处在于,如果异常是在协程执行前引发的,则创建协程的代码必须处理该异常,而如果异常是在协程执行时引发的,则会调用 unhandled\_exception()。

第一种情况只是普通的异常处理,没有调用任何特殊函数。可以将协程创建放在 try-catch 块中,并像在代码中通常做的那样处理可能的异常。

另一方面,如果调用 unhandled\_exception()(在 promise 类型内部),必须在该函数内部实现异常处理功能。

有不同的策略来处理此类异常。其中包括:

\begin{itemize}
\item
重新抛出异常,以便可以在promise类型之外(即在我们的代码中)处理。

\item
终止程序(例如,调用 std::terminate)。

\item
函数为空。协程将崩溃,很可能会使程序崩溃。
\end{itemize}

因为实现的协程非常简单,所以将该函数置为空。

最后一节中,介绍了协程的异常处理机制。正确处理异常非常重要。例如,知道协程内部发生异常后,将无法恢复;那么,最好让协程崩溃并从程序的另一部分(通常是从调用者函数)处理异常。






