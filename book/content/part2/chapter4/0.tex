第 2 章中,介绍了线程可以读取和写入它们所属进程共享的内存。操作系统实现了进程内存访问保护,但对于访问同一进程中共享内存的线程,并没有这样的保护。多个线程对同一内存地址的并发内存写入操作,需要同步机制来避免数据争用并确保数据完整性。

本章中,将详细介绍多个线程并发访问共享内存,所产生的问题以及如何解决这些问题。我们将详细介绍以下主题:

\begin{itemize}
\item
条件竞争 – 是什么以及如何发生

\item
互斥作为一种同步机制,如何通过 std::mutex 在 C++ 中实现

\item
通用的锁管理

\item
什么是条件变量,以及如何将它们与互斥锁一起使用

\item
使用 std::mutex 和 \verb|std::condition_variable|实现完全同步队列

\item
C++20 引入的新同步原语 - 信号量、栅栏和门闩
\end{itemize}

这些都是基于锁的同步机制。无锁技术是下一章的主题。
