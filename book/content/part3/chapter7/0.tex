在上一章中,我们学习了promise、 future和打包任务。在介绍打包任务时,我们提到std::a sync提供了一种更简单的方式来实现同样的结果,代码更少,因此更干净、更简洁。

异步函数 (std::async) 是一个异步运行可调用对象的函数模板,我们还可以通过传递一些定义启动策略的标志来选择执行方法。它是处理异步操作的强大工具,但它的自动管理和对执行线程缺乏控制等方面也使其不适合某些需要细粒度控制或取消的任务。

在本章中,我们将讨论以下主要主题:

\begin{itemize}
\item
什么是异步函数以及如何使用它?

\item
有哪些不同的启动政策?

\item
与以前的方法,特别是打包任务有什么不同?

\item
使用 std::async 有哪些优点和缺点?

\item
实际场景和示例
\end{itemize}