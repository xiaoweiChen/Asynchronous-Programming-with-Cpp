在我们深入研究使用 C++ 进行并行编程之前,在前两章中,我们将重点了解构建并行软件的不同方法以及软件如何与机器硬件交互的一些基础知识。

在本章中,我们将介绍并行编程以及在开发高效、响应迅速、可扩展的并发和异步软件时可以使用的不同范式和模型。

在对开发并行软件的不同方法进行分类时,有很多种方法可以对概念和方法进行分组。由于本书重点介绍使用 C++ 构建的软件,因此我们可以将不同的并行编程范例划分为以下几种:并发、异步编程、并行编程、反应式编程、数据流、多线程编程和事件驱动编程。

根据手头的问题,特定范式可能比其他范式更适合解决给定场景。了解不同的范式将有助于我们分析问题并缩小最佳解决方案的范围。

在本章中,我们将讨论以下主要主题:

\begin{itemize}
\item
什么是并行编程?它为什么重要?

\item
有哪些不同的并行编程范式?为什么我们需要了解它们?

\item
您将在本书中学到什么?
\end{itemize}