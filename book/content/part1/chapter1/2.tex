当任务或计算同时完成时,就会发生并行计算,其中任务是软件应用程序中的执行或工作单元。由于实现并行性的方法有很多,因此了解不同的方法将有助于编写高效的并行算法。这些方法可以通过范例和模型进行描述。

但首先,先对不同的并行计算系统进行分类。

\mySubsubsection{1.2.1.}{系统分类和技术}

并行计算系统最早的分类之一是由 Michael J. Flynn 于 1966 年提出的。 弗林(Flynn)根据并行计算架构可以处理的数据流和指令数量进行了以下分类:

\begin{itemize}
\item
单指令单数据 (SISD) 系统:定义顺序执行程序

\item
单指令多数据 (SIMD) 系统:对大型数据集进行操作,例如:GPU 计算的信号处理数据

\item
多指令单数据 (MISD) 系统:很少使用

\item
多指令多数据 (MIMD) 系统:基于多核和多处理器计算机的(最常见)并行架构
\end{itemize}

\myGraphic{0.5}{content/part1/chapter1/images/1.png}{图 1.1:弗林分类法}

本书不仅介绍了如何使用 C++ 构建软件,还介绍了其如何与底层硬件交互。软件层面上,我们可以进行更有趣的划分或分类,并定义技术。这些,将在后续章节中进行介绍。

\mySamllsection{数据并行}

许多不同的数据单元由在不同处理单元(例如:CPU 或 GPU)中,运行的同一程序或指令序列并行处理。

数据并行性通过相同操作,同时处理多少个不相交的数据集来实现。利用并行性,可以将大型数据集划分为更小且独立的数据块。

因为更多的处理单元可以处理更多的数据,所以该技术还具有高度的可扩展性。

在这个子集中,可以包含 SIMD 指令集,例如 SSE、 AVX、 VMX 或 NEON,这些指令集可通过 C++ 中的内部函数访问。此外,还有用于 NVIDIA GPU 的 OpenMP 和 CUDA 等库。在机器学习训练和图像处理中可以找到它的一些使用示例,该技术与弗林定义的 SIMD 分类有关。

这种分类方式也存在一些缺点——数据必须能够轻松划分为独立的块。这种数据划分和后验合并也会带来一些开销,从而降低并行化的优势。

\mySamllsection{任务并行}

每个CPU核心使用进程或线程运行不同的任务,当这些任务同时接收数据、处理数据并通过消息传递发回它们生成的结果时,就可以实现任务并行。

任务并行的优势在于能够设计异构、细粒度的任务,从而更好地利用处理资源,在设计具有潜在更高加速的解决方案时更加灵活。

根据数据创建的任务之间可能存在依赖关系,并且每个任务的性质不同,因此调度和协调比数据并行更复杂,所以任务创建会增加一些开销。

这里可以引入弗林的 MISD 和 MIMD 分类法,可以在 Web 服务器请求处理系统或用户界面事件处理程序中找到一些示例。

\mySamllsection{流并行}

可将计算分为处理数据子集的各个阶段,来同时处理连续的数据元素序列(也称为数据流)。

阶段可以同时运行。一些阶段生成其他阶段的输入,根据阶段依赖关系构建管道。处理阶段可以将结果发送到下一个阶段,而无需等待整个流数据。

流并行技术在处理连续数据时非常有效。还具有高度可扩展性,可以通过添加更多处理单元来处理更多的输入数据。由于流数据在到达时进行处理,所以无需等待整个数据流发送,内存使用量也减少了。

然而,这些系统也存在一些缺点。由于逻辑处理、错误处理和恢复,这些系统的实现更加复杂。由于还需要实时处理数据流,因此硬件也可能是瓶颈之一。

这些系统的一些示例包括监控系统、传感器数据处理,以及音频和视频流。

\mySamllsection{隐式并行}

编译器、运行时或硬件会并行执行指令。这使得编写并行程序变得更容易,但限制了开发者对所用策略的控制,甚至使分析性能或调试变得更加困难。

\hspace*{\fill}

现在,我们对不同的并行系统和技术有了更好的了解,是时候了解在设计并行程序时可以使用的模型了。

\mySubsubsection{1.2.2.}{并行编程模型}

并行编程模型是一种并行计算机架构,用于表达算法和构建程序。模型越通用,其价值就越大,可以用于更广泛的场景。从这个意义上讲,C++ 通过标准模板库 (STL) 中的库实现了并行模型,可用于实现顺序应用程序中程序的并行执行。

这些模型描述了程序生命周期内,不同任务如何交互以从输入数据中获取结果。其主要区别在于,任务如何交互以及如何处理传入数据。

\mySamllsection{阶段并行}

阶段并行(也称为议程或自由同步范式)中,多个作业或任务并行执行独立计算。在某个时刻,程序需要使用栅栏执行同步交互操作来同步不同的进程。栅栏是一种同步机制,可确保一组任务在其执行过程中到达特定点执行完后,才能继续进行下一个步骤。接下来的步骤将执行其他异步操作,依此类推。

\myGraphic{0.9}{content/part1/chapter1/images/2.png}{图1.2:阶段并行}

这种模型的优点是任务间的交互不会与计算重叠,但各个处理单元之间的工作量和吞吐量很难达到均衡。

\mySamllsection{分而治之}

使用此模型的应用程序使用主任务或作业,将工作量分配给其子任务。子任务并行计算结果将其返回给父任务,父任务将结果合并为最终结果。子任务还可以将分配的任务细分为更小的任务,并创建自己的子任务。

该模型具有与相并联模型相同的缺点,很难实现良好的负载平衡。

\myGraphic{0.5}{content/part1/chapter1/images/3.png}{图 1.3:分而治之模型}

图 1.3 中,可以看到主作业如何将工作划分给几个子任务,以及子任务 2 如何将其分配的工作细分为两个任务。

\mySamllsection{管道模型}

多个任务相互连接,构建虚拟管道。此管道中,各个阶段可以同时运行,并在输入数据时重叠执行。

\myGraphic{1.0}{content/part1/chapter1/images/4.png}{图 1.4:管道模型}

上图中三个任务在由五个阶段组成的流水线中交互,每个阶段都有一些任务在运行,并产生输出结果,供下一个阶段的任务使用。

\mySamllsection{主从模型}

使用主从模型(也称为进程农场),主执行者执行算法的顺序部分,并生成和协调在工作负载中执行并行操作的从属任务。当从属任务完成计算时,将结果通知主执行者,然后主执行者可能会将更多数据发送给从属任务进行处理。

\myGraphic{0.5}{content/part1/chapter1/images/5.png}{图 1.5:主从模型}

缺点是,如果主服务器需要处理太多从服务器或任务太小,主服务器可能会成为瓶颈。在选择由主服务器执行的工作量时,需要权衡每个任务,这也称为任务粒度。当任务较小时,称为细粒度,当任务较大时,称为粗粒度。

\mySamllsection{工作池}

工作池模型中,全局结构保存着要执行的工作项池。主程序会创建作业,从池中获取工作项并执行。

这些作业可以生成更多工作,并将其插入到工作池中。当所有工作都完成后,清空工作池,并行程序便会结束执行。

\myGraphic{0.5}{content/part1/chapter1/images/6.png}{图 1.6:工作池模型}

该机制有利于实现空闲处理单元之间的负载平衡。

在 C++ 中,这个池通常使用无序集合、队列或优先级队列来实现。我们将在本书中后续内容中进行实现。

\hspace*{\fill}

了解了可用于构建并行系统的各种模型,可以继续探索用于开发高效并行任务的并行编程范例了。





























