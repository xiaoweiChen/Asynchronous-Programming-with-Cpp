

同步原语是管理多线程编程中共享资源的并发访问的重要工具。有几种同步原语,每种都有各自的特定用途和特征:

\begin{itemize}
\item
互斥锁:互斥锁用于强制对代码的关键部分进行独占访问。线程可以锁定互斥锁,从而阻止其他线程进入受保护的部分,直到互斥锁解锁。互斥锁保证在任何给定时间只有一个线程可以执行关键部分,从而确保数据完整性并防止竞争条件。

\item
信号量:信号量比互斥量更通用,可用于更广泛的同步任务,包括线程之间的信号发送。信号量维护一个整数计数器,线程可以将其递增(信号发送)或递减(等待)。信号量允许更复杂的协调模式,例如计数信号量(用于资源分配)和二进制信号量(类似于互斥量)。

\item
条件变量:条件变量用于根据特定条件进行线程同步。线程可以阻塞(等待)条件变量,直到特定条件变为真。其他线程可以向条件变量发送信号,从而唤醒等待的线程并继续执行。条件变量通常与互斥锁一起使用,以实现更细粒度的同步并避免繁忙等待。

\item
附加同步原语:除了前面讨论的核心同步原语之外,还有其他几种同步机制:

\begin{itemize}
\item
栅栏:屏障允许一组线程同步执行,确保所有线程在继续执行之前都达到某个点

\item
读写锁:读写锁提供了一种控制共享数据并发访问的方法,允许多个读取器但一次只能有一个写入器

\item
自旋锁:自旋锁是一种互斥锁,涉及忙等待,持续检查内存位置,直到可用
\end{itemize}
\end{itemize}

在第 4 章和第 5 章中,我们将深入了解 C++ 标准模板库 (STL) 中实现的同步原语以及如何使用它们的示例。

\mySubsubsection{1.4.1.}{选择正确的同步原语}

选择适当的同步原语取决于应用程序的具体要求以及所访问的共享资源的性质。以下是一些一般准则:

\begin{itemize}
\item
互斥锁:当需要对关键部分进行独占访问以确保数据完整性并防止竞争条件时,请使用互斥锁

\item
信号量:当需要更复杂的协调模式时使用信号量,例如资源分配或线程之间的信号传递

\item
条件变量:当线程需要等待特定条件变为真才能继续执行时,请使用条件变量
\end{itemize}

有效使用同步原语对于开发安全高效的多线程程序至关重要。通过了解不同同步机制的用途和特性,开发人员可以根据其特定需求选择最合适的原语,实现可靠且可预测的并发执行。



























