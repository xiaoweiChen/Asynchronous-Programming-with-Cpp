异步编程涉及启动操作而不等待它们完成再继续执行下一个任务。这种非阻塞行为允许开发高响应性和高效率的应用程序,能够同时处理大量操作,而不会出现不必要的延迟或浪费计算资源等待任务完成。

异步编程非常重要,尤其是在网络应用程序、用户界面和系统编程的开发中。它使开发人员能够创建能够管理大量请求、执行输入/输出 (I/O) 操作或高效执行并发任务的应用程序,从而显著增强用户体验和应用程序性能。

Linux 操作系统(本书中,我们将重点介绍在代码无法独立于平台的情况下在 Linux 操作系统上进行开发)具有强大的进程管理、对线程的本机支持和高级 I/O 功能,是开发高性能异步应用程序的理想环境。这些系统提供了一组丰富的功能,例如用于进程和线程管理的强大 API、非阻塞 I/O 和进程间通信 (IPC) 机制。

本章介绍了 Linux 环境中异步编程所必需的基本概念和组件。

我们将探讨以下主题:

\begin{itemize}
\item
Linux 中的进程

\item
服务和守护进程

\item
线程和并发
\end{itemize}

在本章结束时,您将对 Linux 中的异步编程环境有一个基本的了解,为后续章节的更深入探索和实际应用奠定基础。































