Boost.Cobalt 中,通道为协程提供了一种异步通信方式,允许以安全高效的方式在生产者和消费者协程之间传输数据。其受到 Golang 通道的启发,允许通过消息传递进行通信,从而促进通过通信共享内存的范式。

通道是一种机制,通过该机制,值从一个协程(生产者)异步传递到另一个协程(消费者)。这种通信是非阻塞的,所以协程可以在等待通道上的数据可用时,或将数据写入容量有限的通道时暂停执行。澄清一下:如果阻塞是指协程暂停,则读取和写入操作都可能阻塞,具体取决于缓冲区大小,但另一方面,从线程的角度来看,这些操作不会阻塞线程。

如果缓冲区大小为零,则读取和写入需要同时发生并充当会合点(同步通信)。如果通道大小大于零且缓冲区未满,则写入操作不会暂停协程。同样,如果缓冲区不为空,则读取操作也不会暂停。

与 Golang 通道类似, Boost.Cobalt 通道是强类型的。通道是为特定类型定义的,并且只能通过该类型发送数据。例如, int 类型的通道(boost::cobalt::channel<int>)只能传输整数。

来看一个通道的示例:

\begin{cpp}
#include <iostream>
#include <boost/cobalt.hpp>
#include <boost/asio.hpp>
boost::cobalt::promise<void> producer(boost::cobalt::channel<int>& ch)
{
    for (int i = 1; i <= 10; ++i) {
        std::cout << "Producer waiting for request\n";
        co_await ch.write(i);
        std::cout << "Producing value " << i << std::endl;
    }
    std::cout << "Producer end\n";
    ch.close();
    co_return;
}
boost::cobalt::main co_main(int, char*[]) {
    boost::cobalt::channel<int> ch;
    auto p = producer(ch);
    while (ch.is_open()) {
        std::cout << "Consumer waiting for next number \n";
        std::this_thread::sleep_for(std::chrono::seconds(5));
        auto n = co_await ch.read();
        std::cout << "Consuming value " << n << std::endl;
        std::cout << n * n << std::endl;
    }
    co_await p;
    co_return 0;
}
\end{cpp}

这个例子中,创建一个大小为 0 的通道和两个协程:生产者promise和充当消费者的 co\_main()。生产者将整数写入通道,消费者将其读回并输出它们的平方。

我们添加了 std::this\_thread::sleep 来延迟程序执行,因此能够看到程序运行时发生的情况。来看一下示例输出的摘录,看看它是如何工作的:

\begin{shell}
Producer waiting for request
Consumer waiting for next number
Producing value 1
Producer waiting for request
Consuming value 1
1
Consumer waiting for next number
Producing value 2
Producer waiting for request
Consuming value 2
4
Consumer waiting for next number
Producing value 3
Producer waiting for request
Consuming value 3
9
Consumer waiting for next number
\end{shell}

消费者和生产者都等待下一个操作发生,生产者将始终等待消费者请求下一个项目。这基本上就是生成器的工作方式,也是使用协程的异步代码中的常见模式。

消费者执行以下代码行:

\begin{cpp}
auto n = co_await ch.read();
\end{cpp}

然后,生产者将下一个数字写入通道并等待下一个请求。这在以下代码行中完成:

\begin{cpp}
co_await ch.write(i);
\end{cpp}

可以在前面的输出摘录的第四行中看到生产者如何返回等待下一个请求。Boost.Cobalt 通道使编写这种异步代码非常干净且易于理解。该示例显示两个协程通过通道进行通信。

下一节将介绍同步函数 - 等待多个协程的机制。








