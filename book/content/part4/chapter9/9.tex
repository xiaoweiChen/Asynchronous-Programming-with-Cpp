
某些 I/O 对象(例如套接字或计时器)通过调用 close() 或 cancel() 方法在对象范围内取消未完成的异步操作。如果异步操作被取消,完成处理程序将收到带有 boost::asio::error::operation\_aborted 代码的错误。

以下示例中,创建了一个计时器,并将其超时时间设置为 5 秒。但在主线程仅休眠两秒后,通过调用其 cancel() 方法取消了计时器,从而使用 boost::asio::error::operation\_aborted 错误代码调用完成处理程序:

\begin{cpp}
#include <boost/asio.hpp>
#include <chrono>
#include <iostream>
#include <thread>

using namespace std::chrono_literals;

void handle_timeout(const boost::system::error_code& ec) {
    if (ec == boost::asio::error::operation_aborted) {
        std::cout << "Timer canceled.\n";
    } else if (!ec) {
        std::cout << "Timer expired.\n";
    } else {
        std::cout << "Error: " << ec.message()
                  << std::endl;
    }
}
int main() {
    boost::asio::io_context io_context;
    boost::asio::steady_timer timer(io_context, 5s);

    timer.async_wait(handle_timeout);

    std::this_thread::sleep_for(2s);
    timer.cancel();

    io_context.run();
    return 0;
}
\end{cpp}

但若想要每个操作都取消,需要设置一个取消槽,当发出取消信号时,该槽将触发。这个取消信号/槽对组成了一个轻量级通道来传达取消操作。取消框架自 1.75 版起在 Boost.Asio 中可用。

这种方法实现了更灵活的取消机制,可以使用同一信号取消多个操作,并与 Boost.Asio 的异步操作无缝集成。同步操作只能通过使用前面描述的 cancel() 或 close() 方法取消;取消槽机制不支持。

修改前面的示例,并使用取消信号/槽来取消计时器。只需要修改在 main() 函数中取消计时器的方式,当执行异步 async\_wait() 操作时,通过使用 boost::asio::bind\_cancellation\_slot() 函数将取消信号和完成处理程序中的槽绑定在一起,即可创建取消槽。

与之前一样,定时器的有效期为五秒。同样,主线程仅休眠两秒。这次,通过调用 cancel\_signal.emit() 函数发出取消信号。该信号将触发对应的取消槽并使用 boost::asio::error::operation\_aborted 错误代码执行完成处理程序,在控制台中输出"Timer expired."的消息;参见以下内容:

\begin{cpp}
int main() {
    boost::asio::io_context io_context;
    boost::asio::steady_timer timer(io_context, 5s);

    boost::asio::cancellation_signal cancel_signal;

    timer.async_wait(boost::asio::bind_cancellation_slot(
        cancel_signal.slot(),
        handle_timeout
    ));

    std::this_thread::sleep_for(2s);
    cancel_signal.emit(
        boost::asio::cancellation_type::all);

    io_context.run();
    return 0;
}
\end{cpp}

发出信号时,必须指定取消类型,让目标操作知道应用程序需要什么,以及操作保证什么,从而控制取消的范围和行为。

各种取消类别如下:

\begin{itemize}
\item
None:不执行取消。如果想测试是否应该发生取消,这会很有用。

\item
Terminal:该操作具有未指定的副作用,因此取消操作的唯一安全方法是关闭或销毁 I/O 对象,其结果是最终的,例如完成任务或交易。

\item
Partial:操作具有明确定义的副作用,完成处理程序可以采取所需的操作来解决问题,所以操作已部分完成并可以恢复或重试。

\item
Total或All:操作没有副作用。取消终端操作和部分操作,可通过停止所有正在进行的异步操作实现全面取消。
\end{itemize}

如果异步操作不支持取消类型,则取消请求将被丢弃。例如,计时器操作支持所有取消类别,但套接字仅支持 Total 和 All,所以如果尝试使用 Partial 取消来取消套接字异步操作,则将忽略此取消。如果 I/O 系统尝试处理不支持的取消请求,这可以防止出现未定义的行为。

此外,操作发起之后,但在操作开始之前或操作完成后,发出的取消请求均无效。

有时,需要按顺序运行一些工作。接下来我们将介绍如何使用 strand 来进行实现。









































