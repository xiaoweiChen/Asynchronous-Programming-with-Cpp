在本章中,我们探讨了操作系统中的进程概念。进程是执行程序和管理计算机上资源的基本实体。我们深入研究了进程生命周期,研究了进程从创建到终止所经历的各个阶段。此外,我们还讨论了 IPC,它对于进程之间的交互和信息交换至关重要。

此外,我们在 Linux 操作系统中介绍了守护进程。守护进程是一种特殊类型的进程,它们作为服务在后台运行,并执行特定任务,例如管理系统资源、处理网络连接或为系统提供其他基本服务。我们还探讨了系统线程和用户线程的概念,它们是与父进程共享相同地址空间的轻量级进程。我们讨论了多线程应用程序的优势,包括改进的性能和响应能力,以及在单个进程内管理和同步多个线程所面临的挑战。

了解多线程产生的不同问题是了解如何解决这些问题的基础。在下一章中,我们将了解如何创建线程,然后在第 4 章和第 5 章中,我们将深入研究标准 C++ 提供的不同同步原语及其不同的应用。