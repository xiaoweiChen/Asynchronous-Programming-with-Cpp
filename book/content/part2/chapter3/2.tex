
C++ 中创建和管理线程的主要库是线程库,让我们回顾一下线程,然后再深入了解线程库提供的功能。

\mySubsubsection{3.2.1.}{什么是线程?回顾一下}

线程的目的是在一个进程中同时执行多个任务。

线程有自己的堆栈、本地数据和 CPU 寄存器,例如指令指针(IP)和堆栈指针(SP),但共享其父进程的地址空间和虚拟内存。

用户空间中,可以区分本机线程和轻量级或虚拟线程。本机线程是操作系统在使用某些内核 API 时创建的线程。 C++ 线程对象创建和管理这些类型的线程。另一方面,轻量级线程类似于本机线程,只由运行时或库模拟。在 C++ 中,协程属于这一组。轻量级线程比本机线程具有更快的上下文切换速度。此外,多个轻量级线程可以在同一个本机线程中运行,并且比本机线程小得多。

本章将介绍原生线程。第 8 章将介绍以协程形式出现的轻量级线程。

\mySubsubsection{3.2.2.}{C++ 线程库}

在 C++ 中,线程允许多个函数同时运行。线程类定义了一个类型安全的本机线程接口。在标准模板库 (STL) 中的 <thread> 头文件中的 std::thread 库中定义,自 C++11 起可用。

C++ STL 中包含线程库之前,开发人员使用特定于平台的库,例如: Unix 或 Linux 操作系统中的 POSIX 线程 (pthread) 库、 Windows NT 和 CE 系统的 C 运行时 (CRT) 和 Win32 库或第三方库(例如:Boost.Threads)。本书中,将仅使用现代 C++ 功能。由于 <thread> 可用并在特定于平台的机制之上提供了可移植的抽象,因此不会使用或解释这些库中的任何一个。第 9 章中,将介绍 Boost.Asio。在第 10 章中,将介绍 Boost.Cobalt。这两个库都提供了处理异步 I/O 操作和协程的高级框架。

现在是时候了解不同的线程操作了。
