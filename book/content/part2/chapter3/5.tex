

实现一个接受间隔和回调函数的计时器。计时器将在每个间隔执行回调函数,用户可以通过调用其 stop() 函数来停止计时器。

以下代码片段展示了计时器的实现

\begin{cpp}
#include <chrono>
#include <functional>
#include <iostream>
#include <syncstream>
#include <thread>

#define sync_cout std::osyncstream(std::cout)

using namespace std::chrono_literals;
using namespace std::chrono;

template<typename Duration>
class Timer {
    public:
    typedef std::function<void(void)> Callback;
    Timer(const Duration interval,
          const Callback& callback) {
        auto value = duration_cast<milliseconds>(interval);
        sync_cout << "Timer: Starting with interval of "
                  << value << std::endl;
        t = std::jthread([&](std::stop_token stop_token) {
            while (!stop_token.stop_requested()) {
                sync_cout << "Timer: Running callback "
                          << val.load() << std::endl;
                val++;
                callback();
                sync_cout << "Timer: Sleeping...\n";
                std::this_thread::sleep_for(interval);
            }
            sync_cout << "Timer: Exit\n";
        });
    }
    void stop() {
        t.request_stop();
    }
    private:
    std::jthread t;
    std::atomic_int32_t val{0};
};
\end{cpp}

Timer 构造函数接受一个回调函数(一个 std::function<void(void)> 对象)和一个 std::chrono: :duration 对象,定义执行回调的时间段或间隔。

使用 lambda 表达式创建一个 std::jthread 对象,循环以时间间隔调用回调。此循环检查是否已通过 \verb|stop_token| 请求停止,该请求可通过 stop() Timer API 函数启用。如果是,则循环退出,线程终止。

使用方法如下:

\begin{cpp}
int main(void) {
    sync_cout << "Main: Create timer\n";
    Timer timer(1s, [&]() {
        sync_cout << "Callback: Running...\n";
    });

    std::this_thread::sleep_for(3s);
    sync_cout << "Main thread: Stop timer\n";
    timer.stop();

    std::this_thread::sleep_for(500ms);
    sync_cout << "Main thread: Exit\n";
    return 0;
}
\end{cpp}

此示例中,启动了计时器,该计时器每秒将打印一次 Callback: Running 消息。三秒后,主线程将调用 timer.stop() 函数,终止计时器线程,主线程等待 500 毫秒后退出。

这是输出:

\begin{shell}
Main: Create timer
Timer: Starting with interval of 1000ms
Timer: Running callback 0
Callback: Running...
Timer: Sleeping...
Timer: Running callback 1
Callback: Running...
Timer: Sleeping...
Timer: Running callback 2
Callback: Running...
Timer: Sleeping...
Main thread: Stop timer
Timer: Exit
Main thread: Exit
\end{shell}

作为练习,可以稍微修改此示例以实现超时类,如果在给定的超时间隔内没有输入事件,则该类会调用回调函数。这是处理网络通信时的一种常见模式,如果一段时间内没有收到数据包,则会发送数据包重放请求。























