正如本章中已经看到的, Boost.Cobalt promise是返回一个值的立即协程,而 Boost.Cobalt 任务是惰性版本的promise。

我们可以将它们视为不会像生成器那样产生多个值的函数,可以反复调用 Promise 来获取多个值,但调用之间不会保留状态(就像生成器一样)。基本上, Promise 是一个可以使用 co\_await 的协程(也可以使用 co\_return)。

Promise 的不同用例包括套接字侦听器接收网络数据包、处理数据包、查询数据库,然后从数据中生成一些结果。一般来说,其功能需要异步等待某个结果,然后对该结果执行某些处理(或者可能只是将其返回给调用者) 。

第一个例子是一个简单的promise,可生成一个随机数(也可以用生成器来完成):

\begin{cpp}
#include <iostream>
#include <random>

#include <boost/cobalt.hpp>

boost::cobalt::promise<int> random_number(int min, int max) {
    std::random_device rd;
    std::mt19937 gen(rd());

    std::uniform_int_distribution<> dist(min, max);
    co_return dist(gen);
}

boost::cobalt::promise<int> random(int min, int max) {
    int res = co_await random_number(min, max);
    co_return res;
}

boost::cobalt::main co_main(int, char*[]) {
    for (int i = 0; i < 10; ++i) {
        auto r = random(1, 100);
        std::cout << "random number between 1 and 100: "
                  << co_await r << std::endl;
    }
    co_return 0;
}
\end{cpp}

上面的代码中,编写了三个协程:

\begin{itemize}
\item
co\_main:Boost.Cobalt 中, co\_main 是一个协同程序,调用 co\_return 来返回一个值。

\item
random():此协程向调用者返回一个随机数。使用 co\_await 调用 random() 来生成随机数,异步等待随机数生成。

\item
random\_number():该协程在两个值(最小值和最大值)之间生成一个均匀分布的随机数,并将其返回给调用者。 random\_number() 也是一个promise。
\end{itemize}

以下协程返回一个随机数的 std::vector<int>。循环调用 co\_await random\_number() 来生成一个包含 n 个随机数的向量:

\begin{cpp}
boost::cobalt::promise<std::vector<int>> random_vector(int min, int
max, int n) {
    std::vector<int> rv(n);
    for (int i = 0; i < n; ++i) {
        rv[i] = co_await random_number(min, max);
    }
    co_return rv;
}
\end{cpp}

上述函数返回 std::vector<int> 的promise。要访问该向量,需要调用 get():

\begin{cpp}
auto v = random_vector(1, 100, 20);
for (int n : v.get()) {
    std::cout << n << " ";
}
std::cout << std::endl;
\end{cpp}

上述代码打印了 v 向量的元素。要访问该向量,需要调用 v.get()。

实现第二个示例来说明promise和任务的执行有何不同:

\begin{cpp}
#include <chrono>
#include <iostream>
#include <thread>

#include <boost/cobalt.hpp>

void sleep(){
    std::this_thread::sleep_for(std::chrono::seconds(2));
}

boost::cobalt::promise<int> eager_promise(){
    std::cout << "Eager promise started\n";
    sleep();
    std::cout << "Eager promise done\n";
    co_return 1;
}

boost::cobalt::task<int> lazy_task(){
    std::cout << "Lazy task started\n";
    sleep();
    std::cout << "Lazy task done\n";
    co_return 2;
}

boost::cobalt::main co_main(int, char*[]){
    std::cout << "Calling eager_promise...\n";
    auto promise_result = eager_promise();
    std::cout << "Promise called, but not yet awaited.\n";

    std::cout << "Calling lazy_task...\n";
    auto task_result = lazy_task();
    std::cout << "Task called, but not yet awaited.\n";

    std::cout << "Awaiting both results...\n";
    int promise_value = co_await promise_result;
    std::cout << "Promise value: " << promise_value
              << std::endl;

    int task_value = co_await task_result;
    std::cout << "Task value: " << task_value
              << std::endl;

    co_return 0;
}
\end{cpp}

这个例子中,实现了两个协程:一个 Promise 和一个 Task。Promise 是 Eager 类型的,它在调用时立即开始执行。而 Task 是 Lazy 类型的,在调用后就会暂停。

运行程序时,会输出所有消息,可确切地知道协程是如何执行的。

co\_main()前三行执行完成后,输出如下:

\begin{shell}
Calling eager_promise...
Eager promise started
Eager promise done
Promise called, but not yet awaited.
\end{shell}

从这些消息中,知道promise已经执行了,直到调用co\_return。

执行 co\_main() 的接下来三行之后,输出有以下新消息:

\begin{shell}
Calling lazy_task...
Task called, but not yet awaited.
\end{shell}

这里,我们看到任务尚未执行。这是一个惰性协程,在调用后会立即挂起,并且此协程尚未输出任何消息。

再执行三行 co\_main(),程序输出的新消息如下:

\begin{shell}
Awaiting both results...
Promise value: 1
\end{shell}

对promise的 co\_await 调用给出了其结果(在本例中设置为 1)并且其执行结束。

最后,在任务上调用 co\_await,然后它执行并返回其值(在本例中设置为 2)。输出如下:

\begin{shell}
Lazy task started
Lazy task done
Task value: 2
\end{shell}

此示例显示了任务如何惰性化并开始暂停,并且仅当调用者对其调用 co\_await 时才恢复执行。

本节中,与生成器的情况一样,使用 Boost.Cobalt 编写promise和任务协程比仅使用纯 C++ 要容易得多,不需要编写 C++ 实现协程所需的所有支持代码。还了解了任务和promise的区别。

下一节中,将研究通道的示例,它是生产者/消费者模型中两个协程之间的通信机制。

