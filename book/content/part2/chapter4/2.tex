当程序运行的结果取决于其指令的执行顺序时,就会发生竞争条件。我们将从一个非常简单的示例开始,展示竞争条件是如何发生的,然后在本章后面,我们将学习如何解决这个问题。

在下面的代码中,计数器全局变量由两个同时运行的线程增加:

\begin{cpp}
#include <iostream>
#include <thread>

int counter = 0;

int main() {
    auto func = [] {
        for (int i = 0; i < 1000000; ++i) {
            counter++;
        }
    };

    std::thread t1(func);
    std::thread t2(func);

    t1.join();
    t2.join();

    std::cout << counter << std::endl;
    return 0;
}
\end{cpp}

运行上述代码三次后,我们得到以下计数器值:

\begin{shell}
1056205
1217311
1167474
\end{shell}

我们在这里看到两个主要问题:首先,计数器的值不正确;其次,程序每次执行都会以不同的计数器值结束。结果是不确定的,而且大多数情况下都是不正确的。如果你非常幸运,你可能会得到正确的值,但这种情况不太可能发生。

这个场景涉及两个线程 t1 和 t2,它们同时运行并修改同一个变量,该变量本质上是某个内存区域。它似乎应该可以正常工作,因为只有一行代码会增加计数器值,从而修改内存内容(顺便说一句,我们使用像 counter++ 中的后增量运算符或像 ++counter 中的前增量运算符都没有关系;结果同样是错误的)。

仔细查看上面的代码,让我们仔细研究以下行:

\begin{cpp}
counter++;
\end{cpp}

它分三步增加计数器:

\begin{itemize}
\item
计数器变量存储的内存地址的内容被加载到 CPU 寄存器中。在本例中, int 数据类型从内存加载到 CPU 寄存器中。

\item
寄存器中的值增加一。

\item
寄存器中的值保存在计数器变量的内存地址中。
\end{itemize}

现在,让我们考虑两个线程尝试同时增加计数器的可能情况。让我们看看表 4.1:

% Please add the following required packages to your document preamble:
% \usepackage{longtable}
% Note: It may be necessary to compile the document several times to get a multi-page table to line up properly
\begin{longtable}{|l|l|}
\hline
线程 1                          & 线程 2                          \\ \hline
\endfirsthead
%
\endhead
%
{[}1{]} 将计数器值装入寄存器 & {[}3{]} 将计数器值装入寄存器 \\ \hline
{[}2{]} 增加寄存器的值  & {[}5{]} 增加寄存器的值  \\ \hline
{[}4{]} 寄存器计数器 & {[}6{]} 寄存器计数器 \\ \hline
\end{longtable}

\begin{center}
表 4.1:两个线程同时增加计数器
\end{center}

线程 1 执行 [1] 并将计数器的当前值(假设为 1)加载到 CPU 寄存器中。然后,它将寄存器中的值加一 [2](现在,寄存器值为 2)。

线程 2 被安排执行,并且 [3] 将计数器的当前值(记住 - 它还没有被修改,所以它仍然为 1 )加载到 CPU 寄存器中。

现在,线程 1 再次被调度执行,并且 [4] 将更新的值存储到内存中。计数器的值现在等于 2 。
最后,线程 2 再次被调度,并执行 [5] 和 [6]。寄存器值增加一,然后将值二存储在内存中。计数器变量只增加了一次,而它应该增加两次,其值应该是三。

发生上述问题是因为计数器上的递增操作不是原子的。如果每个线程都可以在不被中断的情况下执行递增计数器变量所需的三条指令,则计数器将按预期递增两次。但是,根据操作执行的顺序,结果可能会有所不同。这称为竞争条件。

为了避免竞争条件,我们需要确保以受控的方式访问和修改共享资源。实现此目的的一种方法是使用锁。锁是一种同步原语,它一次只允许一个线程访问共享资源。当线程想要访问共享资源时,它必须首先获取锁。一旦线程获取了锁,它就可以访问共享资源而不会受到其他线程的干扰。当线程访问完共享资源后,它必须释放锁,以便其他线程可以访问它。

避免竞争条件的另一种方法是使用原子操作。原子操作是保证在单个不可分割的步骤中执行的操作。这意味着在执行原子操作时,没有其他线程可以干扰该操作。原子操作通常使用设计为不可分割的硬件指令来实现。原子操作将在第 5 章中解释。

在本节中,我们了解了多线程代码产生的最常见和最重要的问题:竞争条件。我们已经看到,根据执行操作的顺序,结果可能会有所不同。考虑到这个问题,我们将在下一节中研究如何解决它。





























