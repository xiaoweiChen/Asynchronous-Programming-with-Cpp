

生成器是一个协同程序,通过从暂停点反复恢复自身来生成一系列元素。

生成器可以看作是一个无限序列,可以生成任意数量的元素。调用函数可以根据需要从生成器中,获取任意数量的新元素。

当我们说无限时,指的是理论上。生成器协同程序将产生没有明确最后一个元素的元素(可以实现具有有限范围的生成器)。但在实践中,必须处理诸如数字序列溢出之类的问题。

让我们从头开始实现一个生成器,应用在本章前面章节中获得的知识。

\mySubsubsection{8.5.1.}{斐波那契数列生成器}

假设正在实现一个应用程序,需要使用斐波那契数列。斐波那契数列是一个序列,其中每个数字都是前两个数字的总和。第一个元素是 0,第二个元素是1,然后应用定义并生成一个又一个元素。

斐波那契数列: $F (n) = F (n − 2) + F (n − 1) ; F (0) = 0, F (1) = 1$

可以用 for 循环生成这些数字,但需要在程序的不同点生成相应的值,所以需要实现一种方法来存储序列的状态,还需要在程序的某个地方保存我们生成的最后一个元素。

协程是解决此问题的一个非常好的方法,将自行保持所需的状态并且将暂停直到请求序列中的下一个数字。

以下是使用生成器协程的代码:

\begin{cpp}
int main() {
    sequence_generator<int64_t> fib = fibonacci();

    std::cout << "Generate ten Fibonacci numbers\n"s;

    for (int i = 0; i < 10; ++i) {
        fib.next();
        std::cout << fib.value() << " ";
    }
    std::cout << std::endl;

    std::cout << "Generate ten more\n"s;

    for (int i = 0; i < 10; ++i) {
        fib.next();
        std::cout << fib.value() << " ";
    }
    std::cout << std::endl;

    std::cout << "Let's do five more\n"s;

    for (int i = 0; i < 5; ++i) {
        fib.next();
        std::cout << fib.value() << " ";
    }
    std::cout << std::endl;

    return 0;
}
\end{cpp}

正如上面的代码所示,生成了所需的数字,而不必担心最后一个元素是什么。序列由协程生成。请注意,尽管理论上该序列是无限的,但程序必须意识到非常大的斐波那契数的潜在溢出。为了实现生成器协程,遵循本章前面解释的原则。

首先实现协程函数:

\begin{cpp}
sequence_generator<int64_t> fibonacci() {
    int64_t a{ 0 };
    int64_t b{ 1 };
    int64_t c{ 0 };

    while (true) {
        co_yield a;
        c = a + b;
        a = b;
        b = c;
    }
}
\end{cpp}

协程只是通过应用公式来生成斐波那契数列中的下一个元素。元素是在无限循环中生成的,但协程会在 co\_yield 之后自行暂停。

返回类型是sequence\_generator结构(我们使用模板来使用32位或64位整数),包含一个承诺类型,非常类似于我们在上一节中看到的协程让步中的promise类型。

在sequence\_generator结构中,添加了两个在实现序列生成器时很有用的函数。

\begin{cpp}
void next() {
    if (!handle.done()) {
        handle.resume();
    }
}
\end{cpp}

next()函数为将要生成的序列中的新斐波那契数恢复协程。

\begin{cpp}
int64_t value() {
    return handle.promise().output_data;
}
\end{cpp}

value() 函数返回最后生成的斐波那契数。

这样,将元素生成与其检索 Qvalue 分离。

请在本书附带的 GitHub 库中查找本示例的完整代码。

\mySamllsubsection{C++23的std::generator}

C++ 中实现即使是最基本的协程也需要一定数量的代码。这种情况可能会在 C++26 中发生变化, C++ 标准库将对协程提供更多支持,这将使我们能够更轻松地编写协程。

C++23 引入了 std::generator 模板类,可以编写基于协程的生成器,而无需编写任何必需的代码,例如:promise类型、返回类型及其所有函数。要运行此示例,需要一个 C++23 编译器,这里使用了 GCC 14.1。 std::generator 在 Clang 中不可用。

来看看使用新的 C++23 标准库功能的斐波那契数列生成器:

\begin{cpp}
#include <generator>
#include <iostream>

std::generator<int> fibonacci_generator() {
    int a{ };
    int b{ 1 };
    while (true) {
        co_yield a;
        int c = a + b;
        a = b;
        b = c;
    }
}
auto fib = fibonacci_generator();

int main() {
    int i = 0;
    for (auto f = fib.begin(); f != fib.end(); ++f) {
        if (i == 10) {
            break;
        }
        std::cout << *f << " ";
        ++i;
    }
    std::cout << std::endl;
}
\end{cpp}

第一步是包含 <generator> 头文件。然后,只需编写协程,所有代码都已为编写完成。前面的代码中,使用迭代器(由 C++ 标准库提供)访问生成的元素,能够使用 range-for 循环、算法和范围。

还可以编写斐波那契生成器的一个版本,来生成一定数量的元素而不是无限数列:

\begin{cpp}
std::generator<int> fibonacci_generator(int limit) {
    int a{ };
    int b{ 1 };
    while (limit--) {
        co_yield a;
        int c = a + b;
        a = b;
        b = c;
    }
}
\end{cpp}

代码的变化非常简单:只需传递希望生成器生成的元素数量,并将其用作 while 循环中的终止条件。

本节中,实现了最常见的协程类型之一——生成器。我们从头开始以及使用 C++23的std::generator 类模板实现了生成器。

我们将在下一节中实现一个简单的字符串解析器协程。










