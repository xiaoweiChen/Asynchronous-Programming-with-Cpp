在本章中,我们学习了如何使用日志记录和调试异步程序。

我们首先使用日志来发现正在运行的软件中的问题,并展示了使用 spdlog 日志库检测死锁的实用性。我们还讨论了许多其他库,并描述了可能适合特定场景的相关功能。

但是,并非所有错误都可以通过使用日志来发现,有些错误可能只能在软件开发生命周期的后期才被发现,当生产中出现一些问题时,甚至在处理程序崩溃和事故时也是如此。调试器是检查正在运行或崩溃的程序、了解其代码路径和查找错误的有用工具。引入了几个示例和调试器命令来处理通用代码,但也特别适用于多线程和异步软件、竞争条件和协程。此外,还引入了 rr 调试器,展示了将反向调试纳入我们的开发人员工具箱的潜力。

在下一章中,我们将学习使用清理器和测试技术来提高异步程序的运行时间和资源使用率的性能和优化技术。