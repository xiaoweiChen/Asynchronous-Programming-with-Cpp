使用 Promise 和 Future 既有优点,也有缺点:

\mySubsubsection{6.3.1.}{优点}

作为管理异步操作的高级抽象,使用Promise和Future来编写和推理并发代码变得更加简单且不容易出错。

Future 和 Promise 支持并发执行任务,让程序能够高效使用多个 CPU 核心。这可以提高计算密集型任务的性能并缩短执行时间。

此外,通过将操作的启动与完成分离来促进异步编程。正如稍后将看到的,这对于 I/O 密集型任务(例如:网络请求或文件操作)特别有用。在这些任务中,程序可以在等待异步操作完成的同时继续执行其他任务。因此,可以返回一个值,也可以返回一个异常,从而允许异常从异步任务传播到等待其完成的调用者代码部分,从而为错误处理和恢复提供了一种更清晰的方式。

它们还提供了一种同步任务完成和检索其结果的机制,这有助于协调并行任务并管理它们之间的依赖关系。

\mySubsubsection{6.3.2.}{缺点}

不幸的是,并非所有都是好消息。

例如,使用 Future 和 Promise 进行异步编程可能会对处理任务之间的依赖关系,或管理异步操作的生命周期时增加复杂性。此外,如果存在循环依赖关系,则可能会发生死锁。

同样,使用Future 和 Promise可能会带来一些性能开销,因为在幕后发生的同步机制涉及协调异步任务和管理共享状态。

与其他并发或异步解决方案一样,使用 Future 和 Promise 的代码调试与同步代码相比更具挑战性,执行流程可能是非线性的并且涉及多个线程。

现在是时候通过一些示例来解决现实问题了。