

现在我们将介绍 C++ 标准库提供的用于支持原子类型和操作的数据类型和函数。正如我们已经看到的,原子操作是不可分割的操作。为了能够在 C++ 中执行原子操作,我们需要使用 C++ 标准库提供的原子类型。


\mySubsubsection{5.5.1.}{C++ 标准库原子类型}

C++ 标准库提供的原子类型在 <atomic> 头文件中定义。

您可以在在线 C++ 参考中查看 <atomic> 标头中定义的所有原子类型的文档,网址为 \url{https://en.cppreference.com/w/cpp/atomic/atomic}。我们不会在这里包含此参考中的所有内容(这就是参考的目的!),但我们将介绍主要概念并使用示例进一步阐述我们的解释。

C++ 标准库提供的原子类型如下:

\begin{itemize}
\item
std::atomic\_flag:原子布尔类型(但与 std::atomic<bool> 不同)。它是唯一保证无锁的原子类型。它不提供加载或存储操作。它是所有原子类型中最基本的。我们将使用它来实现一个非常简单的类似互斥锁的锁。

\item
std::atomic<T>:这是定义原子类型的模板。所有内在类型都有使用此模板定义的相应原子类型。以下是这些类型的一些示例:
\begin{itemize}
\item
std::atomic<bool>(及其别名 atomic\_bool): 我们将使用此原子类型从多个线程实现变量的惰性一次性初始化。

\item
std::atomic<int>(及其别名 atomic\_int):我们已经在简单的计数器示例中看到了这种原子类型。我们将在一个示例中再次使用它来收集统计数据(与计数器示例非常相似)。

\item
std::atomic<intptr\_t>(及其别名 atomic\_intptr\_t)。

\item
C++20 引入了原子智能指针: std::atomic<std::shared\_ptr<U>> 和 std::atomic<std::weak\_ptr<U>{}>。
\end{itemize}

\item
自 C++20 发布以来,出现了一个新的原子类型, std::atomic\_ref<T>。
\end{itemize}

在本章中,我们将重点介绍 std::atomic\_flag 和一些 std::atomic 类型。对于我们在此处提到的其他原子类型,您可以使用上一个链接访问在线 C++ 参考。

在进一步解释这些类型之前,需要澄清一个非常重要的问题:仅仅因为类型是原子的,并不能保证它是无锁的。这里的原子是指不可分割的操作,无锁是指具有特殊的 CPU 原子指令支持。如果某些原子操作没有硬件支持,则 C++ 标准库将使用锁来实现它们。

要检查原子类型是否无锁,我们可以使用任何 std::atomic<T> 类型的以下成员函数:

\begin{itemize}
\item
\verb|bool is_lock_free() const noexcept|:如果此类型的所有原子操作都是无锁的,则返回 true ,否则返回 false(std::atomic\_flag 除外,它保证始终无锁)。其余原子类型可以使用锁(如互斥锁)来实现,以保证操作的原子性。此外,某些原子类型可能只是有时无锁。如果在某个 CPU 中只有对齐的内存访问才能无锁,那么同一原子类型的未对齐对象将使用锁来实现。
\end{itemize}

还有一个常量用于指示原子类型是否始终无锁:

\begin{itemize}
\item
\verb|static constexpr bool is_always_lock_free = /* 实现定义 */|: 如果原子类型始终是无锁的(例如,即使对于未对齐的对象),则此常数的值将为真
\end{itemize}

需要注意的是:原子类型不能保证无锁。 std::atomic<T> 模板并不是一个可以把所有原子类型变成无锁原子类型的神奇机制。

\mySubsubsection{5.5.2.}{C++ 标准库原子操作}

原子操作主要有两种类型:

\begin{itemize}
\item
原子类型的成员函数:例如, std::atomic<int> 具有 load() 成员函数,可以原子地读取其值

\item
自由函数: \verb|const std::atomic_load(const std::atomic<T>* obj)| 函数与前一个函数完全相同
\end{itemize}

您可以在 \url{https://godbolt.org/z/Yhdr3Y1Y8} 上访问以下代码(如果您感兴趣,还可以访问生成的汇编代码)。此代码显示了成员函数和自由函数的使用:

\begin{cpp}
#include <atomic>
#include <iostream>
std::atomic<int> counter {0};
int main() {
    // Using member functions
    int count = counter.load();
    std::cout << count << std::endl;
    count++;
    counter.store(count);

    // Using free functions
    count = std::atomic_load(&counter);
    std::cout << count << std::endl;

    count++;
    std::atomic_store(&counter, count);

    return 0;
}
\end{cpp}

大部分原子操作函数都有一个参数来表示内存顺序。我们在 C++ 内存模型一节中已经解释了什么是内存顺序,以及 C++ 提供了哪些内存顺序类型。

\mySubsubsection{5.5.3.}{示例 – 使用 C++ 实现的简单自旋锁atomic\_flag}

std::atomic\_flag 原子类型是最基本的标准原子类型。它只有两种状态:设置和未设置(我们也可以称之为 true 和 false)。与任何其他标准原子类型相比,它始终是无锁的。因为它非常简单,所以主要用作构建块。

这是原子标志示例的代码:

\begin{cpp}
#include <atomic>
#include <chrono>
#include <iostream>
#include <thread>
#include <vector>

class spin_lock {
    public:
    spin_lock() = default;

    spin_lock(const spin_lock &) = delete;

    spin_lock &operator=(const spin_lock &) = delete;

    void lock() {
        while (flag.test_and_set(std::memory_order_acquire)) {
        }
    }

    void unlock() {
        flag.clear(std::memory_order_release);
    }

private:
    std::atomic_flag flag = ATOMIC_FLAG_INIT;
};
\end{cpp}

在使用 std::atomic\_flag 之前,我们需要对其进行初始化。以下代码显示了如何执行此操作:

\begin{cpp}
std::atomic_flag flag = ATOMIC_FLAG_INIT;
\end{cpp}

这是将 std::atomic\_flag 初始化为确定值的唯一方法。 ATOMIC\_FLAG\_INIT 的值由实现定义。

一旦标志被初始化,我们就可以对其执行两个原子操作:

\begin{itemize}
\item
clear:这会自动将标志设置为 false

\item
test\_and\_set:原子地将标志设置为 true 并获取其先前的值
\end{itemize}

clear 函数只能以放宽、释放或顺序一致性内存顺序调用。 test\_and\_set 函数只能以放宽、获取或顺序一致性调用。使用任何其他内存顺序都会导致未定义的行为。

现在让我们看看如何使用 std::atomic\_flag 实现一个简单的自旋锁。首先,我们知道操作是原子的,因此线程要么清除标志,要么不清除,如果线程清除标志,则标志会被完全清除。线程不可能半清除标志(请记住,对于某些非原子标志,这是可能的)。 test\_and\_set 函数也是原子的,因此标志被设置为 true,我们只需一次即可获得之前的状态。

为了实现基本的自旋锁,我们需要一个原子标志来原子地处理锁状态和两个函数: lock() 来获取锁(就像我们对互斥锁一样)和unlock()来释放锁。

\mySamllsection{简单自旋锁 unlock() 函数}

我们将从最简单的函数 unlock() 开始。它只会重置标志(通过将其设为 false),仅此而已:

\begin{cpp}
void unlock()
{
    flag.clear(std::memory_order_release);
}
\end{cpp}

代码很简单。如果我们省略 std::memory\_order\_seq\_cst 参数,则会应用最严格的内存顺序选项,即顺序一致性。

\mySamllsection{简单自旋锁 lock() 函数}

lock 函数有更多步骤。首先,让我们解释一下它的作用: lock() 必须查看原子标志是否打开。如果关闭,则将其打开并完成。如果标志打开,则继续查看,直到另一个线程将其关闭。我们将使用 test\_and\_set() 来使此函数工作:

\begin{cpp}
void lock()
{
    while (flag.test_and_set(std::memory_order_acquire)) {}
}
\end{cpp}

上述代码的工作方式如下:在 while 循环中, test\_and\_set 将标志设置为 true 并返回先前的值。如果标志已设置,再次设置它不会改变任何内容,并且函数返回 true,因此循环继续设置标志。当 test\_and\_set 最终返回 false 时,这意味着标志已被清除,我们可以退出循环。

\mySamllsection{简单的自旋锁问题}

本章包含了简单的自旋锁实现,以介绍原子类型(std::atomic\_flag,最简单的标准原子类型)和操作(clear 和 test\_and\_set)的使用,但它存在一些严重的问题:

\begin{itemize}
\item
第一个缺点是性能不佳。 repo 中的代码可供您试验。自旋锁的性能会比互斥锁差很多。

\item
线程一直在旋转,等待标志被清除。这种忙碌等待是应该避免的,尤其是在存在线程争用的情况下。
\end{itemize}

您可以尝试上述代码作为本示例。运行该代码时,我们得到了表 5.1 所示的结果。该代码在每个线程中将计数器加 1 2 亿次。

% Please add the following required packages to your document preamble:
% \usepackage{longtable}
% Note: It may be necessary to compile the document several times to get a multi-page table to line up properly
\begin{longtable}{|l|l|l|l|}
\hline
\textbf{} & \textbf{std::mutex} & \textbf{自旋锁} & \textbf{原子计数器} \\ \hline
\endfirsthead
%
\endhead
%
单线程       & 1.03 s              & 1.33 s       & 0.82 s         \\ \hline
双线程       & 10.15 s             & 39.14 s      & 4.52 s         \\ \hline
四线程       & 24.61 s             & 128.84 s     & 9.13 s         \\ \hline
\end{longtable}

\begin{center}
表 5.1:同步原语分析结果
\end{center}

从上表可以看出,简单的自旋锁效果很差,随着线程的增加,效果会越来越差。注意,这个简单的例子仅用于学习,简单的 std::mutex 自旋锁和原子计数器都可以改进,以便原子类型的性能更好。

在本节中,我们了解了 std::atomic\_flag,这是 C++ 标准库提供的最基本的原子类型。有关此类型以及 C++20 中添加的新功能的更多信息,请参阅在线 C++ 参考,网址为 \url{https://en.cppreference.com/w/cpp/atomic/atomic_flag}。

在下一节中,我们将研究如何创建一种简单的方法让线程告诉主线程它已处理了多少个项目。

\mySubsubsection{5.5.4.}{示例 – 线程进度报告}

有时我们想检查线程的进度或在线程完成时收到通知。这可以通过不同的方式实现,例如,使用互斥锁和条件变量,或使用由互斥锁同步的共享变量,如我们在第 4 章中看到的那样。我们还在本章中看到了如何使用原子操作来同步计数器。我们将在以下示例中使用类似的计数器:

\begin{cpp}
#include <atomic>
#include <chrono>
#include <iostream>
#include <thread>

constexpr int NUM_ITEMS{100000};

int main() {
    std::atomic<int> progress{0};

    std::thread worker([&progress] {
        for (int i = 1; i <= NUM_ITEMS; ++i) {
            progress.store(i, std::memory_order_relaxed);
            std::this_thread::sleep_for(std::chrono::milliseconds(1));
        }
    });

    while (true) {
        int processed_items = progress.load(std::memory_order_
        relaxed);
        std::cout << "Progress: "
                  << processed_items << " / " << NUM_ITEMS
                  << std::endl;
        if (processed_items == NUM_ITEMS) {
            break;
        }
        std::this_thread::sleep_for(std::chrono::seconds(10));
    }

    worker.join();

    return 0;
}
\end{cpp}

上述代码实现了一个线程(工作者),用于处理一定数量的项目(此处的处理仅通过使线程休眠来模拟)。每次线程处理一个项目时,它都会增加变量进度。主线程执行一个 while 循环,在每次迭代中,它都会访问进度变量并写入进度报告(处理的项目数)。处理完所有项目后,循环结束。

在此示例中,我们使用 std::atomic<int> 原子类型(原子整数)和两个原子操作:

\begin{itemize}
\item
load():原子地检索进度变量的值

\item
store():原子地修改进度变量的值
\end{itemize}

工作线程处理进度是原子读写的,因此当两个线程访问进度变量时不会发生竞争条件。

load() 和 store() 原子操作有一个额外的参数来指示内存顺序。在这个例子中,我们使用了 std::memory\_order\_relaxed。这是使用宽松内存顺序的典型示例:一个线程增加计数器,另一个线程读取它。我们唯一需要的顺序是读取增加的值,为此,宽松的内存顺序就足够了。

介绍了 load() 和 store() 原子操作来原子地读取和写入变量之后,让我们看另一个简单的统计收集应用程序的示例。

\mySubsubsection{5.5.5.}{示例 – 简单统计数据}

此示例基于与上一个示例相同的理念:一个线程可以使用原子操作将进度(例如,处理的项数)传达给另一个线程。在这个新示例中,一个线程将生成一些数据,另一个线程将读取这些数据。我们需要同步内存访问,因为我们有两个线程共享同一内存,并且其中至少一个线程正在更改内存。与上一个示例一样,我们将使用原子操作来实现此目的。

以下代码声明了我们将用于收集统计数据的原子变量 - 一个用于处理的项目数,另外两个( 分别用于总处理时间和每个项目的平均处理时间):

\begin{cpp}
std::atomic<int> processed_items{0};
std::atomic<float> total_time{0.0f};
std::atomic<double> average_time{0.0};
\end{cpp}

我们使用原子浮点数和双精度数来计算总时间和平均时间。在完整的示例代码中,我们确保这两种类型都是无锁的,这意味着它们使用来自 CPU 的原子指令(所有现代 CPU 都应该具有此功能)。

现在让我们看看工作线程如何使用变量:

\begin{cpp}
processed_items.fetch_add(1, std::memory_order_relaxed);
total_time.fetch_add(elapsed_s, std::memory_order_relaxed);
average_time.store(total_time.load() / processed_items.load(),
std::memory_order_relaxed);
\end{cpp}

第一行以原子方式将处理的项目增加 1。 fetch\_add 函数将变量值加 1 并返回旧值(在本例中我们不使用它)。

第二行将 elapsed\_s(处理一个项目所花费的时间,以秒为单位)添加到 total\_time 变量中,我们使用该变量来跟踪处理所有项目所花费的时间。

然后,第三行通过原子读取 total\_time 和treated\_items 并原子地将结果写入 average\_time 来计算每个项目的平均时间。或者,我们可以使用来自 fetch\_add() 调用的值来计算平均时间,但它们不包括最后处理的项目。我们也可以在主线程中计算 average\_time,但我们在这里在工作线程中执行此操作,仅作为示例并练习使用原子操作。请记住,我们的目标(至少在本章中)不是速度,而是学习如何使用原子操作。

以下是统计示例的完整代码:

\begin{cpp}
#include <atomic>
#include <chrono>
#include <iostream>
#include <random>
#include <thread>

constexpr int NUM_ITEMS{10000};

void process() {
    std::random_device rd;
    std::mt19937 gen(rd());

    std::uniform_int_distribution<> dis(1, 20);

    int sleep_duration = dis(gen);
        std::this_thread::sleep_for(std::chrono::milliseconds(sleep_
    duration));
}

int main() {
    std::atomic<int> processed_items{0};
    std::atomic<float> total_time{0.0f};
    std::atomic<double> average_time{0.0};

    std::thread worker([&] {
        for (int i = 1; i <= NUM_ITEMS; ++i) {
            auto now = std::chrono::high_resolution_clock::now();
            process();
            auto elapsed =
                std::chrono::high_resolution_clock::now() - now;
            float elapsed_s =
                std::chrono::duration<float>(elapsed).count();

            processed_items.fetch_add(1, std::memory_order_relaxed);
            total_time.fetch_add(elapsed_s, std::memory_order_relaxed);
            average_time.store(total_time.load() / processed_items.
            load(), std::memory_order_relaxed);
        }
    });

    while (true) {
        int items = processed_items.load(std::memory_order_relaxed);
        std::cout << "Progress: " << items << " / " << NUM_ITEMS <<
        std::endl;

        float time = total_time.load(std::memory_order_relaxed);
        std::cout << "Total time: " << time << " sec" << std::endl;
        double average = average_time.load(std::memory_order_relaxed);
        std::cout << "Average time: " << average * 1000 << " ms" << std::endl;

        if (items == NUM_ITEMS) {
            break;
        }
        std::this_thread::sleep_for(std::chrono::seconds(5));
    }
    worker.join();

    return 0;
}
\end{cpp}

总结一下目前为止我们看到的情况:

\begin{itemize}
\item
C++ 标准原子类型:我们使用 std::atomic\_flag 实现了一个简单的自旋锁,并且我们使用了一些 std::atomic<T> 类型来实现线程之间的简单数据通信。我们所见过的所有原子类型都是无锁的。

\item
load() 原子操作以原子方式读取原子变量的值。

\item
store() 原子操作以原子方式将新值写入原子变量。

\item
clear() 和 test\_and\_set(), std::atomic\_ 标志提供的特殊原子操作。

\item
fetch\_add(),用于原子地将某个值添加到原子变量并获取其先前的值。整数和浮点类型还实现了 fetch\_sub(),用于从原子变量中减去某个值并返回其先前的值。一些用于执行按位逻辑运算的函数已专门针对整数类型实现: fetch\_and()、 fetch\_or() 和 fetch\_xor()。
\end{itemize}

下表总结了原子类型和操作。有关详尽描述,请参阅在线 C++ 参考: \url{https://en.cppreference.com/w/cpp/atomic/atomic} 表格中显示了三个新操作: exchange、 compare\_exchange\_weak 和 compare\_exchange\_strong 。稍后我们将使用示例对其进行解释。大多数操作(即函数,而不是运算符)都有另一个用于内存顺序的参数。

% Please add the following required packages to your document preamble:
% \usepackage{longtable}
% Note: It may be necessary to compile the document several times to get a multi-page table to line up properly
\begin{longtable}{|l|l|l|l|l|l|}
\hline
\textbf{操作} &
\textbf{\begin{tabular}[c]{@{}l@{}}atomic\_\\ flag\end{tabular}} &
\textbf{\begin{tabular}[c]{@{}l@{}}atomic\\ \textless{}bool\textgreater{}\end{tabular}} &
\textbf{\begin{tabular}[c]{@{}l@{}}atomic\\ \textless{}integral\textgreater{}\end{tabular}} &
\textbf{\begin{tabular}[c]{@{}l@{}}atomic\\ \textless{}floating-point\textgreater{}\end{tabular}} &
\textbf{\begin{tabular}[c]{@{}l@{}}atomic\\ \textless{}other\textgreater{}\end{tabular}} \\ \hline
\endfirsthead
%
\endhead
%
test\_and\_set                                                                                        & YES &     &     &     &     \\ \hline
Clear                                                                                                 & YES &     &     &     &     \\ \hline
Load                                                                                                  &     & YES & YES & YES & YES \\ \hline
Store                                                                                                 &     & YES & YES & YES & YES \\ \hline
fetch\_add, +=                                                                                        &     &     & YES & YES &     \\ \hline
fetch\_sub, -=                                                                                        &     &     & YES & YES &     \\ \hline
fetch\_and, \&=                                                                                       &     &     & YES &     &     \\ \hline
fetch\_or, |=                                                                                         &     &     & YES &     &     \\ \hline
fetch\_xor, \textasciicircum{}=                                                                       &     &     & YES &     &     \\ \hline
++, --                                                                                                &     &     & YES &     &     \\ \hline
Exchange                                                                                              &     & YES & YES & YES & YES \\ \hline
\begin{tabular}[c]{@{}l@{}}compare\_\\ exchange\_weak,\\ compare\_\\ exchange\_\\ strong\end{tabular} &     & YES & YES & YES & YES \\ \hline
\end{longtable}

\begin{center}
表 5.2:原子类型和操作
\end{center}

让我们回顾一下 is\_lock\_free() 函数和 is\_always\_lock\_free 常量。我们看到,如果 is\_lock\_free() 为真,则原子类型具有使用特殊 CPU 指令的无锁操作。原子类型有时可以是无锁的,因此 is\_always\_lock\_free 常量告诉我们该类型是否始终是无锁的。到目前为止,我们看到的所有类型都是无锁的。让我们看看当原子类型是非无锁时会发生什么
。
以下显示了非无锁原子类型的代码:

\begin{cpp}
#include <atomic>
#include <iostream>

struct no_lock_free {
    int a[128];

    no_lock_free() {
        for (int i = 0; i < 128; ++i) {
            a[i] = i;
        }
    }
};

int main() {
    std::atomic<no_lock_free> s;

    std::cout << "Size of no_lock_free: " << sizeof(no_lock_free) << "bytes\n";
    std::cout << "Size of std::atomic<no_lock_free>: " << sizeof(s) <<
    " bytes\n";

    std::cout << "Is std::atomic<no_lock_free> always lock-free: " <<
    std::boolalpha << std::atomic<no_lock_free>::is_always_lock_free <<
    std::endl;

    std::cout << "Is std::atomic<no_lock_free> lock-free: " <<
    std::boolalpha << s.is_lock_free() << std::endl;

    no_lock_free s1;
    s.store(s1);

    return 0;
}
\end{cpp}

执行代码时,您会注意到 std::atomic<no\_lock\_free> 类型不是无锁的。其大小为 512 字节,这是造成这种情况的原因。当我们为原子变量赋值时,该值是原子写入的,但此操作不使用 CPU 原子指令,也就是说,它不是无锁的。此操作的实现取决于编译器,但通常它使用互斥锁或特殊自旋锁(例如 Microsoft Visual C++)。

这里的教训是,所有原子类型都有原子操作,但它们并非都是神奇的无锁类型。如果原子类型不是无锁的,那么使用锁来实现它总是更好的选择。

我们了解到,有些原子类型不是无锁的。现在我们来看另一个示例,该示例展示了我们尚未涉及的原子操作: exchange 和 compare\_exchange 操作。

\mySubsubsection{5.5.6.}{示例 – 惰性一次性初始化}

有时初始化对象的成本可能很高。例如,给定对象可能需要连接到数据库或服务器,而建立此连接可能需要很长时间。在这些情况下,我们应该在使用对象之前初始化它,而不是在程序中定义它时初始化它。这称为延迟初始化。现在让我们假设多个线程需要第一次使用该对象。如果多个线程初始化该对象,那么将创建不同的连接,这将是错误的,因为该对象只打开和关闭一个连接。因此,必须避免多次初始化。为了确保对象只初始化一次,我们将使用一种称为延迟一次性初始化的方法。

下面显示了延迟一次性初始化的代码:

\begin{cpp}
#include <atomic>
#include <iostream>
#include <random>
#include <thread>
#include <vector>

constexpr int NUM_THREADS{8};

void process() {
    std::random_device rd;
    std::mt19937 gen(rd());
    std::uniform_int_distribution<> dis(1, 1000000);

    int sleep_duration = dis(gen);

    std::this_thread::sleep_for(std::chrono::microseconds(sleep_
    duration));
}

int main() {
    std::atomic<int> init_thread{0};

    auto worker = [&init_thread](int i) {
        process();

        int init_value = init_thread.load(std::memory_order::seq_cst);
        if (init_value == 0) {
            int expected = 0;
            if (init_thread.compare_exchange_strong(expected, i,
            std::memory_order::seq_cst)) {
                std::cout << "Previous value of init_thread: " <<
                expected << "\n";
                std::cout << "Thread " << i << " initialized\n";
            } else {
                // init_thread was already initialized
            }
        } else {
            // init_thread was already initialized
        }
    };

    std::vector<std::thread> threads;
    for (int i = 1; i <= NUM_THREADS; ++i) {
        threads.emplace_back(worker, i);
    }

    for (auto &t: threads) {
        t.join();
    }

    std::cout << "Thread: " << init_thread.load() << " initialized\n";
    return 0;
}
\end{cpp}

我们在本章前面看到的原子类型操作表中有一些操作我们还没有讨论过。现在我们将使用一个示例来解释 compare\_exchange\_strong。在示例中,我们有一个以 0 值开头的变量。多个线程正在运行,每个线程都有一个唯一的整数 ID(1、 2、 3 等)。我们希望将变量的值设置为首先设置它的线程的 ID,并且只初始化一次变量。在第 4 章中,我们学习了 std::once\_flag 和 std::call\_once,我们可以用它们来实现这种一次性初始化,但本章是关于原子类型和操作的,所以我们将使用它们来实现我们的目标。

为了确保 init\_thread 变量的初始化仅进行一次,并避免由于来自多个线程的写入访问而导致的竞争条件,我们使用原子 int。行 [1] 原子地读取 init\_thread 的内容。如果值不为 0,则意味着它已经初始化,工作线程不执行任何其他操作。

init\_thread 的当前值存储在 expected 变量中,该变量表示我们尝试初始化 init\_thread 时期望它具有的值。现在行 [2] 执行以下步骤:

\begin{enumerate}
\item
将 init\_thread 当前值与预期值(同样,该值等于 0)进行比较。

\item
如果比较不成功,则将 init\_thread 当前值复制到 expected 中并返回 false。

\item
如果比较成功,则将 init\_thread 当前值复制到 expected 中,然后将 init\_thread 当前值设置为 i 并返回 true。
\end{enumerate}

仅当 compare\_exchange\_strong 返回 true 时,当前线程才会初始化 init\_thread。另外,请注意,我们需要再次进行比较(即使行 [1] 返回 0 作为 init\_thread 的当前值),因为另一个线程可能已经初始化了该变量。

需要注意的是,如果 compare\_exchange\_strong 返回 false,则比较失败,如果返回 true,则比较成功。 compare\_exchange\_strong 始终如此。另一方面,即使比较成功, compare\_exchange\_weak 也可能失败(即返回 false)。使用它的原因是,在某些平台上,在循环内调用它可以提供更好的性能。

关于这两个函数的更多信息,可以参阅在线 C++ 参考: \url{https://en.cppreference.com/w/cpp/ato mic/atomic/compare_exchange} 在有关 C++ 标准库原子类型和操作的本节中,我们看到了以下内容:

\begin{itemize}
\item
最常用的标准原子类型,例如 std::atomic\_flag 和 std::atomic<int>

\item
最常用的原子操作: load()、 store() 和 exchange\_compare\_strong()/exchange\_compare\_weak()

\item
包含这些原子类型和操作的基本示例,包括惰性一次性初始化和线程进度通信
\end{itemize}

我们已经多次提到,大多数原子操作(函数)都允许我们选择想要使用的内存顺序。在下一节中,我们将实现一个无锁编程示例: SPSC 无锁队列。

