在 Linux 操作系统领域,守护进程是一个基本组件,它在后台悄悄运行,默默地执行基本任务,无需交互式用户的直接参与。这些进程传统上用以字母 d 结尾的名称来标识,例如 ssh d 表示安全 Shell (SSH) 守护进程, httpd 表示 Web 服务器守护进程。它们在处理对操作系统及其上运行的应用程序都至关重要的系统级任务方面发挥着至关重要的作用。

守护进程可用于多种用途,包括文件服务、 Web 服务和网络通信,以及日志记录和监控服务。守护进程具有自主性和弹性,从系统启动时开始运行,并持续运行,直到系统关闭。与用户启动和控制的常规进程不同,守护进程具有以下独特特征:

\begin{itemize}
\item
后台操作:
\begin{itemize}
\item
守护进程在后台运行

\item
它们缺乏用于直接用户交互的控制终端

\item
它们不需要用户界面或手动干预来执行任务
\end{itemize}

\item
用户独立性:
\begin{itemize}
\item
守护进程独立于用户会话运行

\item
它们自主运行,无需用户直接参与

\item
它们等待系统事件或特定请求来触发其操作
\end{itemize}

\item
以任务为导向:
\begin{itemize}
\item
每个守护进程都经过定制,以执行特定任务或一组任务

\item
它们旨在处理特定功能或监听特定事件或请求

\item
这可确保高效执行任务
\end{itemize}
\end{itemize}

创建守护进程不仅仅涉及在后台运行进程。为了确保守护进程有效运行,开发人员必须考虑几个关键步骤:

\begin{enumerate}
\item
从终端分离:使用 fork() 系统调用将守护进程从终端分离。父进程在 fork 之后退出,子进程则在后台运行。

\item
会话创建: setsid() 系统调用创建新会话,并将调用进程指定为会话和进程组的领导者。此步骤对于完全从终端分离至关重要。

\item
工作目录更改:为防止阻塞文件系统的卸载,守护进程通常将其工作目录更改为根目录。

\item
文件描述符处理: 守护进程关闭继承的文件描述符, stdin、 stdout 和 stderr 通常会重定向到 /dev/null。

\item
信号处理:正确处理信号(例如用于重新加载配置的 SIGHUP 或用于正常关闭的 SIGTERM )对于有效的守护进程管理至关重要。
\end{enumerate}

守护进程通过各种 IPC 机制与其他进程或守护进程进行通信。

守护进程是许多异步系统架构不可或缺的一部分,提供基本服务无需直接用户交互。守护进程的一些突出用例包括:

\begin{itemize}
\item
Web 服务器: httpd 和 nginx 等守护进程响应客户端请求提供网页,同时处理多个请求并确保无缝的网页浏览。

\item
数据库服务器: mysqld 和 postgresql 等守护进程管理数据库服务,允许各种应用程序异步访问和操作数据库。

\item
文件服务器: smbd、 nfsd 等守护进程提供网络文件服务,实现不同系统之间的异步文件共享和访问。

\item
日志记录和监控: syslogd 和 snmpd 等守护进程收集和记录系统事件,提供系统健康和性能的异步监控。
\end{itemize}

总而言之,守护进程是 Linux 系统必不可少的组件,它们在后台默默地执行关键任务,以确保系统平稳运行和应用程序高效执行。它们的自主性和弹性使它们对于维护系统稳定性以及为用户和应用程序提供基本服务不可或缺。

我们已经了解了进程和守护进程(一种特殊类型的进程)。一个进程可以有一个或多个执行线程。在下一节中,我们将介绍线程。






