本章中,介绍了如何使用 C++ 标准库提供的基于锁的同步原语。

首先解释了条件竞争和互斥的必要性,然后研究了 std::mutex 以及如何使用它来解决条件竞争。还介绍了使用锁进行同步时的主要问题:死锁和活锁。

之后,研究了条件变量,并使用互斥锁和条件变量实现了同步队列。最后,了解了 C++20 中引入的新同步原语:信号量、门闩和栅栏。

最后,研究了 C++ 标准库提供的仅运行一次函数的机制。

本章中,了解了线程同步的基本构成要素以及多线程异步编程的基础,基于锁的线程同步是同步线程最常用的方法。

下一章,将介绍无锁线程同步。首先回顾 C++20 标准库提供的原子性、原子操作和原子类型。展示无锁绑定单生产者单消费者队列的实现,还将介绍 C++ 内存模型。