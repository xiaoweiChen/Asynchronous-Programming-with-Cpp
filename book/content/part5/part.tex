最后一部分中,将重点介绍调试、测试和优化多线程和异步程序性能的基本实践。


首先使用日志记录和高级调试工具和技术(包括反向调试和代码清理器)来识别和解决异步应用程序中的细微错误,例如:崩溃、死锁、竞争条件、内存泄漏和线程安全问题,然后使用 GoogleTest 框架针对异步代码量身定制测试策略。最后,将深入研究性能优化,理解缓存共享、错误共享等关键概念以及如何缓解性能瓶颈。掌握这些技术将提供一个全面的工具包,用于识别、诊断和提高异步应用程序的质量和性能。

本部分包含以下章节:

\begin{itemize}
\item
第 11 章,记录和调试异步软件

\item
第 12 章,清理和测试异步软件

\item
第 13 章,提高异步软件性能
\end{itemize}
