
互斥是并发编程中的一个基本概念,可确保多个线程或进程不会同时访问共享资源(例如:共享变量、代码的关键部分或文件或网络连接)。互斥对于避免条件竞争至关重要。

想象一下,一家小咖啡店只有一台浓缩咖啡机,这台机器一次只能制作一杯浓缩咖啡,所以这台机器是所有咖啡师必须共享的关键资源。

咖啡店有三位咖啡师:爱丽丝、鲍勃和卡罗尔。他们同时使用咖啡机,但不能同时使用:鲍勃将适量的现磨咖啡放入机器中,开始制作浓缩咖啡。然后,爱丽丝做了同样的事情,但先从机器中取出咖啡,以为鲍勃只是忘了做这件事。然后鲍勃从机器中取出浓缩咖啡,之后,爱丽丝发现没有浓缩咖啡了!这是一场灾难——如同之前的计数器程序一样。

为了解决咖啡店的问题,他们任命卡罗尔为机器管理员。在使用机器之前,爱丽丝和鲍勃都会询问她是否可以开始制作新的浓缩咖啡。这样就可以解决问题。

回到计数器程序,如果可以一次只允许一个线程访问计数器(卡罗尔在咖啡店里所做的),软件问题也将得到解决。互斥是一种可用于控制并发线程访问内存的机制。 C++ 标准库提供了 std::mutex 类,这是一个同步原语,用于保护共享数据不被两个或多个线程同时访问。

上一节中,看到的这个新版本的代码实现了两种并发增加计数器的方式:如上一节所述的自由访问,以及使用互斥的同步访问:

\begin{cpp}
#include <iostream>
#include <mutex>
#include <thread>

std::mutex mtx;
int counter = 0;

int main() {
    auto funcWithoutLocks = [] {
        for (int i = 0; i < 1000000; ++i) {
            ++counter;
        };
    };

    auto funcWithLocks = [] {
        for (int i = 0; i < 1000000; ++i) {
            mtx.lock();
            ++counter;
            mtx.unlock();
        };
    };

    {
        counter = 0;
        std::thread t1(funcWithoutLocks);
        std::thread t2(funcWithoutLocks);

        t1.join();
        t2.join();

        std::cout << "Counter without using locks: " << counter <<
        std::endl;
    } {
        counter = 0;
        std::thread t1(funcWithLocks);
        std::thread t2(funcWithLocks);

        t1.join();
        t2.join();

        std::cout << "Counter using locks: " << counter << std::endl;
    }
    return 0;
}
\end{cpp}

当线程运行 funcWithLocks 时,会在增加计数器之前使用mtx.lock()获取锁。当计数器增加,线程就会释放锁(mtx.unlock())。

该锁只能由一个线程拥有。例如,t1 获取了锁,然后 t2 也尝试获取该锁,则 t2 将阻塞等待,直到锁可用。由于任何时候都只有一个线程可以拥有锁,因此此同步原语称为互斥(来自互斥)。运行此程序几次,将始终得到正确的结果: 2000000。

本节中,介绍了互斥的概念,并了解到 C++ 标准库提供了 std::mutex 类作为线程同步的原语。在下一节中,将详细介绍 std::mutex。

\mySubsubsection{4.3.1.}{C++标准库互斥实现}

上一节中,介绍了互斥和互斥锁的概念,以及为什么需要它们来同步并发内存访问。本节中,将了解 C++ 标准库提供的用于实现互斥的类。还将了解 C++ 标准库提供的一些辅助类,它们使互斥锁的使用更加容易。

下表总结了 C++ 标准库提供的互斥锁类及其主要功能:

% Please add the following required packages to your document preamble:
% \usepackage{longtable}
% Note: It may be necessary to compile the document several times to get a multi-page table to line up properly
\begin{longtable}{|l|l|l|l|}
\hline
\textbf{互斥类型}       & \textbf{可访问}                                                    & \textbf{可递归} & \textbf{可超时} \\ \hline
\endfirsthead
%
\endhead
%
std::mutex                   & 独占 & NO  & NO  \\ \hline
std::recursive\_mutex        & 独占 & YES & NO  \\ \hline
std::shared\_mutex        & \begin{tabular}[c]{@{}l@{}}1 - 独占\\ N - 共享\end{tabular} & NO                 & NO               \\ \hline
std::timed\_mutex            & 独占 & NO  & YES \\ \hline
std::recursive\_timed\_mutex & 独占 & YES & YES \\ \hline
std::shared\_timed\_mutex & \begin{tabular}[c]{@{}l@{}}1 - 独占\\ N - 共享\end{tabular} & NO                 & YES              \\ \hline
\end{longtable}

\begin{center}
表 4.2: C++ 标准库中的互斥类
\end{center}

\mySamllsection{std::mutex}

std::mutex 类是在 C++11 中引入的,是 C++ 标准库提供的最重要和最常用的同步原语之一。

std::mutex 是一个同步原语,可用于保护共享数据不被多个线程同时访问。std::mutex 类提供独占、非递归所有权语义。主要特点如下:

\begin{itemize}
\item
调用线程从成功调用 lock() 或 try\_lock() 开始直到调用 unlock() 为止都拥有该互斥锁。

\item
调用线程在调用 lock() 或 try\_lock() 之前不得拥有互斥锁,这是 std::mutex 的非递归所有权语义属性。

\item
当一个线程拥有一个互斥锁时,所有其他线程将阻塞(调用 lock() 时)或收到 false 返回值(调用 try\_lock() 时)。这是 std::mutex 的独占所有权语义。
\end{itemize}

如果拥有互斥锁的线程尝试再次获取,则结果行为未定义。通常,发生这种情况时会引发异常,但这由实现定义。如果线程释放互斥锁后,尝试再次释放它,这也是未定义行为(如前一种情况)。当线程锁定互斥锁时,互斥锁销毁,或者线程在未释放锁的情况下终止,也会导致未定义行为。

std::mutex 类有三种方法:

\begin{itemize}
\item
lock():获取互斥锁。如果互斥锁已锁定,则调用线程将阻止,直到互斥锁解锁。从应用程序的角度来看,这就像调用线程等待互斥锁可用一样。

\item
try\_lock():调用此函数时,返回 true(表示互斥锁已成功锁定)或 false(表示互斥锁已锁定)。try\_lock 是非阻塞的,调用线程可以获取互斥锁,也可以不获取,但不会像调用 lock() 时那样被阻塞。try\_lock() 方法通常用于我们不希望线程等待互斥锁可用时。当我们希望线程继续进行某些处理并稍后尝试获取互斥锁时,将调用try\_lock()。

\item
unlock():调用 unlock() 释放互斥锁。
\end{itemize}

\mySamllsection{std::recursive\_mutex}

std::mutex 类提供独占、非递归所有权语义。虽然独占所有权语义至少对于线程来说总是必需的(毕竟它是一种互斥机制),但在某些情况下,可能需要以递归方式获取互斥锁。例如,递归函数可能需要获取互斥锁。可能还需要在从另一个获取相同互斥锁的函数 f() 调用的函数 g() 中获取互斥锁。

std::recursive\_mutex 类提供独占、递归语义。其主要特点如下:

\begin{itemize}
\item
调用线程可以多次获取同一个互斥锁。将拥有该互斥锁,直到释放该互斥锁的次数与获取该互斥锁的次数相同。例如,如果一个线程递归获取互斥锁三次,将拥有该互斥锁,直到它第三次释放该互斥锁。

\item
递归互斥锁可以递归获取的最大次数未指定,因此由实现定义。一旦互斥锁的获取次数达到最大次数,对 lock() 的调用将抛出 std::system\_error,而对 try\_lock() 的调用将返回 false。

\item
所有权与 std::mutex 相同:如果一个线程拥有一个 std::recursive\_mutex 类,则其他线程在尝试通过调用 lock() 来获取它时都将阻塞,或者在调用 try\_lock() 时将返回 false。
\end{itemize}

std::recursive\_mutex 接口与 std::mutex 完全相同。

\mySamllsection{std::shared\_mutex}

std::mutex 和 std::shared\_mutex 都具有独占所有权语义,并且只能有一个线程成为互斥锁所有者。但某些情况下,可能需要让多个线程同时访问受保护的数据,并只给一个线程独占访问权限。

计数器示例要求每个线程都拥有对单个变量的独占访问权限,如果有只需要读取计数器中当前值的线程,并且只有一个线程来增加其值,那么让读取线程同时访问计数器并给予写入线程独占访问权限会更好。

此功能使用“读者-写者锁”实现。C++ 标准库实现了 std::shared\_mutex 类,具有类似(但不完全相同)的功能。

std::shared\_mutex 与其他互斥类型的主要区别在于它有两个访问级别:

\begin{itemize}
\item
共享:多个线程可以共享同一互斥锁的所有权。通过 lock\_shared()、 try\_lock\_shared ()/unlock\_shared() 来获取/释放共享所有权。当至少一个线程已获取对锁的共享访问权限时,其他线程都无法获得对锁的独占访问权限,但可以获得共享访问权限。

\item
独占:只有一个线程可以拥有互斥锁。通过调用 lock()、 try\_lock()/unlock() 来获取/释放独占所有权。当一个线程获得对锁的独占访问权限时,其他线程都无法获得对锁的共享或独占访问权限。
\end{itemize}

看一个使用 std::shared\_mutex 的简单例子:

\begin{cpp}
#include <algorithm>
#include <chrono>
#include <iostream>
#include <shared_mutex>
#include <thread>

int counter = 0;

int main() {
    using namespace std::chrono_literals;

    std::shared_mutex mutex;

    auto reader = [&] {
        for (int i = 0; i < 10; ++i) {
            mutex.lock_shared();
            // Read the counter and do something
            mutex.unlock_shared();
        }
    };

    auto writer = [&] {
        for (int i = 0; i < 10; ++i) {
            mutex.lock();
            ++counter;
            std::cout << "Counter: " << counter << std::endl;
            mutex.unlock();
            std::this_thread::sleep_for(10ms);
        }
    };

    std::thread t1(reader);
    std::thread t2(reader);
    std::thread t3(writer);
    std::thread t4(reader);
    std::thread t5(reader);
    std::thread t6(writer);

    t1.join();
    t2.join();
    t3.join();
    t4.join();
    t5.join();
    t6.join();

    return 0;
}
\end{cpp}

该示例使用 std::shared\_mutex 同步六个线程:两个线程是写入器,增加计数器的值并需要独占访问权限。其余四个线程只读取计数器,仅需要共享访问权限。请注意,为了使用 std::shared\_mutex,需要包含 <shared\_mutex> 头文件。

\mySamllsection{定时互斥类型}

到目前为止,所看到的互斥类型在我们想要获取独占使用的锁时的行为方式相同:

\begin{itemize}
\item
std::lock():调用线程阻塞,直到锁可用

\item
std::try\_lock():如果锁不可用,则返回 false
\end{itemize}

对于 std::lock() 来说,调用线程可能会等待很长时间,如果线程无法获取锁,可能只需要等待一段时间,然后让线程继续进行一些处理。

为了实现这个目标,可以使用 C++ 标准库提供的定时互斥锁: std::timed\_mutex、 std::recursive\_timed\_mutex 和 std::shared\_time\_mutex。

它们与非定时对应物类似,并实现以下功能允许等待锁在特定时间段内可用:

\begin{itemize}
\item
try\_lock\_for():尝试锁定互斥锁并阻止线程,直到指定的持续时间结束(超时)。如果互斥锁在指定的持续时间之前被锁定,则返回 true;否则,返回 false。

如果指定的持续时间小于或等于零(timeout\_duration.zero()),则该函数的行为与 try\_lock() 完全相同。

由于调度或争用延迟,此函数的阻塞时间可能会超过指定的时间段。

\item
try\_lock\_until():尝试锁定互斥锁,直到指定的超时时间或互斥锁锁定(以先到者为准)。这种情况下,可指定未来的一个时间点作为等待的限制。
\end{itemize}

下面的示例显示如何使用 std::try\_lock\_for():

\begin{cpp}
#include <algorithm>
#include <chrono>
#include <iostream>
#include <mutex>
#include <thread>
#include <vector>

constexpr int NUM_THREADS = 8;
int counter = 0;
int failed = 0;

int main() {
    using namespace std::chrono_literals;

    std::timed_mutex tm;
    std::mutex m;

    auto worker = [&] {
        for (int i = 0; i < 10; ++i) {
            if (tm.try_lock_for(10ms)) {
                ++counter;
                std::cout << "Counter: " << counter << std::endl;
                std::this_thread::sleep_for(10ms);
                m.unlock();
            }
            else {
                m.lock();
                ++failed;
                std::cout << "Thread " << std::this_thread::get_id()
                << " failed to lock" << std::endl;
                m.unlock();
            }
            std::this_thread::sleep_for(12ms);
        }
    };

    std::vector<std::thread> threads;
    for (int i = 0; i < NUM_THREADS; ++i) {
        threads.emplace_back(worker);
    }

    for (auto& t : threads) {
        t.join();
    }

    std::cout << "Counter: " << counter << std::endl;
    std::cout << "Failed: " << failed << std::endl;

    return 0;
}
\end{cpp}

上述代码使用了两个锁: tm(定时互斥锁),用于在获取 tm 成功时同步对 counter 的访问和写入屏幕; m(非定时互斥锁),用于在获取 tm 不成功时同步对 failed 的访问和屏幕输出。

\mySubsubsection{4.3.2.}{使用锁时会遇到的问题}

已经看到了仅使用互斥锁(锁)的示例。如果只需要一个互斥锁,并且正确地获取和释放,那么编写正确的多线程代码通常并不困难。当需要多个锁,代码复杂性就会增加。使用多个锁时的两个常见问题是死锁和活锁。

\mySamllsection{死锁}

为了执行某项任务,一个线程需要访问两个资源,并且不能被两个或多个线程同时访问(需要互斥来正确同步对所需资源的访问)。每个资源都与不同的 std::mutex 类同步。这种情况下,线程必须获取第一个资源互斥锁,然后获取第二个资源互斥锁,最后处理资源并释放两个互斥锁。

当两个线程尝试执行上述处理时,可能会发生如下情况: 线程 1 和线程 2 需要获取两个互斥锁才能执行所需的处理。线程 1 获取第一个互斥锁,线程 2 获取第二个互斥锁。然后,线程 1 将永远阻塞,等待第二个互斥锁可用,而线程 2 将永远阻塞,等待第一个互斥锁可用,这被称为死锁,也是多线程代码中最常见的问题。

\mySamllsection{活锁}

解决死锁的一个可能方法如下:当一个线程尝试获取锁时,只会阻塞一段有限的时间,如果仍然不成功,将释放它可能已获取的锁。

例如,线程 1 获取了第一个锁,线程 2 获取了第二个锁。经过一段时间后,线程 1 仍未获取第二个锁,因此释放了第一个锁。线程 2 也可能完成等待并释放它获取的锁(在此示例中为第二个锁)。

这种解决方案有时可能有效,但并不正确。想象一下:线程 1 已获取第一个锁,并获取了第二个锁。一段时间后,两个线程都释放了已经获取的锁,然后再次获取相同的锁。然后,线程释放锁,然后再次获取锁,依此类推。

线程无法做任何事情,只能获取锁、等待、释放锁,然后重复上述操作。这种情况称为活锁,因为线程不会永远等待(如死锁情况),而是处于某种活动状态,并不断获取和释放锁。

解决死锁和活锁问题的最常见方法是以一致的顺序获取锁。例如,一个线程需要获取两个锁,将始终先获取第一个锁,然后再获取第二个锁。锁将以相反的顺序释放(先释放第二个锁,然后释放第一个锁)。如果第二个线程尝试获取第一个锁,将不得不等到第一个线程释放两个锁,这样就永远不会发生死锁。

本节中,了解了 C++ 标准库提供的互斥锁类。研究了其主要功能以及使用多个锁时可能遇到的问题。

下一节中,将介绍 C++ 标准库提供的机制,以使获取和释放互斥锁更加容易。





























