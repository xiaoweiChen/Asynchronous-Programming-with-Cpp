条件变量是 C++ 标准库提供的另一个同步原语,允许多个线程相互通信,还允许多个线程等待来自另一个线程的通知。条件变量始终与互斥锁相关联。

以下示例中,线程必须等待计数器等于某个值:

\begin{cpp}
#include <chrono>
#include <condition_variable>
#include <iostream>
#include <mutex>
#include <thread>
#include <vector>

int counter = 0;

int main() {
    using namespace std::chrono_literals;

    std::mutex mtx;
    std::mutex cout_mtx;
    std::condition_variable cv;

    auto increment_counter = [&] {
        for (int i = 0; i < 20; ++i) {
            std::this_thread::sleep_for(100ms);
            mtx.lock();
            ++counter;
            mtx.unlock();
            cv.notify_one();
        }
    };

    auto wait_for_counter_non_zero_mtx = [&] {
        mtx.lock();
        while (counter == 0) {
            mtx.unlock();
            std::this_thread::sleep_for(10ms);
            mtx.lock();
        }
        mtx.unlock();
        std::lock_guard<std::mutex> cout_lck(cout_mtx);
        std::cout << "Counter is non-zero" << std::endl;
    };

    auto wait_for_counter_10_cv = [&] {
        std::unique_lock<std::mutex> lck(mtx);
        cv.wait(lck, [] { return counter == 10; });

        std::lock_guard<std::mutex> cout_lck(cout_mtx);
        std::cout << "Counter is: " << counter << std::endl;
    };

    std::thread t1(wait_for_counter_non_zero_mtx);
    std::thread t2(wait_for_counter_10_cv);
    std::thread t3(increment_counter);

    t1.join();
    t2.join();
    t3.join();

    return 0;
}
\end{cpp}

等待某个条件有两种方式:一种是循环等待,并使用互斥锁作为同步机制,这在 wait\_for\_counter\_non\_zero\_mtx 中实现。该函数获取锁,读取 counter 中的值,然后释放锁。然后,休眠 10 毫秒,然后再次获取锁。这在 while 循环中完成,直到 counter 非零。

条件变量简化了前面的代码。 wait\_for\_counter\_10\_cv 函数会等待直到计数器等于 10。线程将等待 cv 条件变量,直到它被 t1 通知,线程会循环增加计数器。

wait\_for\_counter\_10\_cv 函数的工作方式如下:条件变量 cv 等待互斥锁 mtx。调用 wait() 后,条件变量锁定互斥锁并等待,直到条件为真(条件在作为第二个参数传递给 wait 函数的 l ambda 中实现)。如果条件不为真,条件变量将保持等待状态,直到收到信号并释放互斥锁。一旦满足条件,条件变量将结束其等待状态,并再次锁定互斥锁以同步其对计数器的访问。

一个重要问题是,条件变量可能由不相关的线程发出信号,这称为伪唤醒。为了避免因伪唤醒而导致错误,在等待时会检查条件。当条件变量发出信号时,会再次检查条件。

如果发生伪唤醒且计数器为零(条件检查返回 false),则等待将恢复。

另一个线程通过运行increment\_counter来增加计数器。当计数器具有所需的值(在本例中,该值为10),就会向等待线程条件变量发出信号。

有两个函数用于向条件变量发送信号:

\begin{itemize}
\item
cv.notify\_one():仅向其中一个等待线程发送信号

\item
cv.notify\_all():向所有等待线程发出信号
\end{itemize}

本节中,介绍了条件变量,并看到了使用条件变量进行同步的简单示例,以及在某些情况下如何简化同步/等待代码。现在,将注意力转向使用互斥锁和两个条件变量实现同步队列。
























