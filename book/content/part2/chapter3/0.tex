正如前两章中了解到的,线程是程序中最小、最轻量的执行单元。每个线程负责处理由操作系统调度程序在分配的 CPU 资源上运行的一系列指令定义的独特任务。管理程序中的并发性以最大限度地提高 CPU 资源的整体利用率时,线程起着至关重要的作用。

程序启动过程中,内核将执行权交给进程后,C++ 运行时会创建主线程并执行 main() 函数。此后,可以创建其他线程,将程序拆分为可并发运行并共享资源的不同任务。这样,程序就可以处理多个任务,从而提高效率和响应能力。

本章中,将介绍如何使用现代 C++ 功能创建和管理线程的基础知识。后续章节中,将介绍 C++ 锁同步原语(互斥锁、信号量、栅栏和自旋锁)、无锁同步原语(原子变量)、协调同步原语(条件变量),以及使用 C++ 解决或避免使用并发或多线程时的潜在问题(竞争条件或数据竞争、死锁、活锁、饥饿、超额分配、负载平衡和线程耗尽)的方法。

本章中,将讨论以下主要主题:

\begin{itemize}
\item
如何在 C++ 中创建、管理和取消线程

\item
如何向线程传递参数并从线程获取结果

\item
如何让线程休眠或让其他线程执行

\item
jthread 对象是什么
\end{itemize}





