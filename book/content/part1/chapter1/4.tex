
指标是一种测量方法,可以帮助我们了解系统的运行情况并比较不同的改进方法。以下是一些常用于评估系统并行性的指标和公式。

\mySubsubsection{1.4.1.}{并行度}

并行度 (DOP) 是衡量计算机同时执行的操作数量的指标,可用于描述并行程序和多处理器系统的性能。

计算 DOP 时,可以使用同时执行的最大操作数,测量没有瓶颈或依赖性的理想情况。或者,可以使用给定时间点的平均操作数或同时执行的操作数,反映系统实现的实际 DOP。可以使用分析器和性能分析工具来测量特定时间段内的线程数,从而进行近似计算。

DOP 不是一个常数,它是一个在应用程序执行过程中发生变化的动态指标。

例如,考虑一个处理多个文件的脚本工具。这些文件可以按顺序或同时处理,从而提高效率。如果有一台有 N 个核的机器,并且想要处理 N 个文件,可以为每个核分配一个文件。

按顺序处理所有文件的时间如下:

\begin{center}
$t_{total} = t_{file1} + t_{file2} + t_{file3} + ... + t_{fileN} \widetilde{=} N · avg(t_{file}) $
\end{center}

并且,并行处理的时间为:

\begin{center}
$t_{total} = max(t_{file1}, t_{file2}, t_{file3}, ..., t_{fileN}) $
\end{center}

DOP为N,即主动处理单独文件的核数。

并行化所能实现的加速比有一个理论上限,由阿姆达尔定律给出。

\mySubsubsection{1.4.2.}{阿姆达尔定律}

并行系统中,可以认为将 CPU 核心数量增加一倍可以使程序运行速度提高一倍,从而将运行时间减半,但并行化带来的加速并不是线性的。在一定数量的核心之后,由于上下文切换、内存分页等不同情况,运行时间不再减少。

阿姆达尔定律公式计算了任务并行化后理论上的最大加速比:

\begin{center}
$S_{max}(s) = \frac{s}{s + p(1 - s)} = \frac{1}{1 - p + \frac{p}{s}}$
\end{center}

s 是改进部分的加速因子, p 是可并行化部分占整个流程的比例。因此, 1-p 表示不可并行化任务(瓶颈或顺序部分)的比例,而 p/s 表示可并行化部分实现的加速,最大加速受任务的顺序部分限制。可并行化任务的比例越大(p 接近 1),最大加速就越高,直至达到加速因子 (s)。另一方面,当顺序部分变大(p 接近 0)时, $S_{max}$ 趋向于 1,则不可能有任何改进。

\myGraphic{0.8}{content/part1/chapter1/images/9.png}{图1.9:处理器数量和可并行部件百分比的加速限制}

并行系统中的关键路径由最长的依赖计算链定义。由于关键路径几乎不可并行,因此它定义了顺序部分,从而决定了程序可以实现的更快运行时间。例如,如果一个进程的顺序部分占运行时间的 10\%,那么可并行化部分的比例为 p=0.9。在这种情况下,无论有多少个处理器可用,潜在的加速都不会超过 10 倍。

\mySubsubsection{1.4.3.}{古斯塔夫森定律}

阿姆达尔定律公式仅适用于固定规模的问题和不断增加的资源。当使用较大的数据集时,可并行化部分所花费的时间增长速度比顺序部分要快得多。而古斯塔夫森定律公式没那么悲观,也更准确,它考虑了固定的执行时间和使用资源不断增加的问题规模。

古斯塔夫森定律公式计算使用 p 个处理器所获得的加速比如下:

\begin{center}
$S_p = p + (1 - f) · p$
\end{center}

p 是处理器的数量, f 是保持连续的任务比例。因此, (1-f)*p 表示将 (1-f) 任务分布在p 个处理器上进行并行化所实现的加速, p 表示增加资源时所做的额外工作。古斯塔夫森定律公式表明,降低 f 时,加速比受并行化影响,而增加 p 时,加速比受可扩展性影响。

与阿姆达尔定律一样,古斯塔夫森定律公式也是一种近似值,在衡量并行系统的改进时提供了有价值的视角。其他因素也会降低效率,例如:处理器之间的开销通信或内存和存储限制。




































