上一章中,介绍了Promise、 Future和打包任务。在介绍打包任务时,提到std::async提供了一种更简单的方式来实现同样的结果,代码更少,更干净、更简洁。

异步函数 (std::async) 是一个异步运行可调用对象的函数模板,还可以通过传递一些定义启动策略的标志来选择执行方法。它是处理异步操作的强大工具,但其自动管理和对执行线程缺乏控制等,使其不适合某些需要细粒度控制或取消的任务。

在本章中,我们将讨论以下主要主题:

\begin{itemize}
\item
什么是异步函数以及如何使用?

\item
有哪些不同的启动政策?

\item
与以前的方法,特别是打包任务有什么不同?

\item
使用 std::async 有哪些优点和缺点?

\item
实际场景和示例
\end{itemize}