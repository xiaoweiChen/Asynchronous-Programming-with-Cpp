

同步原语是管理多线程编程并发访问共享资源的重要工具。有几种同步原语,每种都有各自的特定用途和特征:

\begin{itemize}
\item
互斥锁:用于强制对代码的关键部分进行独占访问。线程可以锁定互斥锁,从而阻止其他线程进入受保护的部分,直到互斥锁解锁。互斥锁保证在给定时间内只有一个线程可以执行关键部分,从而确保数据完整性并避免竞争条件。

\item
信号量:比互斥量更通用,可用于更广泛的同步任务,包括线程之间的信号发送。信号量维护一个整数计数器,线程可以将其递增(信号发送)或递减(等待)。信号量允许更复杂的协调模式,例如:计数信号量(用于资源分配)和二进制信号量(类似于互斥量)。

\item
条件变量:条件变量用于根据特定条件进行线程同步。线程可以阻塞(等待)条件变量,直到特定条件为真。其他线程可以向条件变量发送信号,从而唤醒等待的线程并继续执行。条件变量通常与互斥锁一起使用,以实现更细粒度的同步,并避免忙等待。

\item
附加同步原语:除了前面讨论的核心同步原语之外,还有其他几种同步机制:

\begin{itemize}
\item
栅栏:允许一组线程同步执行,确保所有线程在继续执行之前都达到某个点

\item
读写锁:读写锁提供了一种控制共享数据并发访问的方法,允许多个读取器但一次只能有一个写入器

\item
自旋锁:自旋锁是一种互斥锁,涉及忙等待,持续检查内存位置,直到可用
\end{itemize}
\end{itemize}

在第 4 章和第 5 章中,将深入了解 C++ 标准模板库 (STL) 中实现的同步原语,以及使用它们的示例。

\mySubsubsection{2.4.1.}{选择正确的同步原语}

选择适当的同步原语取决于应用程序的具体要求,以及所访问的共享资源的性质。以下是一些准则:

\begin{itemize}
\item
互斥锁:需要对关键部分进行独占访问以确保数据完整性并防止竞争条件时,请使用互斥锁

\item
信号量:需要更复杂的协调模式时使用信号量,例如:资源分配或线程之间的信号传递

\item
条件变量:线程需要等待特定条件变为真才能继续执行时,请使用条件变量
\end{itemize}

有效使用同步原语,对于开发安全高效的多线程程序至关重要。通过了解不同同步机制的用途和特性,开发人员可以根据其特定需求选择最合适的原语,实现可靠且和可预测的并发程序。



























