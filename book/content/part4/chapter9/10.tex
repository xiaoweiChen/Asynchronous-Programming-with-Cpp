strand 是严格顺序且非并发的完成处理程序调用。使用 strand,可以使用互斥锁或本书前面介绍的其他同步机制对异步操作进行排序,而无需显式锁定。 strand 可以是隐式的,也可以是显式的。

如本章前面所示,如果我们仅从一个线程执行 boost::asio::io\_context::run(),则所有事件处理程序将在隐式链中执行,因为它们将按顺序逐个排队并从 I/O 执行上下文触发。

另一种隐式链发生在存在链式异步操作时,其中一个异步操作会调度下一个异步操作,依此类推。本章中之前的一些示例已经使用了这种技术,但这里还有另一种。

在这种情况下,如果没有错误,计时器将在 handle\_timer\_expiry() 事件处理程序中通过递归设置到期时间并调用 async\_wait() 方法来不断重新启动自身:

\begin{cpp}
#include <boost/asio.hpp>
#include <chrono>
#include <iostream>

using namespace std::chrono_literals;

void handle_timer_expiry(boost::asio::steady_timer& timer,
                         int count) {
    std::cout << "Timer expired. Count: " << count
    << std::endl;
    timer.expires_after(1s);
    timer.async_wait([&timer, count](
                const boost::system::error_code& ec) {
        if (!ec) {
            handle_timer_expiry(timer, count + 1);
        } else {
            std::cerr << "Error: " << ec.message()
                      << std::endl;
        }
    });
}

int main() {
    boost::asio::io_context io_context;
    boost::asio::steady_timer timer(io_context, 1s);
    int count = 0;
    timer.async_wait([&](
                const boost::system::error_code& ec) {
        if (!ec) {
            handle_timer_expiry(timer, count);
        } else {
            std::cerr << "Error: " << ec.message()
                      << std::endl;
        }
    });
    io_context.run();
    return 0;
}
\end{cpp}

运行此示例将每秒打印 "Timer expired. Count: <number>" 行,并且计数器在每一行上增加。

如果某些工作需要序列化但这些方法不合适,我们可以使用 boost::asio::strand 或其 I/O 上下文执行对象特化 boost::asio::io\_context::strand 来使用显式 strand。使用这些 strand 对象发布的工作将按照进入 I/O 执行上下文队列的顺序序列化其处理程序执行。

在以下示例中,我们将创建一个记录器,将多个线程的写入操作序列化为单个日志文件。

我们将从四个线程记录消息,每个线程写入五条消息。我们期望输出正确,但这次不使用任何互斥锁或其他同步机制。
让我们首先定义 Logger 类:

\begin{cpp}
#include <boost/asio.hpp>
#include <chrono>
#include <fstream>
#include <iostream>
#include <memory>
#include <string>
#include <thread>
#include <vector>

using namespace std::chrono_literals;

class Logger {
    public:
    Logger(boost::asio::io_context& io_context,
            const std::string& filename)
        : strand_(io_context), file_(filename
        , std::ios::out | std::ios::app)
    {
        if (!file_.is_open()) {
            throw std::runtime_error(
            "Failed to open log file");
        }
    }

    void log(const std::string message) {
        strand_.post([this, message](){
            do_log(message);
        });
    }

private:
    void do_log(const std::string message) {
        file_ << message << std::endl;
    }

    boost::asio::io_context::strand strand_;
    std::ofstream file_;
};
\end{cpp}

Logger 构造函数接受 I/O 上下文对象(用于创建 strand 对象 (boost::asio::io\_context::strand) )和 std::string(指定用于打开日志文件或创建日志文件(如果不存在)的日志文件名)。
日志文件已打开以附加新内容。如果在构造函数完成之前文件未打开(意味着访问或创建文件时出现问题),则构造函数将引发异常。

记录器还提供了接受 std::string 的公共 log() 函数,指定一条消息作为参数。此函数使用 strand 将新工作发布到 io\_context 对象中。它通过使用 lambda 函数来实现这一点,通过值捕获记录器实例(对象 this)和消息,并调用私有 do\_log() 函数,其中 std::fstream 对象用于将消息写入输出文件。

程序中只会有一个 Logger 类的实例,由所有线程共享。这样,线程将写入同一个文件。

让我们定义一个 worker() 函数,每个线程将运行该函数以将 num\_messages\_per\_thread 消息写入输出文件:

\begin{cpp}
void worker(std::shared_ptr<Logger> logger, int id) {
    for (unsigned i=0; i < num_messages_per_thread; ++i) {
        std::ostringstream oss;
        oss << "Thread " << id << " logging message " << i;
        logger->log(oss.str());
        std::this_thread::sleep_for(100ms);
    }
}
\end{cpp}

此函数接受指向 Logger 对象的共享指针和线程标识符。它使用前面解释过的 Logger 的公共log() 函数打印所有消息。

为了交错线程执行并严格测试线程的工作方式,每个线程在写入每条消息后将休眠 100 毫秒。

最后,在 main() 函数中,我们启动一个 io\_context 对象和一个工作保护,以避免提前退出 io\_context。然后,创建一个指向 Logger 实例的共享指针,并传递前面解释的必要参数。

使用 worker() 函数并将共享指针传递给记录器以及每个线程的唯一标识符来创建线程池(std::jthread 对象的向量)。此外,还会将运行 io\_context.run() 函数的线程添加到线程池中。

在下面的例子中,因为我们知道所有消息将在不到两秒的时间内打印出来,所以我们使用 io\_context.run\_for(2s) 让 io\_context 仅在那段时间内运行。

当 run\_for() 函数退出时,程序会向控制台打印 Done! 并完成:

\begin{cpp}
const std::string log_filename = "log.txt";
const unsigned num_threads = 4;
const unsigned num_messages_per_thread = 5;

int main() {
    try {
        boost::asio::io_context io_context;
        auto work_guard = boost::asio::make_work_guard(
                                 io_context);
        std::shared_ptr<Logger> logger =
            std::make_shared<Logger>(
                io_context, log_filename);

        std::cout << "Logging "
                  << num_messages_per_thread
                  << " messages from " << num_threads
                  << " threads\n";

        std::vector<std::jthread> threads;
        for (unsigned i = 0; i < num_threads; ++i) {
            threads.emplace_back(worker, logger, i);
        }

        threads.emplace_back([&]() {
            io_context.run_for(2s);
        });
    } catch (std::exception& e) {
        std::cerr << "Exception: " << e.what() << '\n';
    }

    std::cout << "Done!" << std::endl;
    return 0;
}
\end{cpp}

运行此示例将显示以下输出:

\begin{shell}
Logging 5 messages from 4 threads
Done!
\end{shell}

这是生成的 log.txt 日志文件的内容。由于每个线程的睡眠时间相同,因此所有线程和消息都按顺序排列:

\begin{shell}
Thread 0 logging message 0
Thread 1 logging message 0
Thread 2 logging message 0
Thread 3 logging message 0
Thread 0 logging message 1
Thread 1 logging message 1
Thread 2 logging message 1
Thread 3 logging message 1
Thread 0 logging message 2
Thread 1 logging message 2
Thread 2 logging message 2
Thread 3 logging message 2
Thread 0 logging message 3
Thread 1 logging message 3
Thread 2 logging message 3
Thread 3 logging message 3
Thread 0 logging message 4
Thread 1 logging message 4
Thread 2 logging message 4
Thread 3 logging message 4
\end{shell}

如果我们删除工作保护,日志文件将只包含以下内容:

\begin{shell}
Thread 0 logging message 0
Thread 1 logging message 0
Thread 2 logging message 0
Thread 3 logging message 0
\end{shell}

发生这种情况的原因是,第一批工作被迅速发布并从每个线程排队到 io\_object 中,但是 io\_object 在完成调度工作保护并在发布第二批消息之前通知完成处理程序之后退出。

如果我们还删除工作线程中的 sleep\_for() 指令,那么现在日志文件的内容如下:

\begin{shell}
Thread 0 logging message 0
Thread 0 logging message 1
Thread 0 logging message 2
Thread 0 logging message 3
Thread 0 logging message 4
Thread 1 logging message 0
Thread 1 logging message 1
Thread 1 logging message 2
Thread 1 logging message 3
Thread 1 logging message 4
Thread 2 logging message 0
Thread 2 logging message 1
Thread 2 logging message 2
Thread 2 logging message 3
Thread 2 logging message 4
Thread 3 logging message 0
Thread 3 logging message 1
Thread 3 logging message 2
Thread 3 logging message 3
Thread 3 logging message 4
\end{shell}

以前,内容是按消息标识符排序的,现在则按线程标识符排序。这是因为现在,当一个线程启动并运行 worker() 函数时,它会立即发布所有消息,没有任何延迟。因此,第一个线程(线程 0)在第二个线程有机会执行此操作之前将其所有工作排入队列,依此类推。

继续进一步的实验,当我们将内容发布到 strand 中时,我们使用以下指令按值捕获了记录器实例和消息:

\begin{cpp}
strand_.post([this, message]() { do_log(message); });
\end{cpp}

通过值捕获允许运行 do\_log() 的 lambda 函数使用所需对象的副本,并使它们保持活动状态,正如我们在本章前面讨论对象生命周期时所评论的那样。

假设,由于某种原因,我们决定使用以下指令通过引用捕获:

\begin{cpp}
strand_.post([&]() { do_log(message); });
\end{cpp}

然后,生成的日志文件将包含不完整的日志消息,甚至不正确的字符,因为记录器正在从属于 do\_log() 函数执行时不再存在的消息对象的内存区域进行打印。

因此,始终假设异步变化;操作系统可能会执行一些我们无法控制的更改,所以始终要知道什么在我们的控制范围内,最重要的是,什么不在我们的控制范围内。

最后,除了使用 lambda 表达式并通过值捕获 this 和 message 对象外,我们还可以使用 std::bind ,如下所示:

\begin{cpp}
strand_.post(std::bind(&Logger::do_log, this, message));
\end{cpp}

现在让我们学习如何通过使用协程来简化我们之前实现的回显服务器,并通过添加从客户端退出连接的命令来改进它。



















