无法确保软件产品没有错误,因此时不时就会出现错误。这时日志记录和调试就必不可少。

日志记录和调试对于识别和诊断软件系统中的问题至关重要。它们提供了对代码运行时行为的可见性,帮助开发人员跟踪错误、监控性能并了解执行流程。通过有效地使用日志记录和调试,开发人员可以检测错误、解决意外行为并提高整体系统稳定性和可维护性。

撰写本章时,假设读者们已经熟悉使用调试器调试 C++ 程序,并且了解一些基本的调试器命令和术语,例如断点、观察器、框架或堆栈跟踪。要复习这些知识,可以参考本章末尾的“扩展阅读”部分提供的参考资料。

本章中,将讨论以下主要主题:

\begin{itemize}
\item
如何使用日志记录来发现错误

\item
如何调试异步软件
\end{itemize}