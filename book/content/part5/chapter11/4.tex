本章中,介绍如何使用日志记录和调试异步程序。

首先使用日志来发现正在运行的软件中的问题,并展示了使用 spdlog 日志库检测死锁的实用性。还讨论了许多其他库,并描述了可能适合特定场景的相关功能。

但是,并非所有错误都可以通过使用日志来发现,有些错误可能只能在软件开发生命周期的后期才被发现,当生产中出现一些问题时,甚至在处理程序崩溃和事故时也是如此。调试器是检查正在运行或崩溃的程序、了解其代码路径和查找错误的有用工具。引入了几个示例和调试器命令来处理通用代码,但也特别适用于多线程和异步软件、条件竞争和协程。此外,还引入了 rr 调试器,展示了将反向调试纳入开发人员工具箱的潜力。

下一章中,将介绍使用消杀器和测试技术,来提高异步程序的运行时间和资源使用率的性能和优化技术。