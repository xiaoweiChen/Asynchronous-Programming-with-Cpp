本章介绍了原子类型和操作、 C++内存模型以及 SPSC 无锁队列的基本实现。

以下是我们所研究内容的总结:

\begin{itemize}
\item
C++ 标准库原子类型和操作、它们是什么,以及如何通过一些示例展示如何使用。

\item
C++ 内存模型,尤其是它定义的不同内存顺序。请记住,这是一个非常复杂的主题,本节只是对它的基本介绍。

\item
如何实现基本的 SPSC 无锁队列。如前所述,我们将在第 13 章中演示如何提高其性能。性能改进操作的示例包括消除错误共享(当两个变量位于同一缓存行中并且每个变量仅由一个线程修改时会发生这种情况)和减少真共享。如果现在不了解其中何内容,请不要担心,我们将在稍后介绍它并演示如何运行性能测试。
\end{itemize}

这是对原子操作的基本介绍,用于同步来自不同线程的内存访问。某些情况下,原子操作的使用相当容易,类似于收集统计数据和简单的计数器。更复杂的应用程序(例如:SPSC 无锁队列的实现),需要对原子操作有更深入的了解。本章中看到的内容有助于理解基础知识,并为进一步研究这个复杂的主题奠定基础。

下一章中,将介绍 C++ 中异步编程的两个基本构建块,即 promise 和 future。