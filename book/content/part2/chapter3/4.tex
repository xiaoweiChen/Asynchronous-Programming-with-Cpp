
线程本地存储 (TLS) 是一种内存管理技术,允许每个线程拥有私有变量实例。因为此技术消除了访问这些变量的同步机制开销,允许线程存储其他线程无法访问的线程特定数据,从而避免竞争条件并提高性能。

TLS 由操作系统实现,可使用自 C++11 起可用的 \verb|thread_local| 关键字访问。 \verb|thread_local| 提供了一种统一的方法来使用许多操作系统的 TLS 功能,并避免使用特定于编译器的语言扩展来访问 TLS 功能(此类扩展的一些示例是 TLS Windows API、 \verb|__declspec(thread)| MSVC 编译器语言扩展或 \verb|__thread| GCC 编译器语言扩展)。

要将 TLS 与不支持 C++11 或更新版本的编译器一起使用,请使用 Boost.Library。它提供了\verb|boost::thread_specific_ptr| 容器,该容器实现了可移植的 TLS。

线程局部变量可以声明如下:

\begin{itemize}
\item
全局

\item
在命名空间中

\item
作为类静态成员变量

\item
在函数内部;与使用 static 关键字分配的变量具有相同的效果,变量在程序的整个生命周期内分配,并且其值会在下一次函数调用中保留
\end{itemize}

以下示例显示三个线程使用不同的参数调用 multiplyByTwo 函数。此函数将 val 线程局部变量的值设置为参数值,将其乘以 2,然后输出到控制台:

\begin{cpp}
#include <iostream>
#include <syncstream>
#include <thread>

#define sync_cout std::osyncstream(std::cout)

thread_local int val = 0;

void setValue(int newval) { val = newval; }

void printValue() { sync_cout << val << ' '; }
void multiplyByTwo(int arg) {
    setValue(arg);
    val *= 2;
    printValue();
}

int main() {
    val = 1; // Value in main thread
    std::thread t1(multiplyByTwo, 1);
    std::thread t2(multiplyByTwo, 2);
    std::thread t3(multiplyByTwo, 3);
    t1.join();
    t2.join();
    t3.join();
    std::cout << val << std::endl;
}
\end{cpp}

运行此代码片将显示以下输出:

\begin{shell}
2 4 6 1
\end{shell}

这里,可以看到每个线程都对其输入参数进行操作,导致 t1 打印 2、 t2 输出 4 并且 t 3 打印 6。运行主函数的主线程还可以访问其线程局部变量 val,该变量的值在程序启动时设置为 1,但仅在退出程序之前在主函数末尾输出到控制台时使用。

TLS 也存在一些缺点。 TLS 会增加内存使用量,每个线程都会创建一个变量,在资源受限的环境中可能会出现问题。与常规变量相比,访问 TLS 变量可能会产生一些开销。这对于性能至关重要的软件来说可能有风险。

接下来,利用迄今为止学到的诸多技术,来构建一个计时器。



